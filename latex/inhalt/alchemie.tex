{\let\clearpage\relax\chapter{Alchemie}}
Diese Kapitel beschäftigt sich mit der Alchemie. Der komplette Entstehungsprozess von alchemistischen Gebräue, wie Tränke, Öle und Bomben wird im Detail beschrieben. Dazu gehört das Suchen und Sammeln von Ingredienzen, die Gewinnung von Wirkstoffen und natürlich die Herstellung von Tränken, Öle und Bomben. Der Erfolg von allen Aktionen muss durch Talentproben bestätigt werden. Wie diese im Detail aussehen wird in den folgenden Kapitel erläutert. 

Neben den Beschreibungen der einzelnen Aktionen (sammeln von Ingredienzen, Wirkstoffe extrahieren, etc...) gibt es ein Herbarium sowie eine Tränke-, Öle- und Bomben-Liste. Im Herbarium sind alle Zutaten wie Pflanzen und Mineralien aufgeführt, die für die Herstellung von alchemistischen Gebräue und mehr verwendet werden können. Manche Ingredienzen haben nach alchemistischer Aufbereitung auch eigenständige Effekte. Tränke, Öle und Bomben werden aus einer Kombination von spezifischen Ingredienzen und Wirkstoffen hergestellt. Wirkstoffe werden aus Pflanzen und Mineralien extrahiert. 

\section{Suchen und Sammeln von Ingredienzen}

\subsection{Herbarium}

\section{Wirkstoffgewinnung}
\subsection{Wirkstoffe}

\section{Herstellung alchemistischer Gebräue}
Damit man alchemistische Gebräue wie Tränke, Öle und Bomben herstellen kann benötigt der Held den Vorteil \textit{Alchemie}. Alchemistische Gebräue besitzen eine Schwierigkeitsstufe - leicht, mittel oder schwer. Den \textit{Alchemie}-Vorteil gibt es in drei Ausbaustufen. Mit der ersten Stufe können nur leichte Gebräue hergestellt werden. Mit der zweiten Stufe zusätzlich mittlere Gebräue und mit der dritten Stufe, \textit{Alchemie III}, alle. 

Zur Herstellung von leichten Tränke und Öle benötigt man irgendein Alkohol. Für mittlere Tränke und Öle wird ein qualitativ hochwertiger Alkohol benötigt und für schwere Tränke und Öle braucht man \textit{Weiße Möwe}. \textit{Weiße Möwe} ist ein einfacher Trank, der genauso wie alle anderen Tränke hergestellt werden kann. Ob die Herstellung von alchemistischen Gebräue gelingt entscheidet eine Probe auf \textit{Alchemie}. Diese Probe kann durch den \textit{Alchemie}-Vorteil erleichtert werden.\footnote{siehe Kapitel \ref{chap:vorteile} }

\section{Tränke}
Folgende Tränke stehen den Spielern zur Verfügung. 

\begin{tabular}{|p{3.5cm}|p{7cm}|p{1.5cm}|p{3cm}|}
\hline
\textbf{Name} & \textbf{Wirkung} & \textbf{WD} & \textbf{Schwierigkeit} \\

\hline
\textbf{Katze} & Nachtsicht. Man sieht alles in grau, weiß. & 8h & Mittel \\

\hline
\textbf{Weisser Honig} & Vergiftungen (und Effekte eingenommener Tränke) werden neutralisiert & - & Leicht \\

\hline
\textbf{Frauentränen} & Neutralisiert Effekte von Alkohol. Trinkspiel! & - & Leicht \\

\hline
\textbf{Genesungstrank} & Stoppt Blutungen und heilt leichte Wunden. & - & Mittel \\

\hline
\textbf{Heiltrank} & Regeneriert +1LP pro Minute. (Insgesamt +10LP) & 10min & Mittel \\

\hline
\textbf{Parfüm} & Z.B. Als Geschenk. & - & Leicht \\

\hline
\textbf{Fisstech} & Harte Droge. (Vergleichbar mit Heroin).  & 8h & Schwer \\

\hline
\textbf{Weisse Möwe} & Hexer-Haluzinogen. Gute Basis für Hexer-Tränke. & 1h & Hexer \\

\hline
\textbf{Graues Blut} & Starkes Betäubungsgift. Wirkt nach wenigen Sekunden (0AK). Kann nicht nur getrunken werden, sonder auch auf Pfeilspitzen oder Klingen geschmiert werden. & - & Schwer \\

\hline
\textbf{Schwarzes Blut} & Starkes tödliches Gift. Wirkt nach wenigen Minuten (12-24AK). Kann nicht nur getrunken werden, sonder auch auf Pfeilspitzen oder Klingen geschmiert werden. & - & Schwer \\

\hline
\textbf{Milde Schwalbe} & Milde Version von Schwalbe. Heilt +2LP pro AK. & 5AK & Schwer \\

\hline
\textbf{Schwalbe} & Heilt +4LP pro AK. & 3AK & Hexer \\

\hline
\end{tabular}


\section{Öle}
Wie bei Tränken auch, muss ein Spieler das Rezept eines Öls kennen, bevor er es herstellen kann. Öle können problemlos von allen Spielern verwendet werden. Öle werden auf Klingen geschmiert um ihnen Boni gegen bestimmte Monster zu geben. Sie haften 24h lang an der Klinge. Eingeschmierte Klingen können mit Alkohol gereinigt werden (entfernt aufgetragene Öle).

Wenn jemand versucht einen Trank oder Öl herzustellen, der 0 Punkte in \textit{Alchemie} hat muss er bzw. der SL eine geheime Probe durchführen.

\begin{tabular}{|p{4cm}|p{8.5cm}|p{3cm}|}
\hline
\textbf{Name} & \textbf{Wirkung} & \textbf{Schwierigkeit} \\

\hline
\textbf{Geisteröl} & Geister erleiden +5 mehr Schaden. & Leicht \\

\hline
\textbf{Orgoidenöl} & Orgoiden erleiden +5 mehr Schaden. & Mittel \\

\hline
\textbf{Draconidenöl} & Daconiden erleiden +5 mehr Schaden. & Mittel \\

\hline
\textbf{Insektoidenöl} & Insektoiden erleiden +5 mehr Schaden. & Leicht \\

\hline
\textbf{Konstruktöl} & Konstrukte erleiden +5 mehr Schaden. & Leicht \\

\hline
\textbf{Nekrophagenöl} & Nekrophagen erleiden +5 mehr Schaden. & Leicht \\

\hline
\textbf{Hybridenöl} & Hybride erleiden +5 mehr Schaden. & Leicht \\

\hline
\textbf{Bestienöl} & Bestien erleiden +5 mehr Schaden. & Mittel \\

\hline
\textbf{Argentina} & Gegner erleiden +2 mehr Schaden. & Schwer \\

\hline
\textbf{Braunöl} & Gegner bluten. -1LP pro Aktion. Gut gegen Endgegner. & Schwer \\

\hline
\textbf{Henkersgiftserum} & Gegner werden vergiftet. -1LP Giftschaden pro Aktion. & Schwer \\

\hline
\end{tabular}

\section{Bomben}