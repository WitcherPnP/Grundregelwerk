{\let\clearpage\relax\chapter{Alchemie}}
Diese Kapitel beschäftigt sich mit der Alchemie. Der komplette Entstehungsprozess von alchemistischen Gebräue, wie Tränke, Öle und Bomben wird im Detail beschrieben. Dazu gehört das Suchen und Sammeln von Ingredienzen, die Gewinnung von Wirkstoffen und natürlich die Herstellung von Tränken, Öle und Bomben. Der Erfolg von allen Aktionen muss durch Talentproben bestätigt werden. Wie diese im Detail aussehen wird in den folgenden Kapitel erläutert. 

Neben den Beschreibungen der einzelnen Aktionen (sammeln von Ingredienzen, Wirkstoffe extrahieren, etc...) gibt es ein Herbarium sowie eine Tränke-, Öle- und Bomben-Liste. Im Herbarium sind alle Zutaten wie Pflanzen und Mineralien aufgeführt, die für die Herstellung von alchemistischen Gebräue und mehr verwendet werden können. Manche Ingredienzen haben nach alchemistischer Aufbereitung auch eigenständige Effekte. Tränke, Öle und Bomben werden aus einer Kombination von spezifischen Ingredienzen und Wirkstoffen hergestellt. Wirkstoffe werden aus Pflanzen und Mineralien extrahiert. 

Bevor die Regeln der verschiedenen Aktionen erklärt werden, ist es wichtig zu Verstehen, was es mit den Zutaten, ihren Eigenschaften und Konzentrationswerten auf sich hat. Dies ist wichtig um zu verstehen, wie das Ergebnis bestimmte Eigenschaften erhält.

\section{Ingredienzen}
Ingredienzen bezeichnen alchemistische Zutaten, die u.A. für die Herstellung von Tränken, Ölen und Bomben verwendet werden können. \textit{Alchemistische Zutaten} meint Zutaten deren Hauptzweck darin besteht, für alchemistische Gebräue eingesetzt zu werden. Alltägliche Dinge wie beispielsweise Obst kann ebenfalls als Zutat für manche Gebräue dienen, ist in diesem Dokument allerdings nicht mit \textit{Ingredienz} gemeint.

\subsection{Arten}
Man unterscheiden die Ingredienzen in zwei Kategorien. Pflanzen und Mineralien. Beides kommt in der Natur vor und kann von Helden eingesammelt werden. Außerdem kann der Held sowohl aus Pflanzen als auch aus Mineralien Wirkstoffe extrahieren, die für spezifische Gebräue benötigt werden. Jede Ingredienz besteht aus einem Hauptwirkstoff und optional aus einem Sekundärwirkstoff. Den Wirkstoffen und deren Gewinnung sind allerdings eigene Kapitel gewidmet.

\subsubsection{Pflanzen}
Zu den Pflanzen gehören z.B. Kräuter, Sträucher, Früchte, Wurzeln, usw. Pflanzliche Ingredienzen lassen sich auch weiter einteilen. Z.B. Wurzeln, Blüten, Kräuter, u.v.m. Pflanzen kann man an den unterschiedlichsten Orten finden. Es gibt häufig-, , mittelhäufig- und selten vorkommende Pflanzen.

\subsubsection{Mineralien}
Mineralien sind spezielle Ingredienzen, die vor allem unterirdisch oder in Gestein anzutreffen sind. Es gibt aber auch Ausnahmen. Mineralien werden genauso wie Pflanzen auch für die Herstellung von bestimmten Tränken, Ölen und Bomben benötigt. Mineralien müssen flüssig oder in Pulverform sein, damit sie weiter verwendet werden können. Wenn man kleine Gesteinsbrocken mit einem Stößel zerkleinert erhält man Pulver. Dieses Pulver kann auch in Wasser aufgelöst werden, z.B. um es in Flaschen aufbewahren zu können.

\subsection{Aufbewahrung und Haltbarkeit}
Alle Ingredienzen müssen entsprechend gelagert werden damit sie bei der Reise nicht kaputt gehen. Pflanzen halten nachdem sie gepflückt wurden nur zwei Tage an der freien Luft bevor sie verwelken. In kleinen Fiolen (kleine Glasfläschen) halten Pflanzen bis zu drei Tage bevor sie für die alchemistische Weiterverarbeitung nutzlos werden. 

Am besten sind alle Ingredienzen in einem Herbarium aufbewahrt. Dazu müssen Pflanzen
gepresst werden um sie haltbar zu machen. Mineralien können als Pulver in winzigen Beuteln ebenfalls im Herbarium aufbewahrt werden. Alternativ auch separat in Glasflaschen oder einfach als Gesteinsbrocken. Sowohl die Pflanzen als auch die Mineralien (in Pulverform) können verunreinigt werden, wenn sie nicht ordnungsgemäß aufbewahrt werden. In diesem Fall sind sie nicht mehr für die Alchemie zu gebrauchen.

\subsection{Eigenschaften}
Jede Substanz, Ingredienzen, Tränke, Öle, Obst, Getränke haben mehrere für die Alchemie relevanten Eigenschaften. Bei Substanzen, die laut dem Grundregelwerk keine Eigenschaften festlegen, entscheidet der SL über die Eigenschaften. Der Konzentrationswert der jeweiligen Eigenschaften sollte jedoch zwischen 0 und 3 liegen. Doch zuerst sollten geklärt werden, um welche Eigenschaften es sich handelt. Es gibt Grundeigenschaften, die jede Substanz hat und Spezialeigenschaften, die nur manche Substanzen haben.

\subsubsection{Grundeigenschaften}
Es gibt drei Grundeigenschaften:
\begin{itemize}
\item \textbf{Farbe} - (rot, grün, schwarz, ...)
\item \textbf{Geruch} - (süß, Lavendel, bitter, ekelhaft, ...)
\item \textbf{Geschmack} - (süß, sauer, fruchtig, ...)
\end{itemize}

Eine Grundeigenschaft kann auch neutral sein. Bei der Farbe wäre das \textit{transparent} und beim Geruch und Geschmack \textit{neutral}. 

\subsubsection{Spezialeigenschaften}
\label{chap:spezialeigenschaften}
Manche Ingredienzen haben neben den Grundeigenschaften auch noch weitere Eigenschaften. Anders als die Grundeigenschaften haben Spezialeigenschaften keinen Konzentrationswert. Mögliche Spezialeigenschaften sind:

\begin{table}[h]
\begin{center}
\begin{tabular}{|p{4,5cm}|p{11cm}|}
\hline
\textbf{Spezialeigenschaft} & \textbf{Beschreibung} \\ \hline
\textbf{Gift} & Tödliches Gift \\ \hline
\textbf{Betäubungsgift} & Spezielle Form von Gift, die eine Betäubende/Lähmende Wirkung hat. \\ \hline
\textbf{Schlafmittel} & Hat eine schläfrig machende Wirkung. \\ \hline
\textbf{Aufputschmittel} & Hat eine wach machende Wirkung. Kann benutzt werden um die Eigenschaft \textit{Schlafmittel} zu neutralisieren. \\ \hline
\textbf{Giftneutralisierend} & Kann die Eigenschaft \textit{Gift} und \textit{Betäubungsgift} neutralisieren. \\ \hline
\textbf{Rausch} & Bewusstseins verändernde Droge. Wie bei Alkohol. \\ \hline
\textbf{Rauschlindernd} & Neutralisiert Effekte von Alkohol \\ \hline
\textbf{Haluzinogen} & Verursacht Haluzinationen. \\ \hline
\textbf{Explosiv} & Explodiert bei Hitze. \\ \hline
\textbf{Instabil} & Explodiert bei einwirkender kinetischer Energie. Z.B. beim Aufprall, wenn es runterfällt. \\ \hline
\textbf{Ätzend} & Frisst sich durch dünne Holz und Metallplatten. \\ \hline
\textbf{Gesund} & Hat eine heilende Wirkung. \\ \hline
\end{tabular}
\end{center}
\caption{Spezialeigenschaften}
\label{tab:spezialeigenschaften}
\end{table}


\subsection{Konzentrationswerte}
Jede Grundeigenschaft, Farbe, Geruch und Geschmack besitzt einen Konzentrationswert, der je nach Ingredienz unterschiedlich sein kann. Der Wert gibt an, wie dominant diese Eigenschaft ist. Je höher die Dominanz einer bestimmten Eigenschaft ist, desto eher wird sie auch zu einer Eigenschaft des alchemistischen Endergebnisses. 

Beispiel: Wenn für einen Trank hauptsächlich süße Beeren verwendet werden, kann man davon ausgehen, dass der Trank am Ende ebenfalls süß schmeckt. 

Mit Zufall hat das allerdings nichts zu tun. Stattdessen kann exakt berechnet werden, bei welchen Zutaten, welche Grundeigenschaft dominiert und sich entsprechend auf das Ergebnis überträgt. Mit diesem Wissen, kann der Brauvorgang gezielt manipuliert werden um bestimmte Eigenschaften auf dem Endergebnis zu erzielen.\footnote{siehe Kapitel \ref{chap:berechnung_neuer_eigenschaften}} Bei allen Ingredienzen sind die Konzentrationswerte festgelegt. Aber Zutaten, von denen die Eigenschaften nicht bekannt sind, entscheidet der SL über die Eigenschaften und Konzentrationswerte.

\subsubsection{Beispiel} 
Richtwerte für einen knallroten Apfel könnten folgendermaßen aussehen:\\
\textbf{Farbe:} rot (3) \\
\textbf{Geruch:} fruchtig (1) \\
\textbf{Geschmack:} apfel (2)  \\
\textbf{Spezialeigenschaften:} keine

Die Zahlen in Klammern geben jeweils den Konzentrationswert der entsprechenden Eigenschaft an. Die Farbe hat 3, weil es sich um einen "`knallroten"' Apfel handelt. Der Geschmack eines Apfels ist normal, deshalb 2. Da ein Apfel keinen besonderen Geruch hat bekommt er bei Geruch nur eine 1. Und natürlich hat ein Apfel auch keine Spezialeigenschaften wie \textit{Gift}. 

\subsection{Wirkstoffe}
Bestimmte alchemistische Gebräue benötigen zur Herstellung keine spezifische(n) Zutate(n) sondern einen oder mehrere Wirkstoffe. Alle Wirkungsstoffe sind geruchlos, transparent und geschmacklos.

\subsubsection{Wirkstoffe extrahieren}
\label{chap:wirkstoffgewinnung}
Jede Ingredienz besteht aus mindestens einem Hauptwirkstoff und manchmal einem Sekundärwirkstoff. Um aus einer Ingredienz einen (oder mehrere) Wirkstoff(e) zu extrahieren, muss der Held eine \textbf{Probe auf \textit{Alchemie} bestehen} um an den Hauptwirkstoff zu gelangen. Schafft er die Probe mit \textbf{QS 3 oder höher}, erhält er zusätzlich den Sekundärwirkstoff. Sollte die Ingredienz keinen Sekundärwirkstoff beinhalten, bekommt er stattdessen einen zweiten Hauptwirkstoff. 


\section{Suchen und Sammeln}
Bei der Suche von Ingredienzen muss der SL zwischen zwei Varianten unterscheiden. Einer gezielten Suche und einer unbestimmten Suche. Bei der gezielten Suche, sucht der Held eine bestimmte Ingredienz. Bei der unbestimmten Suche sucht der Held irgendwelche Ingredienzen.

\subsection{Gezielte Suche}
Bei der gezielten Suche bestimmt der Spieler vor der Probe, welche Ingredienz er sucht. Zusätzlich gibt er an, wie lange er danach suchen möchte. Je länger er sucht, desto höher ist die Wahrscheinlichkeit, dass er etwas findet. 

\subsection{Häufig vorkommende Ingredienz}
Die Suche von häufig vorkommenden Ingredienzen ist einfach: Einmal pro Suche muss der Spieler eine Probe auf \textit{Sinnesschärfe} bestehen. Danach hat der Spieler einmal pro Ort eine 5/6 Chance, die gesuchte Ingredienz zu finden. Findet der Spieler nichts, kann er alle 30min am selben Ort suchen, bis er etwas gefunden hat. D.h., dass an einem Ort nur einmal etwas gefunden werden kann. Bei jedem folgenden Versuch wird die Change (aber nicht die Menge) um 1/6 erhöht (maximale Erhöhung=2/6). Um die gesuchte Ingredienz zu finden wirft er einen W6. Bei 1 bis 5 hat es der Spieler gefunden. Das Wurfergebnis gibt gleichzeitig die Menge an, die er gefunden hat. Mit der Menge muss nicht unbedingt die Ingredienz selbst gemeint sein. Nachdem die Menge bestimmt wurde kann es sein, dass die Ingredienz eine \textit{Alchemie}-Probe verlangt, die mit einer festgelegten QS bestanden werden muss. Misslingt die Probe wird die Menge für jeden Punkt den der Spieler daneben lag um 1 reduziert.

Wenn der Spieler beispielsweise nach einer Beere sucht macht er eine Probe auf \textit{Sinnesschärfe}. Er besteht die Probe und würfelt einen W6. Er würfelt eine 6. D.h., dass er in den ersten 30min nichts gefunden hat. Nach 30min besteht er eine erneute \textit{Sinnesschärfe}-Probe und würfelt erneut einen W6. Da die Chance diesmal auf 6/6 (5/6+1/6) gestiegen ist, findet er auf jeden Fall etwas. Er würfelt eine 6. Da zwar die Chance gestiegen ist, aber nicht die maximale Menge die er finden kann, zählt die 5. Laut SL bedeutet das, dass der Spieler 5 Sträucher gefunden hat, an dem die Beeren wachsen. Der SL entscheidet, dass er für jeden Strauch einen weiteren W6 würfeln darf um die Menge der Beeren zu bestimmen. Er würfelt eine 1, eine 3 und eine 4. Somit gibt es 8 Beeren die er pflücken kann. Da die Beeren sehr empfindlich sind verlangen sie eine Probe mit QS 2 (steht im Herbarium des Grundregelwerks zu dieser Beere). Der Spieler schafft die Probe leider nur mit 2 Punkten, also QS 1. Da er zwei Punkte unter QS 2 ist, bekommt er 8-2=6 Beeren. Jetzt sind 60min vorbei.

Die Bestimmung der Mengen kann der SL bei Bedarf ändern. Die obigen Zahlen sind lediglich Richtwerte. Nur die Chance, mit der der Spieler etwas findet ist festgelegt.

\subsection{Mittelhäufig/selten vorkommende Ingredienz}
Die Suche nach mittelhäufig und selten vorkommenden Ingredienzen verläuft genauso wie bei den häufig Vorkommenden. Bei mittelhäufigen Ingredienzen sinkt die Chance auf 1/2 (1 bis 3 bei W6) und bei seltenen Ingredienzen auf 1/6 (1 bei W6).

\subsection{Unbestimmte Suche}
Wenn der Spieler keine bestimmte Ingredienz sucht muss der Spieler eine Probe auf \textit{Sinnesschärfe} ablegen. Gelingt ihm die Probe mit QS I, findet er eine zufällige häufig vorkommende Ingredienz. Wie der Zufall aussieht entscheidet der SL. Z.B. kann er den Spieler einen W20 würfeln lassen, wobei jede Zahl für eine andere Ingredienz steht. Bei QS II, findet er eine mittelhäufig oder häufig vorkommende Ingredienz und bei QS III+ eine Mittelhäufige. Seltene Ingredienzen können bei der unbestimmten Suche niemals gefunden werden.

\section{Herstellung}
Damit man alchemistische Gebräue wie Tränke, Öle und Bomben herstellen kann benötigt der Held den Vorteil \textit{Alchemie}. Alchemistische Gebräue besitzen eine Schwierigkeitsstufe - leicht, mittel oder schwer. Den \textit{Alchemie}-Vorteil gibt es in drei Ausbaustufen. Mit der ersten Stufe können nur leichte Gebräue hergestellt werden. Mit der zweiten Stufe zusätzlich mittlere Gebräue und mit der dritten Stufe, \textit{Alchemie III}, alle. 

\subsection{Flüssigkeit als Grundlage}
\label{chap:fluessigkeit_als_grundlage}
Zur Herstellung von leichten Tränke und Öle benötigt man irgendein Alkohol. Für mittlere Tränke und Öle wird ein qualitativ hochwertiger Alkohol benötigt und für schwere Tränke und Öle braucht man \textit{Weiße Möwe}. \textit{Weiße Möwe} ist ein einfacher Trank, der genauso wie alle anderen Tränke hergestellt werden kann. Ob die Herstellung von alchemistischen Gebräue gelingt entscheidet eine Probe auf \textit{Alchemie}. Diese Probe kann durch den \textit{Alchemie}-Vorteil erleichtert werden.\footnote{siehe Kapitel \ref{chap:vorteile}}

Für andere Gebräue können auch beliebig andere Flüssigkeiten eingesetzt werden. Z.B. Wasser. Es ist jedoch zu beachten, dass die Flüssigkeit nicht verschmutzt ist. Somit wäre Regenwasser keine Option. 

\begin{table}
\begin{center}
\begin{tabular}{p{5,5cm}|p{5,5cm}|p{5,5cm}}
\textbf{Billiger Fusel} & \textbf{Guter Schnaps} & \textbf{Exzellenter Schnaps} \\ \hline
& & \\
\textbf{Farbe:} neutral (0) \newline
\textbf{Geruch:} Alkohol (8) \newline
\textbf{Geschmack:} Alkohol (10)
& 
\textbf{Farbe:} neutral (0) \newline
\textbf{Geruch:} Alkohol (7) \newline
\textbf{Geschmack:} Alkohol (8)
& 
\textbf{Farbe:} neutral (0) \newline
\textbf{Geruch:} Alkohol (5) \newline
\textbf{Geschmack:} Alkohol (6)
\end{tabular}
\end{center}
\caption{Eigenschaften von Alkohol}
\label{tab:eigenschaften_von_alkohol}
\end{table}

\subsection{Berechnung neuer Eigenschaften}
\label{chap:berechnung_neuer_eigenschaften}
Wenn man alchemistische Gebräue aus verschiedenen Zutaten herstellt, bekommt das Gebräu ebenfalls bestimmte Eigenschaften. Welche Eigenschaften das sind hängt von den verwendeten Zutaten ab. 

Für jede Grundeigenschaft, werden die Konzentrationswerte der selben Eigenschaft zusammen addiert. Die Eigenschaft, die den höchsten Konzentrationswert hat dominiert und wird auf das Ergebnis übertragen. Haben mehrere Grundeigenschaften den selben Konzentrationswert ergibt sich daraus eine neue Grundeigenschaft. Dabei ist zu beachten, dass auch der Alkohol der zur Herstellung von Gebräue benutzt wird, bestimmte Grundeigenschaften hat. 

Bei Spezialeigenschaften verhält es sich ein wenig anders. Spezialeigenschaften wirken kumulativ und nicht dominant. D.h., dass wenn verschiedene Spezialeigenschaften aufeinander treffen, sich nicht eine Eigenschaft durchsetzt, sonder alle. Allerdings kann es sein, dass sich verschiedene Spezialeigenschaften gegenseitig neutralisieren. Z.B. \textit{Gift} und \textit{Giftneutralisierend}. Kommt eine Spezialeigenschaft mehrfach vor, kann der SL die Wirkung des Gebräus verbessern. Z.B. werden zur Herstellung eines Gifttranks besonders viele giftige Zutaten verwendet, könnte der Trank besonders schnell wirken.

Der endgültige Effekt wird nicht nur von den Spezialeigenschaften festgelegt, sondern auch von der Qualität des Tranks. Wenn einem Trank ein Rezept zu Grunde liegt, wird die Qualität wesentlich verbessert, als wenn man ohne Rezept braut.

\subsubsection{Beispiel 1}
Wenn ich einen Schlaftrank im Sinne von K.O.-Tropfen brauen will, würde ich das Rezept des \textit{Schlaftranks} als Grundlage nehmen. Der \textit{Schlaftrank} ist ein leichter Trank, der nur wenige Zutaten braucht und keine besonderen Eigenschaften hat. Er hat den Effekt von K.O.-Tropfen. Allerdings schmeckt und riecht er stark nach Alkohol, somit kann man ihn nicht unbemerkt in ein Getränk geben. Wenn ich das Rezept mit einem normalen Schnaps braue hat der \textit{Schlaftrank} folgende Eigenschaften:\\
\textbf{Farbe:} transparent \\
\textbf{Geruch:} alkohol (8) \\
\textbf{Geschmack:} alkohol (10) \\
\textbf{Spezialeigenschaft:} Rausch und Schläfrig \\

Möchte ich den Trank rot färben, füge ich einfach eine Zutat hinzu, die die Farbeigenschaft \textit{rot} hat. Dadurch, dass der Trank transparent ist, würde die rote dominante Farbe sofort auf den Trank übergehen. 

Möchte ich zusätzlich noch den Alkoholgeschmack rausbekommen, kann ich den fertigen Trank mit \textit{Frauentränen} kombinieren. \textit{Frauentränen} ist ein leichter Trank, der den Alkohol neutralisiert. Also den Alkohol-Geruch, -Geschmack und Wirkung (Rausch). 

\subsubsection{Beispiel 2}
Wenn ich ein einfaches Schlafmittel herstellen will, das ich auf dem Marktplatz verkaufen kann, würde ich kein Rezept als Grundlage verwenden. Stattdessen würde ich aus ein paar Zutaten mit der Spazialeigenschaft \textit{Schläfrig} und einem guten milden Schnaps ein Schlafmittel brennen. Der milde Schnaps, damit das Mittel nicht so stark nach Alkohol schmeckt. 

\subsection{Tränke und Öle}
Zur Herstellung von Tränken und Ölen muss eine Probe auf \textit{Alchemie} bestanden werden. Außerdem benötigt man mindestens alle benötigten Zutaten. Es können auch mehrere Zutaten verwendet werden, um die Eigenschaften des Tranks zu verändern. Diese Probe kann durch den \textit{Alchemie}-Vorteil vereinfacht werden.

\subsection{Bomben}
Bomben werden grundsätzlich genauso wie Tränke hergestellt. Allerdings gibt es noch zwei weitere Dinge zu beachten. Wie wird die Bombe gezündet und worin wird die Bombe aufbewahrt.

\subsubsection{Zünder}
Es gibt zwei Möglichkeiten eine Bombe zu zünden. Entweder durch eine einfache Zündschnur oder durch eine natürliche Reaktion.

Für eine Zündschnur benötigt man Schwarzpulver und eine brennbare Schnur. Beides wird nach dem Herstellen der Bombe hinzugefügt. Dafür ist unter normalen Bedingungen keine Probe notwendig. Wenn jedoch erschwerte Bedingungen herrschen, beispielsweise durch ein Feuer in der Nähe oder Dunkelheit, kann der SL eine Probe auf \textit{Alchemie} fordern. Misslingt die Probe, explodiert die Bombe und der Spieler erhält 4+1W6 Schaden. Der SL kann abhängig von der Bombe auch weitere Folgen festlegen. Statt Schwarzpulver kann auch jedes andere Mineral mit der Spezialeigenschaft \textit{explosiv} verwendet werden.

Die zweite natürliche Möglichkeit wäre ein Mineral mit der Spezialeigenschaft \textit{Instabil} zu verwenden. Für das Anbringen dieses Zünders gelten die selben Regeln wie für die Zündschnur. Diese Variante kann z.B. für Bomben verwendet werden, die beim Aufprall explodieren sollen oder für Land- oder Wasserminen.

\subsubsection{Gefäße}
In was die Bombe aufbewahrt wird ist für den Effekt und die Wirkung irrelevant. Allerdings können in festen Gefäßen auch weitere Teile wie Nageln reingetan werden, um z.B. eine Splitterbombe zu bekommen. Außerdem kann es für das Auslösen der Explosion wichtig sein. Eine Landmine beispielsweise sollte nicht aus einem massiven Metallkasten bestehen, sondern eher einem einfachen Holzgefäß, das zerbricht wenn jemand darauf tritt.

\subsection{Sonstiges}
Wie bereits erwähnt kann der Spieler auch Brauen, ohne dafür ein Rezept als Grundlage zu verwenden. Für solch eine Zubereitung gelten die selben Regeln wie beim Brauen nach Rezept. Auch muss hier ein Alkohol als Grundlage dienen. Einzige Ausnahme ist die Herstellung von nicht-alkoholischen Getränken wie z.B. Säfte. In solchen Ausnahmen darf auch Wasser oder eine andere Flüssigkeit verwendet werden. Die Grund- und Spezialeigenschaften werden aus den verwendeten Zutaten genauso berechnet wie bisher. Als Zutaten können nur spezifische Dinge und keine Wirkstoffe verwendet werden.

Der Effekt von Spezialeigenschaften bei solchen Gebräuen sind stark abgeschwächt. Ein Gebräu mit der Spezialeingenschaft \textit{Gift} hätte dementsprechend nicht die selbe Wirkung wie ein \textit{Gifttrant}. Stattdessen würde eine Person die den Trank trinkt höchstens Krank werden. Wenn die Wirkung solcher abgeschwächten Spezialeigenschaften nicht eindeutig ist, entscheidet der SL den Effekt.

\subsubsection{Beispiele}
Mögliche Gebräue die ohne Rezept hergestellt werden können: 
\begin{itemize}
	\item Saft, z.B. Apfelsaft
	\item Schlafmittel
	\item Abtreibungsmittel
	\item Schnaps
	\item Parfüm
\end{itemize}


\section{Qualitätsmerkmale}
Die Qualität von Gebräuen (egal ob mit oder ohne Rezept) hängt in erster Linie von den verwendeten Zutaten ab. Dazu gehört auch der verwendete Alkohol. Die Qualität eines Gebräus wird anhand den Konzentrationswerten gemessen. Die Qualität eines Gebräus beeinflusst lediglich den Wert, jedoch nicht den Effekt. Dabei gelten folgende Skalen als Richtwerte. 

\subsection{Geruch-Skala}
\textbf{1-2:} kaum bemerkbar \\
\textbf{3-4:} angenehm \\
\textbf{5:} stark \\
\textbf{6:} intensiv \\
\textbf{7+:} extrem 

\subsection{Farb-Skala}
\textbf{1-2:} leichte Färbung \\
\textbf{3-4:} normale Färbung \\
\textbf{5-6:} deutlich \\
\textbf{7+:} knallig

\subsection{Geschmack-Skala}
\textbf{1-2:} kaum \\
\textbf{3:} spürbar \\
\textbf{4-5:} gut \\
\textbf{6-7:} perfekt \\
\textbf{8+:} extrem

\subsection{Qualitätsbeispiele}
Hier sind ein paar Beispiele die zeigen was gute Werte für bestimmte Gebräue sind. Die folgenden Beispiele richten sich nach den obigen Skalen.

\subsubsection{Schnaps}
\textbf{Geruch:} z.B. Lavendel (\textit{angenehm} oder \textit{stark}) \\
\textbf{Farbe:} egal (\textit{0-2}, \textit{transparent} bis \textit{leichte Färbung}) \\
\textbf{Geschmack:} alkohol-obst (\textit{perfekt})

Für Schnaps gilt außerdem, dass der Geschmack eine Mischung aus alkohol und mindestens einer weiteren Zutat sein sollte. Z.B. \textit{Geschmack: alkohol-apfel (6)}.

\subsubsection{Saft}
\textbf{Geruch:} z.B. Apfel (\textit{angenehm}) \\
\textbf{Farbe:} apfel-gelb (\textit{deutlich} oder \textit{knallig}) \\
\textbf{Geschmack:} fruchtig (\textit{intensiv})

Ein Parfüm mit der Grundeigenschaft \textit{Geruch: lavendel (6)} wäre von exzellenter Qualität. Die Farbe und der Geschmack von Parfüm ist von zweitrangiger Bedeutung.


\section{Herbarium}
Hier sind alle Ingredienzen aufgeführt, die für die Herstellung von Gebräue wie Tränke, Öle oder Bomben benutzt werden können.

\subsection{Heilkraut}
Bla bla bla.
Grundeigenschaften: ...
Spezialeigenschaften: Gesund

\section{Tränke}
\subsection{leicht}
Weiße Möwe - Grundalkohol für schwere Tränke \\
Frauentränen - Neutralisiert Alkohol \\
Heiltrank - Stellt LP wieder her \\

\subsection{mittel}
Gifttrank - tödliches Gift \\
Betäubungstrank - K.O.-Tropfen \\

\subsection{schwer}
Fisstech - Harte Droge \\


\section{Öle}
\subsection{leicht}
Geisteröl \\
\subsection{mittel}
Nekrophagenöl \\
\subsection{schwer}
Gragonidenöl \\


\section{Bomben}
\subsection{mittel}
normale Bombe - normale Bombe die explodiert \\
Stinkbombe - So starker Gestank, der alles und jeden vertreibt \\

\subsection{schwer}




\section{Tränke}
Folgende Tränke stehen den Spielern zur Verfügung. 

\begin{tabular}{|p{3.5cm}|p{7cm}|p{1.5cm}|p{3cm}|}
\hline
\textbf{Name} & \textbf{Wirkung} & \textbf{WD} & \textbf{Schwierigkeit} \\

\hline
\textbf{Katze} & Nachtsicht. Man sieht alles in grau, weiß. & 8h & Mittel \\

\hline
\textbf{Weisser Honig} & Vergiftungen (und Effekte eingenommener Tränke) werden neutralisiert & - & Leicht \\

\hline
\textbf{Frauentränen} & Neutralisiert Effekte von Alkohol. Trinkspiel! & - & Leicht \\

\hline
\textbf{Genesungstrank} & Stoppt Blutungen und heilt leichte Wunden. & - & Mittel \\

\hline
\textbf{Heiltrank} & Regeneriert +1LP pro Minute. (Insgesamt +10LP) & 10min & Mittel \\

\hline
\textbf{Parfüm} & Z.B. Als Geschenk. & - & Leicht \\

\hline
\textbf{Fisstech} & Harte Droge. (Vergleichbar mit Heroin).  & 8h & Schwer \\

\hline
\textbf{Weisse Möwe} & Hexer-Haluzinogen. Gute Basis für Hexer-Tränke. & 1h & Hexer \\

\hline
\textbf{Graues Blut} & Starkes Betäubungsgift. Wirkt nach wenigen Sekunden (0AK). Kann nicht nur getrunken werden, sonder auch auf Pfeilspitzen oder Klingen geschmiert werden. & - & Schwer \\

\hline
\textbf{Schwarzes Blut} & Starkes tödliches Gift. Wirkt nach wenigen Minuten (12-24AK). Kann nicht nur getrunken werden, sonder auch auf Pfeilspitzen oder Klingen geschmiert werden. & - & Schwer \\

\hline
\textbf{Milde Schwalbe} & Milde Version von Schwalbe. Heilt +2LP pro AK. & 5AK & Schwer \\

\hline
\textbf{Schwalbe} & Heilt +4LP pro AK. & 3AK & Hexer \\

\hline
\end{tabular}


\section{Öle}

\begin{tabular}{|p{4cm}|p{8.5cm}|p{3cm}|}
\hline
\textbf{Name} & \textbf{Wirkung} & \textbf{Schwierigkeit} \\

\hline
\textbf{Geisteröl} & Geister erleiden +5 mehr Schaden. & Leicht \\

\hline
\textbf{Orgoidenöl} & Orgoiden erleiden +5 mehr Schaden. & Mittel \\

\hline
\textbf{Draconidenöl} & Daconiden erleiden +5 mehr Schaden. & Mittel \\

\hline
\textbf{Insektoidenöl} & Insektoiden erleiden +5 mehr Schaden. & Leicht \\

\hline
\textbf{Konstruktöl} & Konstrukte erleiden +5 mehr Schaden. & Leicht \\

\hline
\textbf{Nekrophagenöl} & Nekrophagen erleiden +5 mehr Schaden. & Leicht \\

\hline
\textbf{Hybridenöl} & Hybride erleiden +5 mehr Schaden. & Leicht \\

\hline
\textbf{Bestienöl} & Bestien erleiden +5 mehr Schaden. & Mittel \\

\hline
\textbf{Argentina} & Gegner erleiden +2 mehr Schaden. & Schwer \\

\hline
\textbf{Braunöl} & Gegner bluten. -1LP pro Aktion. Gut gegen Endgegner. & Schwer \\

\hline
\textbf{Henkersgiftserum} & Gegner werden vergiftet. -1LP Giftschaden pro Aktion. & Schwer \\

\hline
\end{tabular}

\section{Bomben}