{\let\clearpage\relax\chapter{Religionen}}
Alle Religionen die es in der Welt gibt.

\section{Kult des ewigen Feuers / Orden der Flammenrose}
\subsection{Allgemeines}
\begin{itemize}
\item nach dem zweiten Krieg mit Nilfgaard
\item Hauptsitz: Vizima (Tempelbezirk)
\item Scharlachrote Banner mit dem Symbol einer Rose (und gelblichen Flammen im Hintergrund)
\item verbreitet in \textit{Redanien}, stammt ursprünglich aus \textit{Novigrad}
\end{itemize}

\subsection{Beschreibung}
Fanatismus und nahezu vollständige Unterwerfung sind das Markenzeichen der Geistlichen des Ewigen Feuers. 
Die Religion steht fast allen anderen feindlich gegenüber, Anderlinge eingeschlossen. 
Der Orden der Flammenrose wurde zum militanten Arm des Kultes.\\
~\\
Das Ziel des Ordens der Flammenrose ist die Menschen vor Ungeheuern sowie allem Übel und Bösen zu beschützen. 
Angehörige des Flammenordens glauben an das Ewige Feuer. Es heißt, jeder Bürger, egal welchen Standes, kann dem 
Orden beitreten. 

\subsection{Angesehene Mitglieder}
\begin{itemize}
\item Jacques de Aldersberg
\item Siegfried von Denesle
\item Patrick de Weyze 
\end{itemize}

\section{Kult der Löwenkopfspinne}
\subsection{Allgemeines}
\begin{itemize}
\item Coram Agh Tera, genannt die Löwenkopfspinne
\item "`Göttin des plötzlichen Todes"' $ \rightarrow $ mächtige Flüche
\item Tempel nicht in Städten
\item weltweit verbreitet
\item (sogut wie) überall verboten $ \rightarrow $ offizielle Kleidung nur bei Ritualen
\end{itemize}

\section{Elfen (Aén Seidhe)}
Die Elfen glauben, dass sie von den Göttern geschaffen wurden und sich die Menschen lediglich entwickelt haben. Darum sehen sie in den Menschen oft nicht mehr als "`nackte Affen"'. 

\section{Kult der Melitele}
\subsection{Allgemeines}
\begin{itemize}
\item in den nördlichen Königreichen
\item Hauptsitz: Ellander (Tempel der Melitele)
\item Nebensitze/-orte: Vizima
\end{itemize}

\subsection{Beschreibung}
Göttin mit den drei Gestalten: junges Mädchen, Frau und alte Vettel. Melitele ist die Muttergöttin, deren Fürsorge ihren Kindern gilt, und ihre Gefolgschaft besteht nicht ausschließlich aus Frauen – auch Männer beten zu ihr. Kleriker der Melitele predigen Liebe und Frieden. Sie führen viele Krankenhäuser. 
