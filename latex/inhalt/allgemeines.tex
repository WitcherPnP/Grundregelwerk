\chapter{Abkürzungsverzeichnis}
\begin{longtable}{|p{3cm}|p{12cm}|}
\hline
\textbf{Abkürzung} & \textbf{Bedeutung} \\ \hline
QS & Qualitätsstufe \\ \hline
AW & Ausweichen \\ \hline
TW & Trefferwahrscheinlichkeit \\ \hline
TP & Trefferpunkte (= Schaden) \\ \hline
BL & Blocken \\ \hline
PA & Parieren \\ \hline
GS & Geschwindigkeit \\ \hline
AK & Aktion (im Kampf) \\ \hline
SF & Steifungsfaktor (gibt an, wie schwer eine Fertigkeit aufzuleveln ist) \\ \hline
RW & Reichweite \\ \hline
MTP & Magische Trefferpunkte (= magischer Schaden) \\ \hline
MTW & Magische Trefferwahrscheinlichkeit \\ \hline
MK & Manakosten \\ \hline
M & Mana oder ausgegebenes Mana (für eine Fähigkeit) \\ \hline
ZZ & Zauberzeit (= wie lange dauert es, einen Zauberspruch zu wirken) \\ \hline
LZ & Ladezeit (= wie lange braucht man um eine Waffe nachzuladen) \\ \hline
WD & Wirkungsdauer \\ \hline
QSWX & Anzahl Würfel in Abhängigkeit der erreichten Qualitätsstufe \\
& \textit{Beispiel: QSW6 bei QS II $\rightarrow$ 2W6} \\ \hline
BE & Behinderung, kann körperliche Aktionen wie z.B. \textit{Kraftakt} oder \textit{Klettern} erschweren. AW, INI, MTW und TW sind immer um 1 pro BE erschwert. \\ \hline
EW & Effektwahrscheinlichkeit (= Wahrscheinlichkeit, dass der Effekt einer Glyphe eintritt) \\ \hline
RD & Rüstungsdurchbohrung \\ \hline
RS & Rüstschutz \\ \hline

\caption{Abkürzungsverzeichnis}
\label{tab:Abkuerungsverzeichnis}
\end{longtable}
