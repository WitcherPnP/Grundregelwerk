{\let\clearpage\relax\chapter{Klassen}}
\textbf{Hinweise:}
\begin{itemize}
\item Glyphen kosten 10
\item Kampfsonderfertigkeiten kosten 5-20 Punkte (SL entscheidet)
\item Jede Art von Puntke (z.B. Talent- oder Fähigkeitspunkte) können bei Absprache mit dem SL auch für andere Dinge benutzt werden
\item \textbf{Startgeld}: 1000Kr.
\item Ein Grundattribut darf nicht über 14 sein.
\item Alle Grundattribute zusammen dürfen nicht höher als 100 sein.
\item jeder bekommt einen besonderen Gegenstand, der bis zu ~1000Kr. kostet. Z.B. Pferd, Mobiles Labor, Glyphe(n), Amulett, etc... (Wenn die Spieler damit einverstanden sind?!)
\item Helden starten nur mit zur Klasse passenden Kleidung.
\end{itemize}


\section{Militärische Klassen}
\textbf{Hauptkategorie}: Waffen \& Kampf \\
\textbf{Sekundärkategorie}: Körpertalente \\
\textbf{KK oder GE}: +2 \\
\textbf{MU}: +2 \\
\textbf{Haupttalentpunkte}: 40 \\
\textbf{Sekundärtalentpunkte}: 20 \\
\textbf{Talentpunkte}: 273 \\
\textbf{Kampfsonderfertigkeiten}: Fortgeschrittene Kampfkunst \\
\textbf{Vorteile}: - \\
\textbf{Nachteile}: - \\
\textbf{Waffenspezialiesierung}: Waffentalent auswählen $\rightarrow$ entsprechenden SF auf A.


\section{Motorische Klassen}
\textbf{Hauptkategorie}: Motorische Talente \\
\textbf{Sekundärkategorie}: Waffen \& Kampf (Räuber/Bandit) oder Körpertalente (Dieb) oder Gesellschaftstalente (Trickbetrüger) \\
\textbf{Haupttalentpunkte}: 40 \\
\textbf{Sekundärtalentpunkte}: 20 \\
\textbf{Talentpunkte}: 319 \\
\textbf{Kampfsonderfertigkeiten}: - \\
\textbf{Vorteile}: - \\
\textbf{Nachteile}: - \\
\textbf{Waffentalent}: Ein Waffentalent auf 12.


\section{Soziale Klassen}
\textbf{Hauptkategorie}: Gesellschaftstalente \\
\textbf{Sekundärkategorie}: ? \\
\textbf{CH}: +2 \\
\textbf{Haupttalentpunkte}: 40 \\
\textbf{Sekundärtalentpunkte}: 20 \\
\textbf{Talentpunkte}: 289 \\
\textbf{Fähigkeitspunkte}: - \\
\textbf{Einfache Fähigkeiten}: - \\
\textbf{Mächtige Fähigkeiten}: - \\
\textbf{Heilzauber}: - \\
\textbf{Glyphen}: - \\
\textbf{Kampfsonderfertigkeiten}: - \\
\textbf{Vorteile}: - \\
\textbf{Nachteile}: - \\
\textbf{Waffentalent}: Ein Waffentalent auf 12.

\section{Handwerkliche Klassen}
\textbf{Hauptkategorie}: Handwerkstalente \\
\textbf{Sekundärkategorie}: Körpertalente (Schmied) oder Motorische Talente (Schlosser) \\
\textbf{KK oder FF}: +2 \\
\textbf{Haupttalentpunkte}: 40 \\
\textbf{Sekundärtalentpunkte}: 20 \\
\textbf{Talentpunkte}: 289 \\
\textbf{Kampfsonderfertigkeiten}: - \\
\textbf{Vorteile}: - \\
\textbf{Nachteile}: - \\
\textbf{Waffentalent}: Ein Waffentalent auf 12.

\section{Wissende Klassen}
\subsection{Magiebegabte Klassen}
Der SL oder Spieler hat 10 Punkte zur Verfügung, die er beliebig auf einfache-, mächtige, Heilzauber oder Glyphen aufteilen kann. Mächtige Fähigkeiten und Heilzauber kosten 2 Punkte.

\textit{Beispiel: Wenn der Spieler einen normalen Magier spielen will, kann der SL sagen, dass er sich 4 einfache und 3 mächtige Zauber aussuchen darf. Wenn der Spieler aber Runenzauber beherrschen will, kann ihm der SL zusätzlich die Fähigkeit Runenzauber geben, zieht dafür aber 10 Talentpunkte ab (oder der Spieler muss sich einen Nachteil geben). Die 10 Punkte kann er dann auch auf Glyphen aufteilen. Z.B. 4 Glyphen, 2 einfache und 2 mächtige Zauber.}

\textbf{Hauptkategorie}: Wissenstalente \\
\textbf{Sekundärkategorie}: Gesellschafts- (Magier) oder Naturtalente (Druide, Weideler, Priester) \\
\textbf{KL}: +2
\textbf{Haupttalentpunkte}: 40 \\
\textbf{Sekundärtalentpunkte}: 20 \\
\textbf{Talentpunkte}: 235 \\
\textbf{Fähigkeitspunkte}: 50 \\
\textbf{Einfache Fähigkeiten}: ? \\
\textbf{Mächtige Fähigkeiten}: ? \\
\textbf{Heilzauber}: ? \\
\textbf{Glyphen}: ? \\
\textbf{Kampfsonderfertigkeiten}: - \\
\textbf{Vorteile}: Altersresistenz \\
\textbf{Nachteile}: - \\
\textbf{Waffentalent}: Stabkampf auf 10.

\subsection{Alchemistische Klassen}
Hiermit sind Klassen gemeint, die sich in der Alchemie auskennen. Das können beispielsweise Alchemisten, Chemiker oder Wissenschaftler sein. Sie sind in der Lage Tränke, Öle, Bomben und mehr herzustellen. Der \textit{Alchemie}-Vorteil ist nicht nur für diese Kategorie reserviert. Für solche Klassen gilt die folgende Klassenbeschreibung.

\textbf{Hauptkategorie}: Wissenstalente \\
\textbf{Sekundärkategorie}: Naturtalente \\
\textbf{KL}: +2 \\
\textbf{FF}: +2 \\
\textbf{Haupttalentpunkte}: 40 \\
\textbf{Sekundärtalentpunkte}: 20 \\
\textbf{Talentpunkte}: 265 \\
\textbf{Kampfsonderfertigkeiten}: - \\
\textbf{Vorteile}: \textit{Alchemie II} \\
\textbf{Nachteile}: - \\
\textbf{Waffentalent}: Ein Waffentalent auf 10.\\
\textbf{Talente SF Mod.}: \\
\textit{Pflanzenkunde} $\rightarrow$ A \\
\textit{Magiekunde} $\rightarrow$ C \\ 
\textit{Menschenkenntnis} $\rightarrow$ B \\
\textit{Stabkampf} $\rightarrow$ C \\
\textit{Tanzen} $\rightarrow$ B \\
~\\

\subsubsection{Wissende ohne Magie- und Alchemie-Kenntnis}
In diese Kategorie fallen Leute, die weder über Magie- noch Alchemie-Kenntnisse verfügen. Dazu gehören beispielsweise Lehrer und Akademiker, für die folgende Werte gelten.

\textbf{Hauptkategorie}: Wissenstalente \\
\textbf{Sekundärkategorie}: Gesellschaftstalente \\
\textbf{KL}: +2 \\
\textbf{GE}: +2 \\
\textbf{Haupttalentpunkte}: 40 \\
\textbf{Sekundärtalentpunkte}: 20 \\
\textbf{Talentpunkte}: 265 \\
\textbf{Kampfsonderfertigkeiten}: - \\
\textbf{Vorteile}: - \\
\textbf{Nachteile}: - \\
\textbf{Waffentalent}: Ein Waffentalent auf 10.\\
\textbf{Talente SF Mod.}: \\
\textit{Alchemie} $\rightarrow$ C \\
\textit{Heilkunde} $\rightarrow$ A \\
\textit{Pflanzenkunde} $\rightarrow$ A \\
\textit{Magiekunde} $\rightarrow$ C \\
\textit{Etikette} $\rightarrow$ B \\
\textit{Gassenwissen} $\rightarrow$ B \\
\textit{Menschenkenntnis} $\rightarrow$ B \\
\textit{Überreden} $\rightarrow$ B \\


\section{Naturbezogene Klassen}
\textbf{Hauptkategorie}: Naturtalente \\
\textbf{Sekundärkategorie}: Körpertalente (Holzfäller,Bauer) oder Motorische Talente (Jäger) \\
\textbf{KO}: +2 \\
\textbf{Haupttalentpunkte}: 40 \\
\textbf{Sekundärtalentpunkte}: 20 \\
\textbf{Talentpunkte}: 289 \\
\textbf{Fähigkeitspunkte}: - \\
\textbf{Einfache Fähigkeiten}: - \\
\textbf{Mächtige Fähigkeiten}: - \\
\textbf{Heilzauber}: - \\
\textbf{Glyphen}: - \\
\textbf{Vorteile}: - \\
\textbf{Nachteile}: - \\
\textbf{Waffentalent}: Ein Waffentalent auf 12.

\section{Beispielklassen}
\begin{longtable}{|l|l|}
\hline
\textbf{Klasse} & \textbf{Kategorie} \\ \hline

Barde/in & Soziale Klasse \\ \hline
Handwerker/in & Handwerkliche Klasse \\ \hline
Dieb/in & Motorische Klassen \\ \hline
Schmuggler/in & Motorische Klassen \\ \hline
Arzt/Ärztin & Wissende/Soziale Klasse \\ \hline
Magier/in & Wissende Klassen \\ \hline
Söldner/in & Militärische Klassen \\ \hline
Kaufmann/frau & Soziale Klasse \\ \hline
Priester/in & Wissende Klasse \\ \hline
Jäger & Naturbezogene Klasse \\ \hline

\caption{Beispielklassen}
\label{tab:Beispielklassen}
\end{longtable}
