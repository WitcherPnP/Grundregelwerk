{\let\clearpage\relax\chapter{Vor- und Nachteile}
\label{chap:vor_und_nachteile}}
Alle Vor- und Nachteile, die die Helden besitzen können. Nicht jeder Vor- und Nachteil kann mit jeder Rasse, Klasse, Herkunft oder Anderes verwendet werden.

\section{Vorteile}
\label{chap:vorteile}
\begin{longtable}{|p{5cm}|p{11cm}|}

\hline
\textbf{Reich~I-X (5)} & +500Kr. pro Stufe \\ \hline

\textbf{Adel (5)\footnote{wird durch Klasse oder Rasse festgelegt}} & Der Held ist angesehen, genießt die Privilegien des Adels und kann vom SL Erleichterungen zugesprochen bekommen, wenn er gegenüber Rangniedrigeren agiert. Der Spieler bekommt beim Talent \textit{Etikette} ein A. \\ \hline

\textbf{Alchemie I-III (20)} & Ermöglicht das Herstellen von alchemistischen Gebräuen (Tränke, Öle, Bomben). \newline \textbf{I:} Herstellung von leichten alchemistischen Gebräuen \newline maximal 1 zusätzliche Zutat \newline SF von \textit{Alchemie} und \textit{Pflanzenkunde} wird zu B (wird nicht verschlechtert) \newline \textbf{II:} Herstellung von mittleren alchemistischen Gebräuen \newline maximal 3 zusätzliche Zutaten \newline +2 auf leichte Gebräue \newline \textbf{III:} Herstellung von schweren alchemistischen Gebräuen \newline beliebig viele zusätzliche Zutaten \newline +4 auf leichte Gebräue \newline +2 auf mittlere Gebräue \\ \hline

\textbf{Altersresistenz (-)\footnotemark[1]} & Der Held ist immun gegen natürliche Alterserscheinungen. Negative Auswirkungen des Alters werden bei ihm nicht berücksichtigt. \\ \hline

\textbf{Angenehmer Geruch (6)} & Fertigkeitsproben auf Betören sind um 1 erleichtert. \\ \hline

\textbf{Begabung (10)} & Ein begabter Abenteurer kann bei jeder Probe auf eine bestimmte Fertigkeit einen W20 neu würfeln. Der Spieler kann zunächst die 3W20 würfeln und sich eines der drei Ergebnisse aussuchen, dass er neu würfeln kann. Es gilt das bessere Ergebnis beider Würfe. Eine Begabung muss für jede Fertigkeit einzeln gewählt werden, also zum Beispiel auf ein bestimmtes Talent. Man kann nicht mehrfach in einer Fertigkeit begabt sein und somit zwei oder gar drei W20-Würfe in einer Probe wiederholen. \\ \hline

\textbf{Beidhändig (10)} & Der Held erleidet keine Erschwernisse bei Fertigkeitsproben, wenn er die falsche Hand benutzt. Im Kampf hebt der Vorteil alle Abzüge auf, die durch das Führen einer Waffe mit der falschen Hand anfallen. \\ \hline

\textbf{Dunkelsicht (10)} & Erschwernisse durch Dunkelheit werden um eine Stufe gesenkt. Bei vollständiger Finsternis ist aber Dunkelsicht wirkungslos und es gelten die Abzüge für unsichtbare Ziele. \\ \hline

\textbf{Entfernungssinn (8)} & Proben auf Fernkampf mit Schusswaffen (und nur mit Schusswaffen) sind bei der Entfernungskategorie Weit nur um 1 anstatt um 2 erschwert. \\ \hline

\textbf{Feiner Spürsinn (8)} & Der Abenteurer erhält bei Proben auf \textit{Sinnesschärfe} zur Entdeckung von Geheimtüren, Hohlräumen, versteckten Schubladen und Ähnlichem eine Erleichterung von 1. \\ \hline

\textbf{Flink (8)} & Durch Flink bekommt der Held 1 Punkt auf seinen Geschwindigkeit-Grundwert hinzu. Nicht mit den Nachteilen \textit{Behäbig}, \textit{Fettleibig} oder \textit{Verstümmelt} (Einbeinig) kombinierbar. \\ \hline

\textbf{Fuchssinn (15)} & Der Held kann Fallen in einem Umkreis von 10 Metern erspüren, die mit den herkömmlichen Sinnen nicht entdeckt werden können. \\ \hline

\textbf{Geborener Redner (4)} & Proben auf das Talent \textit{Bekehren \& Überzeugen} (öffentliche Rede) sind um 1 erleichtert. \\ \hline

\textbf{Glück~I-III (20)} & Das für den Helden erreichbare Maximum an Schicksalspunkten steigt hierdurch um 1 je Stufe. \\ \hline

\textbf{Gutaussehend~I-II (12)} & Jede Stufe des Vorteils gibt dem Helden eine Erleichterung von 1 auf \textit{Betören} (Anbändeln, Liebeskünste), \textit{Überreden} (Aufschwatzen, Herausreden, Manipulieren, Schmeicheln) und \textit{Handel} (Feilschen). Außerdem verbessert sich der SF von \textit{Betören} um 1 pro Stufe. Diesen Vorteil können nur Menschen und Elfen haben. \\ \hline

\textbf{Harmonie des Waldes (10)} & Befindet die Heldin sich in einem Wald, so kann sie einmal am Tag eine Probe auf Willenskraft ablegen, um QS/2 Zustandsstufen von Betäubung, Furcht, Paralyse oder Schmerz abzubauen. Wenn mehrere Stufen auf diese Weise aufgehoben werden können, können diese zwischen den Zuständen aufgeteilt werden (z. B. 1 Stufe Furcht und 2 Stufen Betäubung). Die Probe abzulegen dauert 1 Minute. \\ \hline

\textbf{Hass auf... (?)} & Der Held richtet gegen eine Art von Wesen, das beim Erwerb dieses Vorteils festgelegt werden muss, +1 TP im Nahkampf an. Der Typus des Wesens muss dabei näher spezifiziert werden. Je häufiger der Typus des Wesens in der Welt von \textit{The Witcher} vorkommt, desto teurer der AP-Wert für diesen Vorteil. Ausgeschlossen als wählbare Typen sind die Oberkategorien Lebewesen und Nicht Lebende. Neben den verschiedenen Typen und Unterkategorien kann man auch Spezies und Kulturen wählen. Die eigene Spezies/Kultur ist prinzipiell wählbar, jedoch sind Helden mit Hass auf die eigene Spezies/Kultur oft nur schwer spielbar. Der SL hat das letzte Wort bezüglich der Wählbarkeit eines Hassobjektes. Je kleinteiliger ein Typus gewählt wird und je seltener eine potenzielle Begegnung mit ihm ist, desto niedriger die AP-Kosten für den Vorteil. \\ \hline

\textbf{Herausragender Sinn (10)} & Wer über einen herausragenden Sinn verfügt, dessen Proben auf \textit{Sinnesschärfe} sind um 1 erleichtert, wenn die Probe den entsprechenden Sinn betrifft. Folgende Sinne können gewählt werden: Sicht, Gehör, Geruch \& Geschmack, Tastsinn. Herausragender Sinn ist ist mehrfach für die verschiedenen Sinne wählbar. \\ \hline

\textbf{Hohe Lebenskraft I-VII (6)} & Der LP-Grundwert steigt durch diesen Vorteil um 1 Punkt pro Stufe. Nicht mit dem Nachteil \textit{Niedrige Lebenskraft} kombinierbar. \\ \hline

\textbf{Leichter Gang (10)} & Die Heldin kann sich schneller bewegen als normale Menschen. Ihre GS erhält einen Bonus von 1. Zudem erhält sie +1FP bei gelungenen Proben auf Tanzen (bis zu einem Maximum von 18 FP). Vorraussetzung: \textit{Tanzen}-Talent benötigt SF von A oder B und FW von \textit{Tanzen} muss größer als 0 sein. \\ \hline

\textbf{Magieresistent I-III (20)} & Erhöht die Seelenkraft um 1 pro Stufe. Nicht mit dem Nachteil \textit{Magieempfindlich} kombinierbar. \\ \hline

\textbf{Magische Belastbarkeit I-V (4)} & Die maximale Energie (Mana) erhöht sich um 1 pro Stufe. Darf nur von Magiebegabten gewählt werden. \\ \hline

\textbf{Magisches Talent I-III (10)} & Die Probe zum Schöpfen von Energie (Mana) ist um 1 pro Stufe erleichtert. Darf nur von Magiebegabten gewählt werden. \\ \hline 

\textbf{Nochmal auf Anfang (15)} & Wenn eine Talentprobe misslingt, gilt es nicht automatisch als Fehlschlag, sonder kann wiederholt werden. Auch Schicksalspunkt-Würfe können wiederholt werden. \\ \hline

\textbf{Promi (20)} & Der Held ist in seinem Heimatland berühmt. Es wird wie ein König behandelt - wird häufig zum Essen eingeladen und bekommt kostenlos Zimmer zur Verfügung gestellt. \\ \hline

\textbf{Reich I-X (10)} & Der Held beginnt mit 250Kr pro Stufe zusätzlich. Nicht mit dem Nachteil \textit{Arm} kombinierbar. \\ \hline

\textbf{Respektsperson (16)} & Folgende Talente werden um 1 erleichtert: Bekehren \& Überzeugen, Etikette, Überreden und Handel. \\ \hline

\textbf{Runenspezialist I-II (10)} & Verbessert den SF vom \textit{Runenzauber} um 1 pro Stufe. Kann nur von Spielern gewählt werden, die den \textit{Runenzauber} erlernt haben. \\ \hline

\textbf{Saumagen I-II (2)} & Mit Saumagen kann der Held zähes Fleisch problemlos essen, so als ob es normales Fleisch wäre. Bei Stufe II kann er sogar ungenießbares Fleisch verzehren. Es wird ihm nicht sonderlich schmecken, aber er kann es verdauen, ohne dass ihm übel wird und er sich erbrechen muss. Gifte und Krankheitskeime können jedoch nicht durch den Vorteil umgangen werden. Ähnliches gilt auch für pflanzliche Nahrungsmittel. Schwerverdauliche Kost ist essbar, ohne dass es weitere Auswirkungen hat. \\ \hline

\textbf{Schlangenmensch (5)} & Proben auf Körperbeherrschung (Akrobatik, Entwinden) sind um 1 erleichtert. \\ \hline

\textbf{Soziale Anpassungsfähigkeit (10)} & Personen mit Sozialer Anpassungsfähigkeit können Erschwernisse auf Gesellschaftstalente ignorieren, die durch Standesunterschiede entstehen. Die Soziale Anpassungsfähigkeit täuscht aber keine Personen des Hochadels. \\ \hline

\textbf{Starker Glaube (8)} & Bei der temporären Energieerhöhung durch das Beten, gilt 1+1W8 anstelle von 1+W4. Wenn der Standardwert der maximalen Energie erreicht ist, bekommt der Held nach einer Stunde \textit{Verwirrung I} bis er wieder betet. Ist der Held bereits verwirrt, wird die Verwirrung nicht erhöht. Die Verwirrung kann aber erst verschwinden, wenn die maximale Energie über dem Standardwert liegt. Darf nur von magiebegabten Priestern und Druiden gewählt werden. \\ \hline

\textbf{Tierfreund (10)} & Proben auf \textit{Tierkunde} (domestizierte Tiere, Wildtiere) gelten bei Gelingen der Probe als um 1 QS besser, allerdings kann sich die QS dabei nie über 6 erhöhen. Der Held darf nicht den Vorteil \textit{Hass auf} die Tierart haben. \\ \hline

\textbf{Unscheinbar (4)} & Helden mit diesem Vorteil erhalten eine Erleichterung von 1 auf \textit{Gassenwissen} (Beschatten). \\ \hline

\textbf{Verbesserte Regeneration I-III (10)} & Für jede Stufe des Vorteils bekommt der Held 1 zusätzlichen LP dazu, wenn er regeneriert. Nicht mit Nachteil \textit{Schlechte Regeneration} kombinierbar. \\ \hline

\textbf{Vertrauenserweckend (10)} & Der SL kann Erleichterungen auf manche (Gesellschafts-)Talente geben. \\ \hline

\textbf{Vieltrinker (8)} & Der SF von \textit{Zechen} verbessert sich um eine Stufe. Außerdem bekommt der Spieler eine Erleichterung von 1 auf \textit{Zechen}. \\ \hline

\textbf{Waffenbegabung (10)} & Eine Waffenbegabung erlaubt bei kritischen Erfolgen und Patzern bei einer Probe auf eine Kampftechnik eine Wiederholung des Bestätigungswurfs. \\ \hline

\textbf{Zeitgefühl (2)} & Der Charakter hat ein ausgezeichnetes Zeitgefühl und kann, ohne den Stand der Sonne zu sehen oder auf andere Hilfsmittel zurückzugreifen, die Tageszeit genau bestimmen. \\ \hline

\textbf{Zäher Hund (20)} & Der Held erleidet lediglich die nächstniedrigere Stufe von \textit{Schmerz}. Bei Schmerz Stufe IV wird der Held dennoch handlungsunfähig (K.O.). Bei Schmerz Stufe I wird so behandelt, als hätte der Held keine Stufe Schmerz. Zusätzlich erhöht sich die Zähigkeit um 1. \\ \hline

\caption{Vorteile}
\label{tab:Vorteile}
\end{longtable}

\section{Nachteile}
\label{chap:nachteile}
\begin{longtable}{|p{5cm}|p{11cm}|}
\hline

\textbf{Alkoholiker (30)} & Der SF von \textit{Zechen} verbessert sich um zwei Stufen. Außerdem bekommt der Spieler eine Erleichterung von 2 auf \textit{Zechen}. Wenn der Spieler nicht mindestens eine Stufe Betäubung wegen Alkohol hat, bekommt er eine Erschwernis von 2 auf alles körperliche. Wenn der Held lange keinen Alkohol trinkt, darf der SL die Erschwernis auf 1 verringern oder sogar komplett streichen. (Solange bis der Held wieder was trinkt.) \\ \hline

\textbf{Angst vor ... I-III (10)} & Pro Stufe der Angst erleidet der Held eine Stufe Furcht, so lange er dem Auslöser ausgesetzt ist. \\ \hline

\textbf{Arm I-IV (10)} & Der Held beginnt mit 250Kr pro Stufe weniger. Nicht mit dem Vorteil \textit{Reich} kombinierbar. \\ \hline

\textbf{Behäbig (8)} & Durch Behäbig verliert ein Held 2 Punkte von seinem Geschwindigkeit-Grundwert (GS, Standard: 10). \\ \hline

\textbf{Eingeschränkter Sinn (15)} & Wer über einen Eingeschränkten Sinn verfügt, dessen Proben auf Sinnesschärfe sind um 2 erschwert, wenn die Probe den entsprechenden Sinn betrifft. Eingeschränkte Sicht erschwert zudem Proben auf Fernkampf um 2. Folgende Sinne können gewählt werden: Sicht, Gehör, Tastsinn. Ein Held kann bis zu zwei eingeschränkte Sinne besitzen. \\ \hline

\textbf{Farbenblind (8)} & Der Abenteurer kann Farben nicht mehr erkennen. Das kann sich bei einigen Proben, z.B. \textit{Alchemie}, \textit{Etikette} (Zuordnung von Wappen) oder \textit{Orientierung} negativ auswirken. Ob eine Probe abgelegt werden kann oder um wie hoch die Erschwernis ist, entscheidet der SL. \\ \hline

\textbf{Fettleibig (25)} & Proben auf \textit{Klettern}, \textit{Körperbeherrschung}, \textit{Tanzen} und \textit{Verbergen} sind um 1 erschwert. Der Nachteil muss mit \textit{Behäbig} kombiniert werden. \\ \hline

\textbf{Geiz (8)} & Der Held kann keine Gegenstände kaufen, deren Preis höher als normal ist. \\ \hline

\textbf{Giftempfindlich (10)} & Gift richtet 1 Schaden pro Aktion mehr an als normal. \\ \hline

\textbf{Gläsern (3)} & Die Wundschwelle des Helden sinkt durch den Nachteil um 1. D.h., dass der Held die erste Stufe \textit{Schmerz} erleidet, wenn ihm $\frac{1}{4}$-1 LP fehlen. \\ \hline

\textbf{Hasslich I-II (12)} & Durch jede Stufe dieses Nachteils erleidet der Held eine Erschwernis von 1 auf \textit{Betören}, \textit{Überreden} und \textit{Handel}. Zusätzlich wird der SF von \textit{Betören} um 1 pro Stufe verschlechtert. \\ \hline

\textbf{Jagdwildgeruch (10)} & Der SL kann immer, wenn der Held nahe genug an hungrigen Raubtieren ist (50 Meter Radius) und diese ihn wittern, mit dem W6 würfeln. Bei einer 1-2 halten sie den Abenteurer für Beute und greifen ihn an. Sollte es zu einem Kampf mit einer ganzen Heldengruppe kommen, so greifen die Raubtiere bevorzugt einen Helden mit dem Nachteil \textit{Jagdwildgeruch} an. Als Raubtiere gelten beispielsweise Jaguare und Würgeschlangen, auf keinen Fall aber reine Pflanzenfresser. Welche Tiere von dem Geruch genau angelockt werden, entscheidet der SL. \\ \hline

\textbf{Körperliche Auffälligkeit (2)} & Ein Held kann maximal zwei Körperliche Auffälligkeiten wählen (z.B. eine Narbe). Die Körperliche Auffälligkeit zieht gelegentlich Erschwernisse auf Gesellschaftstalente aufgrund von Aberglaube oder Misstrauen nach sich. \\ \hline

\textbf{Körperliche Schwäche (20)} & Erschwernis auf alles körperliche um 1. (KK darf nicht größer als 10 sein) \\ \hline

\textbf{Lichtempfindlich (20)} & Der Held erleidet eine Stufe Schmerz, sobald er Sonnenlicht ausgesetzt ist, das heller ist als Dämmerlicht. Diese Auswirkung lässt sich vermeiden, wenn der Betroffene seinen kompletten Körper inklusive der Augen verhüllt. \\ \hline

\textbf{Lächerlicher Name (3)} & Die Heldin hatte schon immer einen lächerlichen Namen, oder aber sie hat sich irgendwann einen  ehrenrührigen Beinamen verdient. Wenn die Heldin  ihren Namen nennt, muss ihr Gegenüber eine Probe  auf Willenskraft bestehen, um nicht unangemessen zu  reagieren. Hat sich jemand an den Namen gewöhnt,  entfällt diese Probe. Voraussetzung ist ein möglichst lächerlicher Name. \\ \hline

\textbf{Magieempflindlich I-III (10)} & Verringert die Seelenkraft um 1 pro Stufe. Nicht mit dem Vorteil \textit{Magieresistent} kombinierbar. \\ \hline

\textbf{Niedrige Lebenskraft I-VII (4)} & Der LP-Grundwert sinkt durch diesen Nachteil um 1 Punkt pro Stufe. Nicht mit dem Vorteil \textit{Hohe Lebenskraft} kombinierbar. \\ \hline

\textbf{Pech I-III (20)} & Der Held startet pro Stufe des Nachteils mit 1 Schicksalspunkt weniger als üblich. \\ \hline

\textbf{Pechmagnet (5)} & Bei jeder zufälligen Bestimmung des SLs (z.B. welcher Held wird von dem Pfeil getroffen?) sind die Chancen bei einem Helden mit diesem Nachteil mindestens doppelt so hoch wie normal. \\ \hline

\textbf{Persönlichkeitsschwächen (5-10)} & In passenden Situationen kann der SL durchaus für die angegebenen Fertigkeiten eine Erschwernis von 1 aussprechen, wenn er dies für angemessen hält. Es dürfen maximal zwei Persönlichkeitsschwächen pro Held gewählt werden. Beispiele sind: \textbf{Arroganz (10)}, \textbf{Eitelkeit (10)}, \textbf{Neid (5)}, \textbf{Streitsucht (10)}, \textbf{Unheimlich (8)}, \textbf{Verwöhnt/Faul (10)}, \textbf{Vorurteile gegenüber ... (5)}. \\ \hline

\textbf{Prinzipientreue I-III (10)} & Wer gegen seine Prinzipien verstößt, erleidet eine Erschwernis von 1 pro Stufe auf alle Fertigkeitsproben. Dieser Zustand dauert mindestens eine Stunde an, über die genaue Länge entscheidet der Meister unter Berücksichtigung der Schwere des Verstoßes. Ein Held kann mehreren Prinzipien folgen. \\ \hline

\textbf{Raubtiergeruch (8)} & Der Meister kann immer, wenn der Held nahe genug an domestizierten Tieren ist und diese ihn wittern, mit dem W6 würfeln. Bei einer 1 versuchen sie, vor der vermeintlichen Bedrohung zu fliehen. Hunde und Katzen meiden die Nähe eines Helden mit diesem Nachteil, Pferde scheuen und selbst Schafe und Ziegen fliehen blökend, sobald er sich ihnen nähert. Bei Proben auf Tierkunde und Reiten im Umgang mit domestizierten Tieren kann ein Held mit Raubtiergeruch maximal 1 QS übrig behalten. Nicht mit dem Nachteil  \textit{Unfähig} (Reiten, Tierkunde) kombinierbar. \\ \hline

\textbf{Schlafwandler (10)} & Kein Held sollte mehr als einmal pro Woche schlafwandeln. Am Tag nach dem Schlafwandeln erhält der Held 24 Stunden lang eine Stufe des Zustands Betäubung durch die geringe Erholung in der Nacht. Die Regeneration ist für den Held in der Nacht, in der er schlafwandelt um 1 gesenkt. \\ \hline

\textbf{Schlechte Angewohnheit (2)} & Ein Held kann so viele Schlechte Angewohnheiten wählen, wie er möchte, erhält hierdurch allerdings insgesamt maximal 6 AP. In seltenen Fällen kann der SL auch Erschwernisse auf Gesellschaftstalente vergeben. \\ \hline

\textbf{Schlechte Eigenschaft (?)} & In jeder Situation, in der der Held mit einem potenziellen Auslöser seiner Schlechten Eigenschaft konfrontiert wird, muss er eine Probe auf \textit{Willenskraft} bestehen, um sich zu beherrschen. Gelingt ihm diese Probe, ist alles in Ordnung, ansonsten muss er entsprechend der schlechten Eigenschaft agieren. Seine schlechte Eigenschaft hat ihn so lange im Griff, wie er dem Auslöser ausgesetzt ist. Der SL kann für die Probe auf \textit{Willenskraft} entsprechend der Stärke des Auslösers Erschwernisse oder Erleichterungen aussprechen. Es dürfen bis zu zwei schlechte Eigenschaften pro Held gewählt werden. Kombinationen, die sich ausschließen (z. B. Geiz und Verschwendungssucht), dürfen nicht gewählt werden. Das letzte Wort darüber hat der Spielleiter. \\ \hline

\textbf{Schlechte Regeneration (10)} & Für jede Stufe des Nachteils muss der Held beim Regenerationswurf 1 Punkt abziehen (bis zu einem Minimum von 0). Nicht mit dem Vorteil \textit{Verbesserte Regeneration} kombinierbar. \\ \hline

\textbf{Sensibler Geruchssinn (10)} & Solange der Held dem Geruch ausgesetzt ist, erleidet er eine Stufe Verwirrung. \\ \hline

\textbf{Sprachfehler (15)} & Mögliche Erschwernis um 1, wenn es darum geht mit anderen zu sprechen, z.B. \textit{Handel}. \\ \hline

\textbf{Tollpatsch (10)} & Erschwert alle Handwerkstalente um 1. Der SL kann den Nachteil auch für weitere Missgeschicke verwenden. \\ \hline

\textbf{Unfähig (10)} & Der Held muss bei einer Probe auf das Talent, das er ausgewählt hat, das beste Würfelergebnis einer Teilprobe neu würfeln. Das zweite Ergebnis ist bindend. Ein Abenteurer kann maximal zwei Unfähigkeiten besitzen. \\ \hline

\textbf{Verpflichtungen I-III (10)} & Der Held muss Anweisungen desjenigen befolgen, dem er verpflichtet ist, oder mit den Konsequenzen leben. Ein Held kann mehreren Institutionen oder Personen verpflichtet sein, er bekommt jedoch nur einmal Abenteuerpunkte für den Nachteil (und zwar für die höchste Verpflichtungsstufe). \\ \hline

\textbf{Verstümmelt (5-30)} & Dem Held fehlt ein Körperteil. \textbf{Einarmig (30)}, \textbf{Einäugig (10)}, \textbf{Einbeinig (30)}, \textbf{Einhändig (20)}, \textbf{Einohrig (5)} \\ \hline

\textbf{Wilde Magie (10)} & 19er werden bei Fertigkeitsproben auf Zauber in Hinsicht auf die Bestimmung eines Patzers wie eine 20 behandelt. Nur von Magiebegabten wählbar. \\ \hline

\textbf{Zerbrechlich (20)} & Erschwernisse durch den Zustand Schmerz werden wie bei einer Stufe höher behandelt. Die Handlungsunfähigkeit (K.O.) tritt somit bereits bei Stufe III ein. \\ \hline

\caption{Nachteile}
\label{tab:Nachteile}
\end{longtable}