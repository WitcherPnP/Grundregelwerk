{\let\clearpage\relax\chapter{Vor- und Nachteile}
\label{chap:vor_und_nachteile}}
Alle Vor- und Nachteile, die die Helden besitzen können. Nicht jeder Vor- und Nachteil kann mit jeder Rasse, Klasse, Herkunft oder Anderes verwendet werden.

\section{Vorteile}
\label{chap:vorteile}
\begin{longtable}{|p{5cm}|p{11cm}|}

\hline
\textbf{Reich~I-X (5)} & +500Kr. pro Stufe \\ \hline
\textbf{Gutaussehend~I-II (12)} & Jede Stufe des Vorteils gibt dem Helden eine Erleichterung von 1 auf \textit{Betören} (Anbändeln, Liebeskünste), \textit{Überreden} (Aufschwatzen, Herausreden, Manipulieren, Schmeicheln) und \textit{Handel} (Feilschen). Außerdem verbessert sich der SF von \textit{Betören} um 1 pro Stufe. Diesen Vorteil können nur Menschen und Elfen haben. \\ \hline
\textbf{Vertrauenserweckend (10)} & Der SL kann Erleichterungen auf manche (Gesellschafts-)Talente geben. \\ \hline
\textbf{Schlangenmensch (5)} & Proben auf Körperbeherrschung (Akrobatik, Entwinden) sind um 1 erleichtert. \\ \hline
\textbf{Begabung (10)} & Ein begabter Abenteurer kann bei jeder Probe auf eine bestimmte Fertigkeit einen W20 neu würfeln. Der Spieler kann zunächst die 3W20 würfeln und sich eines der drei Ergebnisse aussuchen, dass er neu würfeln kann. Es gilt das bessere Ergebnis beider Würfe. Eine Begabung muss für jede Fertigkeit einzeln gewählt werden, also zum Beispiel auf ein bestimmtes Talent. Man kann nicht mehrfach in einer Fertigkeit begabt sein und somit zwei oder gar drei W20-Würfe in einer Probe wiederholen. \\ \hline
\textbf{Beidhändig (10)} & Der Held erleidet keine Erschwernisse bei Fertigkeitsproben, wenn er die falsche Hand benutzt. Im Kampf hebt der Vorteil alle Abzüge auf, die durch das Führen einer Waffe mit der falschen Hand anfallen. \\ \hline
\textbf{Dunkelsicht (10)} & Erschwernisse durch Dunkelheit werden um eine Stufe gesenkt. Bei vollständiger Finsternis ist aber Dunkelsicht wirkungslos und es gelten die Abzüge für unsichtbare Ziele. \\ \hline
\textbf{Entfernungssinn (8)} & Proben auf Fernkampf mit Schusswaffen (und nur mit Schusswaffen) sind bei der Entfernungskategorie Weit nur um 1 anstatt um 2 erschwert. \\ \hline
\textbf{Flink (8)} & Durch Flink bekommt ein Held 2 Punkte auf seinen Geschwindigkeit-Grundwert (GS, Standard: 10) hinzu. \\ \hline
\textbf{Nochmal auf Anfang (15)} & Wenn eine Talentprobe misslingt, gilt es nicht automatisch als Fehlschlag, sonder kann wiederholt werden. \\ \hline
\textbf{Respektsperson (16)} & Folgende Talente werden um 1 erleichtert: Bekehren \& Überzeugen, Etikette, Überreden und Handel. \\ \hline
\textbf{Promi (20)} & Der Held ist in seinem Heimatland berühmt. Es wird wie ein König behandelt - wird häufig zum Essen eingeladen und bekommt kostenlos Zimmer zur Verfügung gestellt. \\ \hline
\textbf{Angenehmer Geruch (6)} & Fertigkeitsproben auf Betören sind um 1 erleichtert. \\ \hline
\textbf{Geborener Redner (4)} & Proben auf das Talent \textit{Bekehren \& Überzeugen} (öffentliche Rede) sind um 1 erleichtert. \\ \hline
\textbf{Glück~I-III (20)} & Das für den Helden erreichbare Maximum an Schicksalspunkten steigt hierdurch um 1 je Stufe. \\ \hline
\textbf{Herausragender Sinn (10)} & Wer über einen herausragenden Sinn verfügt, dessen Proben auf \textit{Sinnesschärfe} sind um 1 erleichtert, wenn die Probe den entsprechenden Sinn betrifft. Folgende Sinne können gewählt werden: Sicht, Gehör, Geruch \& Geschmack, Tastsinn. Herausragender Sinn ist ist mehrfach für die verschiedenen Sinne wählbar. \\ \hline
\textbf{Ausgeprägte magische Energie I-III (10)} & Erhöht das maximale Mana um 10 pro Stufe \\ \hline
\textbf{Unscheinbar (4)} & Helden mit diesem Vorteil erhalten eine Erleichterung von 1 auf \textit{Gassenwissen} (Beschatten). \\ \hline
\textbf{Verbesserte Magieschöpfung (12)} & Magier können auch an nicht-magischen Orten Mana schöpfen, indem sie sich 2AK darauf konzentrieren $\rightarrow$ +1W6 Mana. \\ \hline
\textbf{Soziale Anpassungsfähigkeit (10)} & Personen mit Sozialer Anpassungsfähigkeit können Erschwernisse auf Gesellschaftstalente ignorieren, die durch Standesunterschiede entstehen. Die Soziale Anpassungsfähigkeit täuscht aber keine Personen des Hochadels. \\ \hline
\textbf{Waffenbegabung (10)} & Eine Waffenbegabung erlaubt bei kritischen Erfolgen und Patzern bei einer Probe auf eine Kampftechnik eine Wiederholung des Bestätigungswurfs. \\ \hline
\textbf{Feiner Spürsinn (8)} & Der Abenteurer erhält bei Proben auf \textit{Sinnesschärfe} zur Entdeckung von Geheimtüren, Hohlräumen, versteckten Schubladen und Ähnlichem eine Erleichterung von 1. \\ \hline
\textbf{Altersresistenz (-)\footnotemark[1]} & Der Held ist immun gegen natürliche Alterserscheinungen. Negative Auswirkungen des Alters werden bei ihm nicht berücksichtigt. \\ \hline
\textbf{Adel (5)\footnote{wird durch Klasse oder Rasse festgelegt}} & Der Held ist angesehen, genießt die Privilegien des Adels und kann vom SL Erleichterungen zugesprochen bekommen, wenn er gegenüber Rangniedrigeren agiert. Der Spieler bekommt beim Talent \textit{Etikette} ein A. \\ \hline
\textbf{Zäher Hund (20)} & Der Held erleidet lediglich die nächstniedrigere Stufe von \textit{Schmerz}. Bei Schmerz Stufe IV wird der Held dennoch handlungsunfähig (K.O.). Bei Schmerz Stufe I wird so behandelt, als hätte der Held keine Stufe Schmerz. \\ \hline
\textbf{Vieltrinker (8)} & Der SF von \textit{Zechen} verbessert sich um eine Stufe. Außerdem bekommt der Spieler eine Erleichterung von 1 auf \textit{Zechen}. \\ \hline

\caption{Vorteile}
\label{tab:Vorteile}
\end{longtable}

\section{Nachteile}
\label{chap:nachteile}
\begin{longtable}{|p{5cm}|p{11cm}|}
\hline
\textbf{Angst vor ... I-III (10)} & Pro Stufe der Angst erleidet der Held eine Stufe Furcht, so lange er dem Auslöser ausgesetzt ist. \\ \hline
\textbf{Arm (10)} & Der Held beginnt nur mit 250Kr. und einfacher Kleidung. \\ \hline
\textbf{Behäbig (8)} & Durch Behäbig verliert ein Held 2 Punkte von seinem Geschwindigkeit-Grundwert (GS, Standard: 10). \\ \hline
\textbf{Sensibler Geruchssinn (10)} & Solange der Held dem Geruch ausgesetzt ist, erleidet er eine Stufe Verwirrung. \\ \hline
\textbf{Eingeschränkter Sinn (15)} & Wer über einen Eingeschränkten Sinn verfügt, dessen Proben auf Sinnesschärfe sind um 2 erschwert, wenn die Probe den entsprechenden Sinn betrifft. Eingeschränkte Sicht erschwert zudem Proben auf Fernkampf um 2. Folgende Sinne können gewählt werden: Sicht, Gehör, Tastsinn. Ein Held kann bis zu zwei eingeschränkte Sinne besitzen. \\ \hline
\textbf{Unfähig (10)} & Der Held muss bei einer Probe auf das Talent, das er ausgewählt hat, das beste Würfelergebnis einer Teilprobe neu würfeln. Das zweite Ergebnis ist bindend. Ein Abenteurer kann maximal zwei Unfähigkeiten besitzen. \\ \hline
\textbf{Gier (8)} & Der Held kann keine Gegenstände kaufen, deren Preis höher als normal ist. \\ \hline
\textbf{Prinzipientreue I-III (10)} & Wer gegen seine Prinzipien verstößt, erleidet eine Erschwernis von 1 pro Stufe auf alle Fertigkeitsproben. Dieser Zustand dauert mindestens eine Stunde an, über die genaue Länge entscheidet der Meister unter Berücksichtigung der Schwere des Verstoßes. Ein Held kann mehreren Prinzipien folgen. \\ \hline
\textbf{Schlafwandler (10)} & Kein Held sollte mehr als einmal pro Woche schlafwandeln. Am Tag nach dem Schlafwandeln erhält der Held 24 Stunden lang eine Stufe des Zustands Betäubung durch die geringe Erholung in der Nacht. Die Regeneration ist für den Held in der Nacht, in der er schlafwandelt um 1 gesenkt. \\ \hline
\textbf{Tollpatsch (10)} & Erschwert alle Handwerkstalente um 1. Der SL kann den Nachteil auch für weitere Missgeschicke verwenden. \\ \hline
\textbf{Verstümmelt (5-30)} & Dem Held fehlt ein Körperteil. \textbf{Einarmig (30)}, \textbf{Einäugig (10)}, \textbf{Einbeinig (30)}, \textbf{Einhändig (20)}, \textbf{Einohrig (5)} \\ \hline
\textbf{Farbenblind (8)} & Der Abenteurer kann Farben nicht mehr erkennen. Das kann sich bei einigen Proben, z.B. \textit{Alchemie}, \textit{Etikette} (Zuordnung von Wappen) oder \textit{Orientierung} negativ auswirken. Ob eine Probe abgelegt werden kann oder um wie hoch die Erschwernis ist, entscheidet der SL. \\ \hline
\textbf{Fettleibig (25)} & Proben auf \textit{Klettern}, \textit{Körperbeherrschung}, \textit{Tanzen} und \textit{Verbergen} sind um 1 erschwert. Der Nachteil muss mit \textit{Behäbig} kombiniert werden. \\ \hline
\textbf{Hasslich I-II (12)} & Durch jede Stufe dieses Nachteils erleidet der Held eine Erschwernis von 1 auf \textit{Betören}, \textit{Überreden} und \textit{Handel}. Zusätzlich wird der SF von \textit{Betören} um 1 pro Stufe verschlechtert. \\ \hline
\textbf{Körpergebundene Kraft (5)} & Sobald der Held, aus welchem Grund auch immer, einen Teil seiner Haarpracht einbüßt (z.B. durch Feuer oder Abschneiden, nicht normaler Haarausfall), verliert er sofort 10 Mana. Diese können normal regeneriert werden. \\ \hline
\textbf{Lichtempfindlich (20)} & Der Held erleidet eine Stufe Schmerz, sobald er Sonnenlicht ausgesetzt ist, das heller ist als Dämmerlicht. Diese Auswirkung lässt sich vermeiden, wenn der Betroffene seinen kompletten Körper inklusive der Augen verhüllt. \\ \hline
\textbf{Pech (20)} & Der Held startet pro Stufe des Nachteils mit 1 Schicksalspunkt weniger als üblich. \\ \hline
\textbf{Pechmagnet (5)} & Bei jeder zufälligen Bestimmung des SLs (z.B. welcher Held wird von dem Pfeil getroffen?) sind die Chancen bei einem Helden mit diesem Nachteil mindestens doppelt so hoch wie normal. \\ \hline
\textbf{Persönlichkeitsschwächen (5-10)} & In passenden Situationen kann der SL durchaus für die angegebenen Fertigkeiten eine Erschwernis von 1 aussprechen, wenn er dies für angemessen hält. Es dürfen maximal zwei Persönlichkeitsschwächen pro Held gewählt werden. Beispiele sind: \textbf{Arroganz (10)}, \textbf{Eitelkeit (10)}, \textbf{Neid (5)}, \textbf{Streitsucht (10)}, \textbf{Unheimlich (8)}, \textbf{Verwöhnt/Faul (10)}, \textbf{Vorurteile gegenüber ... (5)}. \\ \hline
\textbf{Sprachfehler (15)} & Mögliche Erschwernis um 1, wenn es darum geht mit anderen zu sprechen, z.B. \textit{Handel}. \\ \hline
\textbf{Zerbrechlich (20)} & Erschwernisse durch den Zustand Schmerz werden wie bei einer Stufe höher behandelt. Die Handlungsunfähigkeit (K.O.) tritt somit bereits bei Stufe III ein. \\ \hline
\textbf{Giftempfindlich (10)} & Gift richtet 1 Schaden pro Aktion mehr an als normal. \\ \hline
\textbf{Körperliche Schwäche (20)} & Erschwernis auf alles körperliche um 1. (KK darf nicht größer als 10 sein) \\ \hline
\textbf{Alkoholiker (30)} & Der SF von \textit{Zechen} verbessert sich um zwei Stufen. Außerdem bekommt der Spieler eine Erleichterung von 2 auf \textit{Zechen}. Wenn der Spieler nicht mindestens eine Stufe Betäubung wegen Alkohol hat, bekommt er eine Erschwernis von 2 auf alles körperliche. Wenn der Held lange keinen Alkohol trinkt, darf der SL die Erschwernis auf 1 verringern oder sogar komplett streichen. (Solange bis der Held wieder was trinkt.) \\ \hline

\caption{Nachteile}
\label{tab:Nachteile}
\end{longtable}