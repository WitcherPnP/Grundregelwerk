{\let\clearpage\relax\chapter{Kampf}}
In diesem Kapitel werden die möglichen Aktionen der Helden während dem Kampf aufgelistet und beschrieben. Nicht jede Aktion kann von jedem Helden verwendet werden. Das hängt von folgenden Faktoren ab: Rasse, Klasse, Waffe, körperliche Behinderungen. Nicht jede Aktion kann von jedem Helden verwendet werden. Jenachdem, wie wenig Kampferfahrung ein Charakter hat, fallen ein paar Aktionen weg. Für Spezialmanöver (u.A. Parieren) wird die Kampfsonderfertigkeit \textit{Fortgeschrittene Kampfkunst} benötigt.

\section{Behinderung}
Behinderung, oder auch Belastung, kann durch alles Mögliche entstehen. Z.B. durch schwere Last, die man am Körper trägt (schwere Rüstung). Ein Punkt Belastung wirkt sich wie folgt auf Aktionen und Werte aus: -1AW, -1INI, -1GS. Auch Talente können dadurch erschwert werden: -1 \textit{Klettern}, -1 \textit{Schwimmen}, -1 \textit{Tanzen}. Es können auch weitere Talente betroffen sein, wenn es der SL für angebracht hält.

\section{Bewegung}
Standardmäßig hat man eine maximale Geschwindigkeit (GS) von 10m pro Aktion (im Sprint). Für jede Behinderung (BE) verringert sich die Geschwindigkeit um 1m. Die mindeste Geschwindigkeit beträgt 2m. Wenn man sich in einer Aktion bewegen und jemanden angreifen will, wird die GS um 50\% reduziert (nur bei Nahkampfangriffen). 

\section{Allgemeine Kampfregeln}
Nahkämpfer haben die Möglichkeit einen normalen Angriff pro Aktion auszuführen. Dabei muss eine Probe auf das Talent passend zur geführten Waffe bestanden werden. Nahkampfangriffe können geblockt, pariert oder ausgewichen werden. Beim Blocken muss gewählt werden, mit welcher Waffe oder welchem Schild geblockt werden soll. Wird ein Angriff erfolgreich geblockt oder wird ihm ausgewichen, erleidet man keinen Schaden. 

Beim Parieren wird zuerst eine Probe auf Blocken oder Ausweichen abgelegt. Misslingt die Probe wird man getroffen und kann nicht mehr zum Gegenschlag ansetzen. D.h., dass die zweite Probe auf Parieren entfällt. Wenn jedoch die Probe gelingt, hat man den Angriff erfolgreich abgewehrt und wirft eine Probe auf Parieren, um dem Angreifer einen Gegenschlag zu versetzen. Gelingt die Parieren-Probe auch, taumelt der Gegner und kann während der gesamten restlichen Runde nicht mehr Ausweichen, Blocken oder Parieren. Misslingt die Parieren-Probe kommt man selber ins taumeln. Das kann man verhindern, indem man erfolgreich auf Körperbeherrschung würfelt. Man kann nur Angriffe von Humanoiden parieren.

Beim Ausweichen hat man auch die Möglichkeit einen Ausweichschritt zu machen, wenn man die Kampfsonderfertigkeit \textit{Fortgeschrittene Kampfkunst} hat. Dabei wird eine Probe auf Ausweichen, erleichtert um 1, abgelegt. Bei Erfolg ist er dem Angriff ausgewichen und kann sich einen Schritt (1-2m) in jede beliebige Richtung bewegen.

Für jeden Angriff nach dem ersten, den man in einer Runde abwehren muss bekommt man eine Erschwernis von -2 auf Blocken (BL), Parieren (PA) und Ausweichen (AW). -1, wenn man die Kampfsonderfertigkeit \textit{Meisterliche Kampfkunst} hat.

Wenn man jemanden würgt, ist er nach 5 Aktionen bewusstlos. Nach weiteren 5 Aktionen ist er tot. Jede Runde (solange er bei Bewusstsein ist) kann der Gewürgte eine Probe auf \textit{Willenskraft} oder \textit{Kampftechnik} ablegen, um sich aus dem Würgegriff zu befreien. Der SL hat die Möglichkeit die Probe zu erschweren, wenn er es aufgrund der Staturen beider Personen für angebracht hält.

\section{Fernkampfregeln}
Fernkämpfer können genauso wie Nahkämpfer einen Angriff pro Aktion ausführen. Jedoch müssen Bögen und Armbrüste nach jedem Schuss wieder geladen werden. Erst wenn die Waffe geladen ist, kann sie abgefeuert werden. Armbrüste können bereits vor dem Kampf geladen werden - Bögen nicht. Ein Bogen benötigt immer 2 Hände, während eine Armbrust mit einer Hand abgefeuert werden kann. Armbrüste müssen jedoch mit zwei Händen nachgeladen werden. 

Bögen verschießen Pfeile und Armbrüste verschießen Bolzen. Pfeile und Bolzen können nur mit Schilden geblockt werden. Das Ausweichen vor Pfeilen und Bolzen bei einer Distanz von maximal 30m ist immer um 2 erschwert. Bei 31-80m Entfernung ist das Ausweichen um 1 erschwert. Ab 81m gibt es keine Erschwernis mehr.

Entscheidet sich ein Fernkämpfer zum Angriff muss er zuerst ein Ziel bestimmen. Je nach Ziel, kann man Boni oder Mali bekommen. Welche Boni oder Mali erhält, hängt von verschiedenen Kriterien ab.

\subsection{Distanz} 
Abhängig von der Waffe gelten unterschiedliche Modifikationen für die Reichweite. Es gibt die Kategorien nah, mittel und weit für die Distanz zum Ziel.

\begin{longtable}{|p{3cm}|p{5cm}|}
\hline
\multicolumn{2}{|l|}{\textbf{Modifikationen durch Reichweite}} \\ \hline
\textbf{Nah} & +2TW, +1TP \\ \hline
\textbf{Mittel} & +/-0TW \\ \hline
\textbf{Weit} & -2TW, -1TP \\ \hline

\caption{Modifikationen durch Reichweite (RW)}
\label{tab:ReichweitenModifikationen}
\end{longtable}

\subsection{Bewegung} 
Beim Fernkampf zu Fuß, kann deine Bewegung oder die des Ziels Einfluss auf die Trefferwahrscheinlichkeit (TW) haben. Stillstehende Ziele sind leichter zu treffen als bewegte Ziele.

\begin{longtable}{|p{6cm}|p{5cm}|}
\hline
\multicolumn{2}{|l|}{\textbf{Modifikationen durch Bewegung}} \\ \hline
\textbf{Ziel steht still} & +2TW \\ \hline
\textbf{Ziel bewegt sich leicht (max. 5m)} & +/-0TW \\ \hline
\textbf{Ziel bewegt sich schnell (min. 6m} & -2TW \\ \hline
\textbf{Ziel schlägt Haken zusätzlich} & -4TW \\
& GS des Ziels halbiert \\ \hline
\textbf{Schütze geht (max. GS/2; Laufen und Schießen in einer Aktion)} & -2TW \\ \hline
\textbf{Schütze rennt (max. GS; Rennen und Schießen in einer Aktion)} & -4TW \\ \hline

\caption{Modifikationen durch Bewegung}
\label{tab:BewegungModifikationen}
\end{longtable}

Für den berittenen Fernkampf gelten besondere Regeln.

\begin{longtable}{|p{4cm}|p{7cm}|}
\hline
\multicolumn{2}{|l|}{\textbf{Modifikationen vom Pferderücken}} \\ 
\multicolumn{2}{|l|}{\textit{(Details in einem separaten Kapitel über Reittiere)}} \\ \hline
\textbf{Tier steht} & +/-0TW \\ \hline
\textbf{Tier im Schritt} & -4TW \\ \hline
\textbf{Tier im Trab} & fast unmöglich (Glückstreffer bei einer gewürfelten 1 auf W20) \\ \hline
\textbf{Tier im Galopp} & -8TW \\ \hline

\caption{Modifikationen vom Pferderücken}
\label{tab:PferderückenModifikationen}
\end{longtable}

\subsection{Größe}
Auch die Größe des Ziels spielt eine wichtige Rolle.

\begin{longtable}{|p{5cm}|p{6cm}|}
\hline
\multicolumn{2}{|l|}{\textbf{Modifikationen durch Grö{\ss}e}} \\ \hline
\textbf{Winzig} (Ratte, Kröte, Spatz, Hase) & -8TW \\ \hline
\textbf{Klein} (Rehkitz, Schaf, Ziege) & -4TW \\ \hline
\textbf{Mittel} (Mensch, Zwerg, Pferd) & +/-0TW \\ \hline
\textbf{Groß} (Troll, Rind) & +4TW \\ \hline
\textbf{Riesig} (Wyvern, Riese) & +8TW \\ \hline

\caption{Modifikationen durch Grö{\ss}e}
\label{tab:GrösseModifikationen}
\end{longtable}

\subsection{Sicht}
Auch die Sicht zum Ziel ist wichtig für die Treffsicherheit. 

\begin{longtable}{|p{5cm}|p{6cm}|}
\hline
\multicolumn{2}{|l|}{\textbf{Modifikationen durch eingeschränkte Sicht}} \\ \hline
\textbf{Gute Sicht} & +/-0TW \\ \hline
\textbf{Eingeschränkte Sicht / Deckung I} (Nebel, Mondlicht, Gestrüpp) & -2TW \\ \hline
\textbf{Ziel nur als Silhouette erkennbar / Deckung II} (Dichter Nebel, Sternenlicht) & -4TW \\ \hline
\textbf{Ziel unsichtbar} & fast unmöglich (Glückstreffer bei einer gewürfelten 1 auf W20) \\ \hline

\caption{Modifikationen durch eingeschränkte Sicht}
\label{tab:SichtModifikationen}
\end{longtable}


\section{Status und Sonderaktionen}
\begin{longtable}{|p{4cm}|p{5cm}|p{6.5cm}|}
\hline
\textbf{Aktion} & \textbf{Schwierigkeit Mod.} & \textbf{Dauer} \\

\hline
\textbf{Trank nehmen} & - & 2AK (+1AK aus Inventar holen) \\

\hline
\textbf{Wunde/Blutung versorgen} & - & 2AK (+1AK falls Hilfsmittel aus dem Inventar geholt werden muss. Z.B. Verbandszeug) \\

\hline
\caption{Zusätzliche Aktionen im Kampf}
\label{tab:ZusätzlicheAktionen}
\end{longtable}

\begin{longtable}{|p{4cm}|p{12cm}|}
\hline
\textbf{Status} & \textbf{Schwierigkeit Mod.} \\

\hline
\textbf{Liegend} & -4AW, -4BL, 25\%GS \\
                 & alle anderen Angriffs- und Verteidigungs-Aktionen sind im liegenden Zustand nicht möglich. Wenn man eine liegende Person angreift, bekommt man eine Erleichterung von 4. \\

\hline
\textbf{Kriechend (auch geduckt)} & -2TW Angriff., -2AW, -2BL, 50\%GS \\
                 & alle anderen Angriffs- und Verteidigungs-Aktionen sind im kriechenden/geduckten Zustand nicht möglich. Wenn man eine kriechende Person angreift, bekommt man eine Erleichterung von 2. \\
      
\hline
\textbf{Deckung I} (einfache Deckung. Z.B. Baum oder kleiner Fels) & Erschwertes Ziel für Fernkämpfer $\rightarrow$ -2TW für Angreifer. \\

\hline
\textbf{Deckung II} (gute Deckung. Z.B. großer Baum oder großer Fels) & Erschwertes Ziel für Fernkämpfer $\rightarrow$ -4TW für Angreifer. \\

\hline
\textbf{Taumeln} & Wenn eine Person taumelt, kann sie nichts mehr machen. Weder Ausweichen, Blocken oder Parieren. Sobald eine Person das Gleichgewicht verliert, kann sie eine Probe auf \textit{Körperbeherrschung} ablegen, um sich zu stabilisieren. Gelingt es nicht, wird die Probe in jeder Runde wiederholt, bis sie gelingt (oder man tot ist oder hinfällt). Wenn eine taumelnde Person Schaden bekommt, fällt sie hin $\rightarrow$ der Status \textit{Taumeln} wird durch \textit{Liegend} ersetzt. Wenn man eine taumelnde Person angreift, bekommt man eine Erleichterung von 4. \\

\hline
\textbf{Blutung} & -1TP pro AK. WD: 10AK. Kann z.B. durch einen Verband gestoppt werden. \\

\hline
\textbf{Starke Blutung} & -2TP pro AK. WD: 15AK. Kann z.B. durch einen Verband gestoppt werden. \\

\hline
\textbf{Leicht Verwundet} & Passiert nichts. Mehrere leichte Wunden können zu einer mittelschweren Verwundung führen. \\

\hline
\textbf{Mittelschwer Verwundet} & Kann in manchen Situationen zu Erschwernissen führen. Wird vom SL bestimmt. Kann auch den Status Blutend herbeiführen. \\

\hline
\textbf{Stark Verwundet} & Je nach Wunde, kann z.B. ein Körperteil nicht mehr verwendet werden. Kann zu starker Blutung, Ohnmacht oder auch zum Tod führen. \\

\hline
\caption{Liste aller Status}
\label{tab:Status}
\end{longtable}

Von \textit{Liegend} auf \textit{Kriechend} benötigt man eine Kampfaktion (AK). Eine weitere AK wird benötigt um aufzustehen: 

\begin{itemize}
\item \textit{Liegend} +2AK $\rightarrow$ \textit{Stehend}
\item \textit{Liegend} +1AK $\rightarrow$ \textit{Kriechend}
\item \textit{Kriechend} +1AK $\rightarrow$ \textit{Stehend}
\item \textit{Stehend} +1AK $\rightarrow$ \textit{Liegend} (hinschmeißen)
\item \textit{Stehend} +0AK $\rightarrow$ \textit{Kriechend}
\end{itemize}

\textit{Deckung I} \& \textit{II} hilf nur gegen magische und physische Fernkampfangriffe und nicht gegen Nahkampfangriffe.


\section{Kampfsonderfertigkeiten}
Es gibt mehrere Sonderfertigkeiten für den Kampf, die diverse Boni geben. Manche Klassen haben von vornerein bestimmte Kampfsonderfertigkeiten. Bei anderen Klassen kann sich der Spielen auch eine bestimmte Anzahl aussuchen.

\begin{longtable}{| p{4cm} | p{8cm} | p{3cm} |}
\hline
\textbf{Name} & \textbf{Beschreibung} & \textbf{Anmerkung} \\

\hline
\textbf{Fortgeschrittene Kampfkunst} & Ermöglicht die Anwendung von Kampftechniken, sowie\textit{Auchweichrolle} und \textit{Parieren} & Kann nur von kampfbezogenen Klassen verwendet werden. +6 auf \textit{Raufen} und \textit{Kampftechnik}. \\

\hline 
\textbf{Meisterliche Kampfkunst} & Die Erschwernis beim Standhalten von aufeinander folgenden Schlägen wird anstatt um 2, um 1 pro Stufe erhöht. Zusätzlich werden Proben auf \textit{Kampftechniken} um 2 erleichtert. & \textit{Forgeschrittene Kampfkunst} muss bereits erlernt sein. \\

\hline
\textbf{Schwert-kampfkunst~I-III} & Bonus auf Angriffe mit Schwertern: +1 pro Stufe & höhere Stufen sind nur durch das Aufleveln erreichbar \\

\hline
\textbf{Präzisionsschuss~I-III} & Erschwernis um 2 pro Stufe und +2 TP pro Stufe & \\

\hline
\textbf{Schneller Schuss} & Verkürzt die Ladezeit und benötigte Zeit zum Ziehen der Fernkampfwaffe um 1 & \\

\hline
\textbf{Ruhige Hand~I-II} & Bonus auf Angriffe mit Fernkampfwaffen: +1 pro Stufe & \\

\hline
\textbf{Berittener Schütze} & Geht das Reittier (\textit{Schritt}), ist das Schießen nicht mehr erschwert. Im Galopp ist es nur noch um 4 erschwert. Der Schütze kann zudem aus vollem Galopp nach hinten schießen. Im Trab sind weiterhin Glückstreffer möglich (also bei einer 1 auf W20). Zur Benutzung beider Hände muss keine Probe abgelegt werden und es kann auch im Galopp geschossen werden. & \\

\hline
\textbf{Faustkampftechnik I-II} & Bonus beim Raufen: +1 pro Stufe & \\

\hline
\textbf{Ausweichen I-III} & Erleichtert das Ausweichen um +2 pro Stufe & \\

\hline
\textbf{Parieren I-III} & Erleichtert das Parieren um +1 pro Stufe & \\

\hline
\textbf{Diener der Dunkelheit I-III} & Stufe+1W4 TP zusätzlich in der Nacht. Gilt auch für Fähigkeiten, die Schaden verursachen. & Können nur von Magiebegabten verwendet werden. \\

\hline
\textbf{Diener des Lichts I-III} & Stufe+1W4 TP zusätzlich am Tag. Gilt auch für Fähigkeiten, die Schaden verursachen. & Können nur von Magiebegabten verwendet werden. \\

\hline
\caption{Kampfsonderfertigkeiten}
\label{tab:Kampfsonderfertigkeiten}
\end{longtable}


\section{Kampftechniken}
Um spezielle Kampftechniken anwenden zu können benötigt man den Vorteil \textit{fortgeschrittene Kampfkunst}. Alle Kampftechniken können mit allen Waffen oder Fäusten angewandt werden. Für Fäuste gilt: -6TP. Ob der Held trifft und wie viel Schaden er dabei verursacht wird genauso ausgewürfelt, wie bei einem normalen Angriff. Durch Kampftechniken können lediglich Werte, wie der Schaden oder die Anzahl getroffener Gegner erhöht werden. Dazu vor dem normalen Schalg eine separate Probe abgelegt werden. Wie die Probe im Detail aussieht legt jede Kampftechnik für sich fest.

\subsection{Rundumschlag}
Trifft beliebig viele Gegner um sich herum. Wie weit sich der Held bei diesem Schlag drehen soll, entscheidet der Spieler. Er kann also auch nur 3 nebeneinander stehende Gegner angreifen, die in seiner Reichweite sind, obwohl 5 Gegner um ihn herum stehen. Richtwerte lauten folgendermaßen: 1 Gegner = 90°, 2 Gegner = 180°, 3 Gegner = 270°, 4 Gegner = 360°. Zu beachten ist, dass durch diese Technik auch verbündete getroffen werden können.

Um den \textit{Rundumschlag} auszuführen muss eine Probe auf \textit{Kampftechnik} bestanden werden. Je nachdem wie weit sich der Held dabei dreht, entstehen verschiedene Effekte.

\textbf{0°-45°:} - \\
\textbf{46°-180°:} +1W6 TP durch den zusätzlichen Schwung. \\
\textbf{181°-270°:} 2+1W6 TP und zusätzliche Probe auf \textit{Körperbeherrschung} (Gleichgewicht). Misslingt die Probe gerät der Held ins Taumeln. \\
\textbf{271°-360°:} 4+1W6 TP und zusätzliche Probe auf \textit{Körperbeherrschung} (Gleichgewicht). Misslingt die Probe gerät der Held ins Taumeln. \\