\section{Herbarium}
Hier sind alle Ingredienzen aufgeführt, die für die Herstellung von Gebräue wie Tränke, Öle oder Bomben benutzt werden können. Ingredienzen können eine \textit{Pflückprobe} und \textit{Utensilien} definieren. Die \textit{Pflückprobe} gibt an wie eine Probe auf \textit{Alchemie} erfüllt werden muss, damit die Ingredienz erfolgreich gepflückt wird. \textit{Utensilien} geben an was zum Pflücken benötigt wird. Wenn nicht anders angegeben, muss die \textit{Pflückprobe} ausgeführt werden, wenn kein angegebenes \textit{Utensil} verwendet wird. Wenn die \textit{Pflückprobe} misslingt wird die Ingredienz zerstört (wenn nicht anders angegeben).

\subsection{Alraunenwurzel}
% Beschreibung der Frucht. Hier kommen auch Erntehinweise (Dornen, empfindliche Frucht) rein. Die benötigten Utensilien, wie beispielsweise Sichel oder Handschuhe kommen in die untere Tabelle und nicht in die Beschreibung. 

Die Alraunenwurzel, auch Mandragora oder Mannswurz genannt ist eine giftige Pflanze, enthält aber magische Eigenschaften. Druiden verwenden diese Wurzel häufig.

\begin{table}[h]
\begin{center}
\begin{tabular}{|l|l|p{1cm}|l|l|}
	\cline{1-2} \cline{4-5}
	\textbf{Hauptwirkstoff} & Quebrith && \textbf{Pflückprobe} & QS II \\ \cline{1-2} \cline{4-5}
	\textbf{Sekundärwirkstoff} & Nigredo && \textbf{Menge} & 1 pro Pflanze \\ \cline{1-2} \cline{4-5}
	\textbf{Farbe} & braun (2) && \textbf{Vorkommen} & Wald (häufig) \\ \cline{1-2} \cline{4-5}
	\textbf{Geruch} & neutral (0) && \textbf{Region} & gemäßigte Temperaturen \\ \cline{1-2} \cline{4-5}
	\textbf{Geschmack} & bitter-süß (2) && \textbf{Wert} & 15Kr \\ \cline{1-2} \cline{4-5}
	\textbf{Spezialeigenschaften} & Gift && \textbf{Utensilien} & (Hand)spaten \\ \cline{1-2} \cline{4-5}
\end{tabular}
\end{center}
\caption{Alraunenwurzel}
\label{tab:alraunenwurzel}
\end{table}

\subsection{Balissafrucht}
Balissafrucht ist eine essbare Frucht die am Balissastrauch wächst. Sie zeichnet sich durch eine zarte magische Resonanz aus. 

\begin{table}[h]
\begin{center}
\begin{tabular}{|l|l|p{1cm}|l|l|}
	\cline{1-2} \cline{4-5}
	\textbf{Hauptwirkstoff} & Quebrith && \textbf{Pflückprobe} & - \\ \cline{1-2} \cline{4-5}
	\textbf{Sekundärwirkstoff} & Rubedo && \textbf{Menge} & 2+1W6 pro Strauch \\ \cline{1-2} \cline{4-5}
	\textbf{Farbe} & violett (2) && \textbf{Vorkommen} & Feld (mittel), Wald (häufig) \\ \cline{1-2} \cline{4-5}
	\textbf{Geruch} & fruchtig (1) && \textbf{Region} & überall \\ \cline{1-2} \cline{4-5}
	\textbf{Geschmack} & süß (2) && \textbf{Wert} & 12Kr \\ \cline{1-2} \cline{4-5}
	\textbf{Spezialeigenschaften} & - && \textbf{Utensilien} & - \\ \cline{1-2} \cline{4-5}
\end{tabular}
\end{center}
\caption{Balissafrucht}
\label{tab:balissafrucht}
\end{table}


\subsection{Berberrohrfrucht}
Die Berberrohrfrucht schmeckt sauer und ist saftig. Sie wächst an einem Strauch mit Dornen. Der Strauch wird bis zu 1,5m hoch. 

Misslingt die \textit{Pflückprobe} (aber \textit{Alchemie}-Probe gelingt) bekommt der Held 1 Schaden pro QS die er zu niedrig ist. Misslingt die \textit{Alchemie}-Probe fällt der Held in den Strauch und bekommt zusätzlich 1 Schaden pro Wert den er unterschritten hat. 

Bsp.: Misslingt die Probe um 2 bekommt er 2 Schaden und zusätzlich 3 Schaden wegen der nicht erreichten QS IV.

\newpage

\begin{table}[h]
\begin{center}
\begin{tabular}{|l|l|p{1cm}|l|l|}
	\cline{1-2} \cline{4-5}
	\textbf{Hauptwirkstoff} & Äther && \textbf{Pflückprobe} & QS IV \\ \cline{1-2} \cline{4-5}
	\textbf{Sekundärwirkstoff} & Albedo && \textbf{Menge} & 1W12 pro Strauch \\ \cline{1-2} \cline{4-5}
	\textbf{Farbe} & grün (2) && \textbf{Vorkommen} & Feld (häufig) \\ \cline{1-2} \cline{4-5}
	\textbf{Geruch} & neutral && \textbf{Region} & überall \\ \cline{1-2} \cline{4-5}
	\textbf{Geschmack} & sauer (2) && \textbf{Wert} & 21Kr \\ \cline{1-2} \cline{4-5}
	\textbf{Spezialeigenschaften} & - && \textbf{Utensilien} & Handschuh \\ \cline{1-2} \cline{4-5}
\end{tabular}
\end{center}
\caption{Berberrohrfrucht}
\label{tab:berberrohrfrucht}
\end{table}


\subsection{Büschelkrautblüten}
Mit den auffällig roten Blüten ist Büschelkraut eine weitverbreitete Pflanze, die auf Wiesen und feuchten Böden wächst. Die Blütenblätter enthalten ein leichtes Halluzinogen, sind giftig und werden für Fisstech verwendet. 

\begin{table}[h] 
\begin{center} 
\begin{tabular}{|l|l|p{1cm}|l|l|} 
  	\cline{1-2} \cline{4-5} 
  	\textbf{Hauptwirkstoff} & Hydragenum && \textbf{Pflückprobe} & - \\ \cline{1-2} \cline{4-5} 
  	\textbf{Sekundärwirkstoff} & Rubedo && \textbf{Menge} & 1 pro Pflanze \\ \cline{1-2} \cline{4-5} 
  	\textbf{Farbe} & rot (3) && \textbf{Vorkommen} & \brcell{Sumpf (selten)\\Feld (selten)} \\ \cline{1-2} \cline{4-5} 
  	\textbf{Geruch} & büschelkraut (2) && \textbf{Region} & überall \\ \cline{1-2} \cline{4-5} 
  	\textbf{Geschmack} & bitter (3) && \textbf{Wert} & 15Kr \\ \cline{1-2} \cline{4-5} 
  	\textbf{Spezialeigenschaften} & Gift, Halluzinogen && \textbf{Utensilien} & - \\ \cline{1-2} \cline{4-5} 
\end{tabular} 
\end{center} 
\caption{Büschelkrautblüten} 
\label{tab:bueschelkrautblueten} 
\end{table}


\subsection{Eisenkraut}
Eisenkraut ist eine weit verbreitete, blühende Pflanze. 

\begin{table}[h] 
\begin{center} 
\begin{tabular}{|l|l|p{1cm}|l|l|} 
  	\cline{1-2} \cline{4-5} 
  	\textbf{Hauptwirkstoff} & Quebrith && \textbf{Pflückprobe} & - \\ \cline{1-2} \cline{4-5} 
  	\textbf{Sekundärwirkstoff} & Albedo && \textbf{Menge} & 1 pro Pflanze \\ \cline{1-2} \cline{4-5} 
  	\textbf{Farbe} & rosa (1) && \textbf{Vorkommen} & Ödland (häufig) \\ \cline{1-2} \cline{4-5} 
  	\textbf{Geruch} & eisenkraut (1) && \textbf{Region} & überall \\ \cline{1-2} \cline{4-5} 
  	\textbf{Geschmack} & bitter (1) && \textbf{Wert} & 15Kr \\ \cline{1-2} \cline{4-5} 
  	\textbf{Spezialeigenschaften} & - && \textbf{Utensilien} & - \\ \cline{1-2} \cline{4-5} 
\end{tabular} 
\end{center} 
\caption{Eisenkraut} 
\label{tab:eisenkraut} 
\end{table}


\subsection{Feainnwedd}
Feainnwedd ist eine Blume mit wundervollem Duft, die einer Legende nach nur an Orten wächst, an dem \textit{Älteren-Blut} vergossen wurde sowie im \textit{Tal der Blumen}. Die Elfen erzählen, dass Feainnwedd erst zu wachsen begann, nachdem \textit{Lara Dorren} gestorben war.

In der alten Sprache bedeutet Faen "`Sonne"' und wedd "`Kind"', was hier möglicherweise "`Kind der Sonne"' oder "`Sonnenkind"' heißt. 

\begin{table}[h] 
\begin{center} 
\begin{tabular}{|l|l|p{1cm}|l|l|} 
  	\cline{1-2} \cline{4-5} 
  	\textbf{Hauptwirkstoff} & Karmin && \textbf{Pflückprobe} & - \\ \cline{1-2} \cline{4-5} 
  	\textbf{Sekundärwirkstoff} & Rubedo && \textbf{Menge} & 1 pro Pflanze \\ \cline{1-2} \cline{4-5} 
  	\textbf{Farbe} & lila (2) && \textbf{Vorkommen} & \brcell{siehe Beschreibung \\ (selten)} \\ \cline{1-2} \cline{4-5} 
  	\textbf{Geruch} & Feainnwedd (2) && \textbf{Region} & siehe Beschreibung \\ \cline{1-2} \cline{4-5} 
  	\textbf{Geschmack} & neutral && \textbf{Wert} & 50Kr \\ \cline{1-2} \cline{4-5} 
  	\textbf{Spezialeigenschaften} & - && \textbf{Utensilien} & - \\ \cline{1-2} \cline{4-5} 
\end{tabular} 
\end{center} 
\caption{Feainnwedd} 
\label{tab:feainnwedd} 
\end{table}


\subsection{Geisblatt}
Geisblatt nennen sich die Blätter einer widerstandsfähigen Pflanze die vorzugsweise auf Ödland wächst als auch auf Wiesen und Feldern. 

\begin{table}[h] 
\begin{center} 
\begin{tabular}{|l|l|p{1cm}|l|l|} 
  	\cline{1-2} \cline{4-5} 
  	\textbf{Hauptwirkstoff} & Quebrith && \textbf{Pflückprobe} & - \\ \cline{1-2} \cline{4-5} 
  	\textbf{Sekundärwirkstoff} & Albedo && \textbf{Menge} & 1 pro Pflanze \\ \cline{1-2} \cline{4-5} 
  	\textbf{Farbe} & grün (2) && \textbf{Vorkommen} & \brcell{Ödland (mittel) \\ Feld (mittel)} \\ \cline{1-2} \cline{4-5} 
  	\textbf{Geruch} & neutral && \textbf{Region} & \brcell{überall \\ Brokilon (häufig)} \\ \cline{1-2} \cline{4-5} 
  	\textbf{Geschmack} & neutral && \textbf{Wert} & 21Kr \\ \cline{1-2} \cline{4-5} 
  	\textbf{Spezialeigenschaften} & - && \textbf{Utensilien} & - \\ \cline{1-2} \cline{4-5} 
\end{tabular} 
\end{center} 
\caption{Geisblatt} 
\label{tab:geisblatt} 
\end{table}


\subsection{Grünschimmel}
Grünschimmel wächst an den Wänden dunkler und feuchter Orte, wie z. B. in natürlichen Höhlen oder auch in Abwasserkanälen. 

Zum Pflücken benötigt man etwas zum abscharben, z.B. ein Messer.

\begin{table}[h] 
\begin{center} 
\begin{tabular}{|l|l|p{1cm}|l|l|} 
  	\cline{1-2} \cline{4-5} 
  	\textbf{Hauptwirkstoff} & Rebis && \textbf{Pflückprobe} & QS 3 \\ \cline{1-2} \cline{4-5} 
  	\textbf{Sekundärwirkstoff} & Rubedo && \textbf{Menge} & 2W4 pro Ort \\ \cline{1-2} \cline{4-5} 
  	\textbf{Farbe} & grün (2) && \textbf{Vorkommen} & \brcell{Höhle (häufig) \\ Sumpf (selten)} \\ \cline{1-2} \cline{4-5} 
  	\textbf{Geruch} & übelriechend (3) && \textbf{Region} & überall \\ \cline{1-2} \cline{4-5} 
  	\textbf{Geschmack} & schimmel (2) && \textbf{Wert} & 15Kr \\ \cline{1-2} \cline{4-5} 
  	\textbf{Spezialeigenschaften} & - && \textbf{Utensilien} & Messer \\ \cline{1-2} \cline{4-5} 
\end{tabular} 
\end{center} 
\caption{Grünschimmel} 
\label{tab:gruenschimmel} 
\end{table}


\subsection{Hanfasern}
Eisenkraut ist eine weit verbreitete, blühende Pflanze. 

\begin{table}[h]
\begin{center}
\begin{tabular}{|l|l|p{1cm}|l|l|}
	\cline{1-2} \cline{4-5}
	\textbf{Hauptwirkstoff} &  && \textbf{Pflückprobe} &  \\ \cline{1-2} \cline{4-5}
	\textbf{Sekundärwirkstoff} &  && \textbf{Menge} & 1 pro Pflanze \\ \cline{1-2} \cline{4-5}
	\textbf{Farbe} &  && \textbf{Vorkommen} &  \\ \cline{1-2} \cline{4-5}
	\textbf{Geruch} &  && \textbf{Region} & überall \\ \cline{1-2} \cline{4-5}
	\textbf{Geschmack} &  && \textbf{Wert} & Kr \\ \cline{1-2} \cline{4-5}
	\textbf{Spezialeigenschaften} &  && \textbf{Utensilien} &  \\ \cline{1-2} \cline{4-5}
\end{tabular}
\end{center}
\caption{Hanfasern}
\label{tab:hanfasern}
\end{table}


\subsection{Hopfendolden}
Eisenkraut ist eine weit verbreitete, blühende Pflanze. 

\begin{table}[h]
\begin{center}
\begin{tabular}{|l|l|p{1cm}|l|l|}
	\cline{1-2} \cline{4-5}
	\textbf{Hauptwirkstoff} &  && \textbf{Pflückprobe} &  \\ \cline{1-2} \cline{4-5}
	\textbf{Sekundärwirkstoff} &  && \textbf{Menge} & 1 pro Pflanze \\ \cline{1-2} \cline{4-5}
	\textbf{Farbe} &  && \textbf{Vorkommen} &  \\ \cline{1-2} \cline{4-5}
	\textbf{Geruch} &  && \textbf{Region} & überall \\ \cline{1-2} \cline{4-5}
	\textbf{Geschmack} &  && \textbf{Wert} & Kr \\ \cline{1-2} \cline{4-5}
	\textbf{Spezialeigenschaften} &  && \textbf{Utensilien} &  \\ \cline{1-2} \cline{4-5}
\end{tabular}
\end{center}
\caption{Hopfendolden}
\label{tab:hopfendolden}
\end{table}


\subsection{Hundspetersilie}
Eisenkraut ist eine weit verbreitete, blühende Pflanze. 

\begin{table}[h]
\begin{center}
\begin{tabular}{|l|l|p{1cm}|l|l|}
	\cline{1-2} \cline{4-5}
	\textbf{Hauptwirkstoff} &  && \textbf{Pflückprobe} &  \\ \cline{1-2} \cline{4-5}
	\textbf{Sekundärwirkstoff} &  && \textbf{Menge} & 1 pro Pflanze \\ \cline{1-2} \cline{4-5}
	\textbf{Farbe} &  && \textbf{Vorkommen} &  \\ \cline{1-2} \cline{4-5}
	\textbf{Geruch} &  && \textbf{Region} & überall \\ \cline{1-2} \cline{4-5}
	\textbf{Geschmack} &  && \textbf{Wert} & Kr \\ \cline{1-2} \cline{4-5}
	\textbf{Spezialeigenschaften} &  && \textbf{Utensilien} &  \\ \cline{1-2} \cline{4-5}
\end{tabular}
\end{center}
\caption{Hundspetersilie}
\label{tab:hundspetersilie}
\end{table}


\subsection{Ignatia Blüten}
Eisenkraut ist eine weit verbreitete, blühende Pflanze. 

\begin{table}[h]
\begin{center}
\begin{tabular}{|l|l|p{1cm}|l|l|}
	\cline{1-2} \cline{4-5}
	\textbf{Hauptwirkstoff} &  && \textbf{Pflückprobe} &  \\ \cline{1-2} \cline{4-5}
	\textbf{Sekundärwirkstoff} &  && \textbf{Menge} & 1 pro Pflanze \\ \cline{1-2} \cline{4-5}
	\textbf{Farbe} &  && \textbf{Vorkommen} &  \\ \cline{1-2} \cline{4-5}
	\textbf{Geruch} &  && \textbf{Region} & überall \\ \cline{1-2} \cline{4-5}
	\textbf{Geschmack} &  && \textbf{Wert} & Kr \\ \cline{1-2} \cline{4-5}
	\textbf{Spezialeigenschaften} &  && \textbf{Utensilien} &  \\ \cline{1-2} \cline{4-5}
\end{tabular}
\end{center}
\caption{Ignatia Blüten}
\label{tab:ignatia_blueten}
\end{table}


\subsection{Krähenauge}
Eisenkraut ist eine weit verbreitete, blühende Pflanze. 

\begin{table}[h]
\begin{center}
\begin{tabular}{|l|l|p{1cm}|l|l|}
	\cline{1-2} \cline{4-5}
	\textbf{Hauptwirkstoff} &  && \textbf{Pflückprobe} &  \\ \cline{1-2} \cline{4-5}
	\textbf{Sekundärwirkstoff} &  && \textbf{Menge} & 1 pro Pflanze \\ \cline{1-2} \cline{4-5}
	\textbf{Farbe} &  && \textbf{Vorkommen} &  \\ \cline{1-2} \cline{4-5}
	\textbf{Geruch} &  && \textbf{Region} & überall \\ \cline{1-2} \cline{4-5}
	\textbf{Geschmack} &  && \textbf{Wert} & Kr \\ \cline{1-2} \cline{4-5}
	\textbf{Spezialeigenschaften} &  && \textbf{Utensilien} &  \\ \cline{1-2} \cline{4-5}
\end{tabular}
\end{center}
\caption{Krähenauge}
\label{tab:kraehenauge}
\end{table}


\subsection{Mistelzweig}
Eisenkraut ist eine weit verbreitete, blühende Pflanze. 

\begin{table}[h]
\begin{center}
\begin{tabular}{|l|l|p{1cm}|l|l|}
	\cline{1-2} \cline{4-5}
	\textbf{Hauptwirkstoff} &  && \textbf{Pflückprobe} &  \\ \cline{1-2} \cline{4-5}
	\textbf{Sekundärwirkstoff} &  && \textbf{Menge} & 1 pro Pflanze \\ \cline{1-2} \cline{4-5}
	\textbf{Farbe} &  && \textbf{Vorkommen} &  \\ \cline{1-2} \cline{4-5}
	\textbf{Geruch} &  && \textbf{Region} & überall \\ \cline{1-2} \cline{4-5}
	\textbf{Geschmack} &  && \textbf{Wert} & Kr \\ \cline{1-2} \cline{4-5}
	\textbf{Spezialeigenschaften} &  && \textbf{Utensilien} &  \\ \cline{1-2} \cline{4-5}
\end{tabular}
\end{center}
\caption{Mistelzweig}
\label{tab:mistelzweig}
\end{table}


\subsection{Mutterkornsamen}
Eisenkraut ist eine weit verbreitete, blühende Pflanze. 

\begin{table}[h]
\begin{center}
\begin{tabular}{|l|l|p{1cm}|l|l|}
	\cline{1-2} \cline{4-5}
	\textbf{Hauptwirkstoff} &  && \textbf{Pflückprobe} &  \\ \cline{1-2} \cline{4-5}
	\textbf{Sekundärwirkstoff} &  && \textbf{Menge} & 1 pro Pflanze \\ \cline{1-2} \cline{4-5}
	\textbf{Farbe} &  && \textbf{Vorkommen} &  \\ \cline{1-2} \cline{4-5}
	\textbf{Geruch} &  && \textbf{Region} & überall \\ \cline{1-2} \cline{4-5}
	\textbf{Geschmack} &  && \textbf{Wert} & Kr \\ \cline{1-2} \cline{4-5}
	\textbf{Spezialeigenschaften} &  && \textbf{Utensilien} &  \\ \cline{1-2} \cline{4-5}
\end{tabular}
\end{center}
\caption{Mutterkornsamen}
\label{tab:mutterkornsamen}
\end{table}


\subsection{Nieswurzblüten}
Eisenkraut ist eine weit verbreitete, blühende Pflanze. 

\begin{table}[h]
\begin{center}
\begin{tabular}{|l|l|p{1cm}|l|l|}
	\cline{1-2} \cline{4-5}
	\textbf{Hauptwirkstoff} &  && \textbf{Pflückprobe} &  \\ \cline{1-2} \cline{4-5}
	\textbf{Sekundärwirkstoff} &  && \textbf{Menge} & 1 pro Pflanze \\ \cline{1-2} \cline{4-5}
	\textbf{Farbe} &  && \textbf{Vorkommen} &  \\ \cline{1-2} \cline{4-5}
	\textbf{Geruch} &  && \textbf{Region} & überall \\ \cline{1-2} \cline{4-5}
	\textbf{Geschmack} &  && \textbf{Wert} & Kr \\ \cline{1-2} \cline{4-5}
	\textbf{Spezialeigenschaften} &  && \textbf{Utensilien} &  \\ \cline{1-2} \cline{4-5}
\end{tabular}
\end{center}
\caption{Nieswurzblüten}
\label{tab:nieswurzblueten}
\end{table}


\subsection{Pimentzurzel}
Eisenkraut ist eine weit verbreitete, blühende Pflanze. 

\begin{table}[h]
\begin{center}
\begin{tabular}{|l|l|p{1cm}|l|l|}
	\cline{1-2} \cline{4-5}
	\textbf{Hauptwirkstoff} &  && \textbf{Pflückprobe} &  \\ \cline{1-2} \cline{4-5}
	\textbf{Sekundärwirkstoff} &  && \textbf{Menge} & 1 pro Pflanze \\ \cline{1-2} \cline{4-5}
	\textbf{Farbe} &  && \textbf{Vorkommen} &  \\ \cline{1-2} \cline{4-5}
	\textbf{Geruch} &  && \textbf{Region} & überall \\ \cline{1-2} \cline{4-5}
	\textbf{Geschmack} &  && \textbf{Wert} & Kr \\ \cline{1-2} \cline{4-5}
	\textbf{Spezialeigenschaften} &  && \textbf{Utensilien} &  \\ \cline{1-2} \cline{4-5}
\end{tabular}
\end{center}
\caption{Pimentzurzel}
\label{tab:pimentzurzel}
\end{table}


\subsection{Schöllkraut}
Eisenkraut ist eine weit verbreitete, blühende Pflanze. 

\begin{table}[h]
\begin{center}
\begin{tabular}{|l|l|p{1cm}|l|l|}
	\cline{1-2} \cline{4-5}
	\textbf{Hauptwirkstoff} &  && \textbf{Pflückprobe} &  \\ \cline{1-2} \cline{4-5}
	\textbf{Sekundärwirkstoff} &  && \textbf{Menge} & 1 pro Pflanze \\ \cline{1-2} \cline{4-5}
	\textbf{Farbe} &  && \textbf{Vorkommen} &  \\ \cline{1-2} \cline{4-5}
	\textbf{Geruch} &  && \textbf{Region} & überall \\ \cline{1-2} \cline{4-5}
	\textbf{Geschmack} &  && \textbf{Wert} & Kr \\ \cline{1-2} \cline{4-5}
	\textbf{Spezialeigenschaften} &  && \textbf{Utensilien} &  \\ \cline{1-2} \cline{4-5}
\end{tabular}
\end{center}
\caption{Schöllkraut}
\label{tab:schoellkraut}
\end{table}


\subsection{Sewanten}
Eisenkraut ist eine weit verbreitete, blühende Pflanze. 

\begin{table}[h]
\begin{center}
\begin{tabular}{|l|l|p{1cm}|l|l|}
	\cline{1-2} \cline{4-5}
	\textbf{Hauptwirkstoff} &  && \textbf{Pflückprobe} &  \\ \cline{1-2} \cline{4-5}
	\textbf{Sekundärwirkstoff} &  && \textbf{Menge} & 1 pro Pflanze \\ \cline{1-2} \cline{4-5}
	\textbf{Farbe} &  && \textbf{Vorkommen} &  \\ \cline{1-2} \cline{4-5}
	\textbf{Geruch} &  && \textbf{Region} & überall \\ \cline{1-2} \cline{4-5}
	\textbf{Geschmack} &  && \textbf{Wert} & Kr \\ \cline{1-2} \cline{4-5}
	\textbf{Spezialeigenschaften} &  && \textbf{Utensilien} &  \\ \cline{1-2} \cline{4-5}
\end{tabular}
\end{center}
\caption{Sewanten}
\label{tab:sewanten}
\end{table}


\subsection{Wolfsaloe}
Eisenkraut ist eine weit verbreitete, blühende Pflanze. 

\begin{table}[h]
\begin{center}
\begin{tabular}{|l|l|p{1cm}|l|l|}
	\cline{1-2} \cline{4-5}
	\textbf{Hauptwirkstoff} &  && \textbf{Pflückprobe} &  \\ \cline{1-2} \cline{4-5}
	\textbf{Sekundärwirkstoff} &  && \textbf{Menge} & 1 pro Pflanze \\ \cline{1-2} \cline{4-5}
	\textbf{Farbe} &  && \textbf{Vorkommen} &  \\ \cline{1-2} \cline{4-5}
	\textbf{Geruch} &  && \textbf{Region} & überall \\ \cline{1-2} \cline{4-5}
	\textbf{Geschmack} &  && \textbf{Wert} & Kr \\ \cline{1-2} \cline{4-5}
	\textbf{Spezialeigenschaften} &  && \textbf{Utensilien} &  \\ \cline{1-2} \cline{4-5}
\end{tabular}
\end{center}
\caption{Wolfsaloe}
\label{tab:wolfsaloe}
\end{table}


\subsection{Wolfsbann}
Eisenkraut ist eine weit verbreitete, blühende Pflanze. 

\begin{table}[h]
\begin{center}
\begin{tabular}{|l|l|p{1cm}|l|l|}
	\cline{1-2} \cline{4-5}
	\textbf{Hauptwirkstoff} &  && \textbf{Pflückprobe} &  \\ \cline{1-2} \cline{4-5}
	\textbf{Sekundärwirkstoff} &  && \textbf{Menge} & 1 pro Pflanze \\ \cline{1-2} \cline{4-5}
	\textbf{Farbe} &  && \textbf{Vorkommen} &  \\ \cline{1-2} \cline{4-5}
	\textbf{Geruch} &  && \textbf{Region} & überall \\ \cline{1-2} \cline{4-5}
	\textbf{Geschmack} &  && \textbf{Wert} & Kr \\ \cline{1-2} \cline{4-5}
	\textbf{Spezialeigenschaften} &  && \textbf{Utensilien} &  \\ \cline{1-2} \cline{4-5}
\end{tabular}
\end{center}
\caption{Wolfsbann}
\label{tab:wolfsbann}
\end{table}


\subsection{Zaunrübenwurzel}
Eisenkraut ist eine weit verbreitete, blühende Pflanze. 

\begin{table}[h]
\begin{center}
\begin{tabular}{|l|l|p{1cm}|l|l|}
	\cline{1-2} \cline{4-5}
	\textbf{Hauptwirkstoff} &  && \textbf{Pflückprobe} &  \\ \cline{1-2} \cline{4-5}
	\textbf{Sekundärwirkstoff} &  && \textbf{Menge} & 1 pro Pflanze \\ \cline{1-2} \cline{4-5}
	\textbf{Farbe} &  && \textbf{Vorkommen} &  \\ \cline{1-2} \cline{4-5}
	\textbf{Geruch} &  && \textbf{Region} & überall \\ \cline{1-2} \cline{4-5}
	\textbf{Geschmack} &  && \textbf{Wert} & Kr \\ \cline{1-2} \cline{4-5}
	\textbf{Spezialeigenschaften} &  && \textbf{Utensilien} &  \\ \cline{1-2} \cline{4-5}
\end{tabular}
\end{center}
\caption{Zaunrübenwurzel}
\label{tab:zaunruebenwurzel}
\end{table}
