{\let\clearpage\relax\chapter{Alchemistische Gebräue}}
\label{chap:alchemistische_gebraeue}
In diesem Kapitel sind alle herstellbare Tränke, Öle und Bomben inkl. Wirkung und Rezept aufgelistet. Tränke, Öle und Bomben sind in die Kategorien leicht, mittel und schwer eingeteilt. Für leichte Gebräue wird der Vorteil \textit{Alchemie I}, für mittlere \textit{Alchemie II} und für schwere \textit{Alchemie III} benötigt. Je nach Vorteilsstufe dürfen Rezepte um eine bestimmte Anzahl Ingredienzen erweitert werden (siehe Kapitel \ref{chap:vorteile}).

\section{Leichte Tränke}
Leichte Tränke benötigen \textit{starken Alkohol} als Grundstoff. Tränke mit der Spezialeigenschaft \textit{Dickflüssig} dürfen auch auf Gegenstände aufgetragen werden.

\begin{longtable}{|p{3cm}|p{2,5cm}|p{8cm}|p{1,5cm}|}
\hline
\textbf{Trank} / Wert & \textbf{Rezept} & \textbf{Effekt} & \textbf{Dauer} \\ \hline

\textbf{Bindekraut} \newline 80Kr & Ä/Q/K & +50\% Resistenz gegenüber Gift und Säure. Benötigt die \textbf{Spezialeigenschaft} \textit{Giftneutralisierend}. & 8h \\ \hline

\textbf{Frauentränen} \newline 100Kr & R/Ä/Q & Befreit den Körper schlagartig von negativen Effekten des Alkoholkonsums & sofort \\ \hline

\textbf{Katze} \newline 126Kr & \brcell{R/Q/Q} & Katze gewährt Sicht bei völliger Dunkelheit. & 8h \\ \hline

\textbf{Kuss} \newline 130Kr & \brcell{V/V/K} & Erhöht den Widerstand gegen Bluten und stoppt jede akute Blutung. & 8h \\ \hline

\textbf{Leichter Genesungstrank} \newline 120Kr & Ä/R/V & Heilt Krankheiten und stellt +1W6 LP wieder her. Benötigt \textbf{Spezialeigenschaft} \textit{Gesund}. & sofort \\ \hline

\textbf{Leichtes Schlafelixier} \newline 80Kr & V/Q/K & Lässt jemanden einschlafen. Wirkt erst nach 30min. Braucht \textbf{Spezialeigenschaft} \textit{Schläfrig}. & 8h \\ \hline

\textbf{Parfüm} \newline 120Kr & V/Ä/Q & Geschenk für Frauen. Es sollten wohlriechende Zutaten verwendet werden. & n.a. \\ \hline

\textbf{Schwarzpulver} \newline 40Kr & \brcell{Kaliumnitrat \\ Schwefel \\ Kohle} & \textbf{Spezialeigenschaft} \textit{Explosiv}. Dient als Grundstoff für Bomben. & - \\ \hline

\textbf{Waldkauz} \newline 88Kr & V/Ä/Ä & Erhöht deutlich die Ausdauer (+1 Erleichterung auf alles Körperliche). Benötigt die \textbf{Spezialeigenschaft} \textit{Aufputschmittel}. & 8h \\ \hline

\textbf{Weide} \newline 80Kr & Ä/Ä/Q & Weide verbessert die Koordination und Schadensresistenz. -2TP pro Angriff. & 8h \\ \hline

\textbf{Weißer Honig} \newline 60Kr & V/R/Ä & Reduziert die Giftigkeit auf Null und bricht alle Effekte anderer Tränke ab. Benötigt \textbf{Spezialeigenschaft} \textit{Giftneutralisierend}. & sofort \\ \hline

\textbf{Weiße Möwe} \newline 200Kr & V/V/R & Qualitativ hochwertiger Alkohol. Alchemistische Basis für schwere Tränke. \textbf{Spezialeigenschaft:} \textit{Halluzinogen} & 8h \\ \hline

\caption{Leichte Tränke}
\label{tab:Leichte_traenke}
\end{longtable}

\section{Mittlere Tränke}
Mittlere Tränke benötigen \textit{hochwertigen Alkohol} als Grundstoff. Tränke mit der Spezialeigenschaft \textit{Dickflüssig} dürfen auch auf Gegenstände aufgetragen werden.

\begin{longtable}{|p{3cm}|p{2,5cm}|p{8cm}|p{1,5cm}|}
\hline
\textbf{Trank} / Wert & \textbf{Rezept} & \textbf{Effekt} & \textbf{Dauer} \\ \hline

\textbf{Absud Raffards des Weissen} \newline 300Kr & V/R/H/H & Heilt Krankheiten und stellt 2+1W8 LP wieder her. Benötigt \textbf{Spezialeigenschaft} \textit{Gesund} 2 mal. & sofort \\ \hline

\textbf{Betäubungs- elixier} \newline 190Kr & K/K/V/R & Betäubendes Gift, das sofort wirkt. Benötigt \textbf{Spezialeigenschaft} \textit{Betäubungsgift} 2 mal. & 1h \\ \hline

\textbf{Bindekrautlack} \newline 90Kr & Ä/Q/Q/K & Eine dickflüssige Substanz mit der Gegenstände lackiert werden können. Der Lack ist Gift- und Säureresistent. Benötigt \textbf{Spezialeigenschaften} \textit{Giftneutralisierend} 1 mal und \textit{Dickflüssig} 2 mal. & $\infty$ \\ \hline

\textbf{Giftelixier} \newline 220Kr & V/R/R/H & Tödliches Gift, wenn alles getrunken wird. Sonst Bewusstlosigkeit. & sofort \\ \hline

\textbf{Goldener Pirol} \newline 260Kr & V/V/Ä/Ä & Immunität gegenüber Gift und Säure. Benötigt die \textbf{Spezialeigenschaft} \textit{Giftneutralisierend} 2 mal. & 8h \\ \hline

\textbf{Mariborwald} \newline 148Kr & R/R/Ä/Q & Erhöht deutlich die Ausdauer (+2 Erleichterung auf alles Körperliche). Benötigt die \textbf{Spezialeigenschaft} \textit{Aufputschmittel}. & 8h \\ \hline

\textbf{Schlafelixier} \newline 160Kr & V/Q/K/H & Lässt jemanden einschlafen. Wirkt nach wenigen Minuten. Braucht \textbf{Spezialeigenschaft} \textit{Schläfrig} 2 mal. & 8h \\ \hline

\textbf{Schneesturm} \newline 200Kr & V/V/R/R & Steigert die Reflexe und Reaktionszeit, steigert die Chancen des Parierens und Attacken zu entgehen. Ausweicheffektivität gesteigert um 50\%; Pariereffektivität gesteigert um 50\%. & 20min \\ \hline

\textbf{Schwalbe} \newline 300Kr & V/R/R/Ä & Stellt jede Stunde 2 LP wieder her. (Insgesamt +16LP) & 8h \\ \hline

\caption{Mittlere Tränke}
\label{tab:mittlere_traenke}
\end{longtable}

\newpage

\section{Schwere Tränke}
Schwere Tränke benötigen \textit{qualitativ hochwertigen Alkohol} als Grundstoff. Tränke mit der Spezialeigenschaft \textit{Dickflüssig} dürfen auch auf Gegenstände aufgetragen werden.

\begin{longtable}{|p{3cm}|p{2,5cm}|p{8cm}|p{1,5cm}|}
\hline
\textbf{Trank} / Wert & \textbf{Rezept} & \textbf{Effekt} & \textbf{Dauer} \\ \hline

\textbf{De Vries' Extrakt} \newline 250Kr & R/Ä/Q/Q/H & Ermöglicht es, lebende Kreaturen aufzuspüren, auch wenn sie unsichtbar sind oder durch Wände hindurch. & 1h \\ \hline

\textbf{Donner} \newline 300Kr & V/R/H/K/K & Erhöht den Schaden um 2W6. Erschwert \textit{Blocken} und \textit{Parieren} um 4 und Ausweichen um 2. & 8h \\ \hline 

\textbf{Fisstech} \newline 350Kr & Q/H/H/H/K & Eine starke Droge. Auf den Konsum erfolgt sofortige Bewusstlosigkeit. Benötigt \textbf{Spezialeigenschaft} \textit{Gift}/\textit{Halluzinogen} 3 mal. & 8h \\ \hline

\textbf{Petris Zaubertrank} \newline 270Kr & R/R/Q/H/K & Erhöht die Magie-Intensität. +4 aus alle Proben, die mit Magie zu tun haben. & 8h \\ \hline

\textbf{Schwarzes Blut} \newline 300Kr & V/V/V/R/Ä & Tödliches Gift für Menschen und manche Monster. Benötigt \textbf{Spezialeigenschaft} \textit{Gift} 3 mal. & sofort \\ \hline

\textbf{Vollmond} \newline 400Kr & Q/H/H/K/K & Maximale LP werden verdoppelt (+100\%LP). Benötigt \textbf{Spezialeigenschaft} \textit{Gesund} 2 mal und \textit{Aufputschmittel} 2 mal. & 8h \\ \hline

\textbf{Wolf} \newline 280Kr & V/V/H/H/K & Erhöht die Chance auf kritische Treffer (betrifft auch Runen) um 50\%. & 8h \\ \hline

\textbf{Wolverin} \newline 300Kr & Ä/Ä/Q/H/K & Erhöht den Schaden um +2W6, wenn der Held weniger als 50\% LP hat. & 8h \\ \hline

\textbf{Würger} \newline 360Kr & R/Q/H/K/K & Verursacht unerträgliche Schmerzen. -5TP pro halbe Stunde (insgesamt -25LP). Benötigt \textbf{Spezialeigenschaft} \textit{Gift} oder \textit{Betäubungsgift} 3 mal. & 2,5h \\ \hline

\caption{Schwere Tränke}
\label{tab:schwere_traenke}
\end{longtable}


\section{Leichte Öle}
Leichte Öle benötigen Fett als Grundstoff. Fett darf durch eine Ingredienz mit der Spezialeigenschaft \textit{Dickflüssig} ersetzt werden. Leichte Öle sind im Schnitt etwa 100Kr Wert.

\begin{longtable}{|p{4cm}|p{2,5cm}|p{9cm}|}
\hline
\textbf{Öl} & \textbf{Rezept} & \textbf{Effekt} \\ \hline

\textbf{Nekrophagenöl} & V/V/Ä & Eine Klinge, die mit dieser Substanz überzogen ist, fügt Nekrophagen einen erhöhten Schaden von 100\% hinzu. \\ \hline

\textbf{Geisteröl} & V/R/Ä & Erhöht Geistern zugefügten Schaden um 100\%. Geisteröl ist wirksam auf dem Stahl- und Silberschwert.  \\ \hline

\caption{Leichte Öle}
\label{tab:leichte_oele}
\end{longtable}


\section{Mittlere Öle}
Mittlere Öle benötigen Fett als Grundstoff. Fett darf durch zwei Ingredienzen mit der Spezialeigenschaft \textit{Dickflüssig} ersetzt werden. Mittlere Öle sind im Schnitt etwa 130Kr Wert.

\begin{longtable}{|p{4cm}|p{2,5cm}|p{9cm}|}
\hline
\textbf{Öl} & \textbf{Rezept} & \textbf{Effekt} \\ \hline

\textbf{Insektoidenöl} & V/R/Ä/Q & Eine Silberklinge, die mit dieser Substanz überzogen ist, fügt Kreaturen mit insektenartiger Physiologie 100\% mehr Schaden hinzu. \\ \hline

\textbf{Ornithosaurusöl} & V/R/Q/H & Eine Klinge versehen mit dieser Substanz fügt Ornothosauriern erhöhten Schaden von 100\% hinzu.  \\ \hline

\caption{Mittlere Öle}
\label{tab:mittlere_oele}
\end{longtable}


\section{Schwere Öle}
Schwere Öle benötigen Fett als Grundstoff. Fett darf durch zwei Ingredienzen mit der Spezialeigenschaft \textit{Dickflüssig} ersetzt werden. Schwere Öle sind im Schnitt etwa 170Kr Wert.

\begin{longtable}{|p{4cm}|p{2,5cm}|p{9cm}|}
\hline
\textbf{Öl} & \textbf{Rezept} & \textbf{Effekt} \\ \hline

\textbf{Argentia} & R/Q/Q/H/H & Erhöht den mit einem Silberschwert zugefügten Schaden um 50\%. Verringert die Wirksamkeit eines Stahlschwertes um 50\%. \\ \hline

\textbf{Braunöl} & R/Ä/Q/K/K/ & Erhöht den Blutverlust bei einer Verwundung und führt letztlich in den Tod. Kreaturen, die kein Kreislaufsystem haben, sind immun gegen Braunöl. Wunde macht 5LP Schaden pro Runde. \\ \hline

\textbf{Crinfridöl} & V/Ä/H/K/K & Eine Klinge, auf der dieses Öl verwendet wird, verursacht beim Feind lähmende Schmerzen. Crinfridöl ist wirkungslos gegenüber Gegnern, die immun gegen Schmerz sind. Benötigt \textbf{Spezialeigenschaft} \textit{Betäubungsgift} 2 mal und \textit{Gift}. \\ \hline

\textbf{Henkersgiftserum} & V/Ä/R/H/K & Dringt durch offene Wunden in den Blutkreislauf des Gegners ein und vergiftet ihn. Ist nutzlos bei Kreaturen, deren Physiologie sich vom Menschen unterscheidet. Benötigt \textbf{Spezialeigenschaft} \textit{Gift} 3 mal. \\ \hline

\textbf{Vampiröl} & V/Q/Q/H/K & Eine Klinge, die mit dieser Substanz überzogen ist, verursacht einen erhöhten Schaden von 100\% gegenüber Vampiren.  \\ \hline

\caption{Schwere Öle}
\label{tab:schwere_oele}
\end{longtable}



\section{Leichte Bomben}
Leichte Bomben benötigen \textit{hochwertiges Pulver} als Grundstoff. Zusätzlich benötigen sie eine Zündvorrichtung und ein Gefäß. (siehe Kapitel \ref{chap:herstellung_von_bomben}). Leichte Bomben sind im Schnitt etwa 200Kr Wert.

\begin{longtable}{|p{4cm}|p{2,5cm}|p{9cm}|}
\hline
\textbf{Bombe} / Radius & \textbf{Rezept} & \textbf{Effekt} \\ \hline

\textbf{König und Königin} \newline 2m & R/Ä/Ä & Explosion setzt ein Gas frei, das Gegner in Angst und Schrecken versetzt. Benötigt \textbf{Spezialeigenschaft} \textit{Halluzinogen}. \\ \hline

\textbf{Sprengkapsel} \newline 0,5m & R/Q/K & Explodiert in einem \textbf{0,5m Radius} und verursacht Schaden. Die Explosion besitzt genug Kraft um Türen oder Schlösser aufzubrechen. Benötigt \textbf{Spezialeigenschaft} \textit{Explosiv}. Verursacht \textbf{15TP Schaden} an Gegnern im Explosionsradius. Der Schaden wird durch RS verringert. \\ \hline

\textbf{Stinkbombe} \newline 2m & R/R/Q & So starker Gestank, der alles und jeden vertreibt. Endprodukt benötigt Geruch-Eigenschaft \textit{Übelriechend (5+)}. Je höher der Konzentrationswert, desto übler ist der Gestank. Benötigt keine Zündvorrichtung. \\ \hline

\caption{Leichte Bomben}
\label{tab:leichte_bomben}
\end{longtable}


\section{Mittlere Bomben}
Mittlere Bomben benötigen \textit{hochwertiges Pulver} als Grundstoff. Zusätzlich benötigen sie eine Zündvorrichtung und ein Gefäß. (siehe Kapitel \ref{chap:herstellung_von_bomben}). Mittlere Bomben sind im Schnitt etwa 250Kr Wert.

\begin{longtable}{|p{4cm}|p{2,5cm}|p{9cm}|}
\hline
\textbf{Bombe} / Radius & \textbf{Rezept} & \textbf{Effekt} \\ \hline

\textbf{Samum} \newline 2,5m & Ä/K/R/R & Setzt eine lähmende Druckwelle frei. Benötigt \textbf{Spezialeigenschaft} \textit{Betäubungsgift}. \\ \hline

\textbf{Serrikanische Sonne} \newline 10m & Q/Ä/K/R & Blendet in der Nähe befindliche Gegner. Benötigt \textbf{Spezialeigenschaft} \textit{Leuchtend}. \\ \hline

\textbf{Teufelsbovist} \newline 2,5m & Ä/Ä/H/R & Vergiftet durch Giftwolken in der Nähe befindliche Feinde, wirkungslos bei Gegnern, die immun gegen Gift sind. Benötigt \textbf{Spezialeigenschaft} \textit{Gift}. Benötigt keine Zündvorrichtung. \\ \hline

\textbf{Wyverntraum} \newline 2m & R/Q/K/H & Schwache Version von \textit{Drachentraum}. Setzt leicht entzündbare Gase frei. Benötigt keine Zündvorrichtung. Benötigt \textbf{Spezialeigenschaft} \textit{Explosiv} 2 mal. Mit Zündvorrichtung wird das Gas beim Aufprall entzündet, was eine Explosion mit 2m Radius verursacht. Schaden für alle Gegner in der Nähe. Abfallender Schaden, je weiter weg man steht: \textbf{0m = 30TP}, \textbf{0,5m = 20TP}, \textbf{1m = 10TP}, \textbf{2m = 1TP}. Der Schaden wird durch RS verringert. \\ \hline

\caption{Mittlere Bomben}
\label{tab:mittlere_bomben}
\end{longtable}


\section{Schwere Bomben}
Schwere Bomben benötigen \textit{qualitativ hochwertiges Pulver} als Grundstoff. Zusätzlich benötigen sie eine Zündvorrichtung und ein Gefäß. (siehe Kapitel \ref{chap:herstellung_von_bomben}). Schwere Bomben sind im Schnitt etwa 300Kr Wert.

\begin{longtable}{|p{4cm}|p{2,5cm}|p{9cm}|}
\hline
\textbf{Bombe} / Radius & \textbf{Rezept} & \textbf{Effekt} \\ \hline

\textbf{Drachentraum} \newline 3m & R/Q/K/K/H & Setzt leicht entzündbare Gase frei. Benötigt keine Zündvorrichtung. Benötigt \textbf{Spezialeigenschaft} \textit{Explosiv} 3 mal. Mit Zündvorrichtung wird das Gas beim Aufprall entzündet, was eine enorme Explosion mit \textbf{3m Radius} verursacht. Schaden für alle Gegner in der Nähe. Abfallender Schaden, je weiter weg man steht: \textbf{0m = 50TP}, \textbf{1m = 40TP}, \textbf{2m = 15TP}, \textbf{3m = 1TP}. Der Schaden wird durch RS verringert. \\ \hline

\caption{Schwere Bomben}
\label{tab:schwere_bomben}
\end{longtable}