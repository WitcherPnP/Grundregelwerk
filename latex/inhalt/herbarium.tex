{\let\clearpage\relax\chapter{Herbarium}}
Hier sind alle Ingredienzen aufgeführt, die für die Herstellung von Gebräue wie Tränke, Öle oder Bomben benutzt werden können. Ingredienzen können eine \textit{Pflückprobe} und \textit{Utensilien} definieren. Die \textit{Pflückprobe} gibt an wie eine Probe auf \textit{Alchemie} erfüllt werden muss, damit die Ingredienz erfolgreich gepflückt wird. \textit{Utensilien} geben an was zum Pflücken benötigt wird. Wenn nicht anders angegeben, muss die \textit{Pflückprobe} ausgeführt werden, wenn kein angegebenes \textit{Utensil} verwendet wird. Wenn die \textit{Pflückprobe} misslingt wird die Ingredienz zerstört (wenn nicht anders angegeben).

\section{Pflanzen}

\subsection{Alraunenwurzel}
Die Alraunenwurzel oder auch Mandragora ist eine giftige Pflanze, enthält aber magische Eigenschaften. Druiden verwenden diese Wurzel häufig. Die gelben Beeren und die Blätter sind essbar. Die Wurzeln sondern bei Verletzung ein Toxin mit stark halluzinogener Wirkung ab. Die Wurzeln haben oft das Aussehen eines Menschen.

Mit Handschuhen und/oder mit Werkzeug zum Graben (z.B. ein Spaten) wird die \textit{Pflückprobe} um je 1 QS einfacher.

Bei Fehlschlag 1W4 Tage schwere Halluzinationen.

\begin{table}[h] 
\begin{center} 
\begin{tabular}{|l|l|p{1cm}|l|l|} 
  	\cline{1-2} \cline{4-5} 
  	\textbf{Hauptwirkstoff} & Quebrith && \textbf{Pflückprobe} & QS III \\ \cline{1-2} \cline{4-5} 
  	\textbf{Sekundärwirkstoff} & Nigredo && \textbf{Menge} & 1 pro Pflanze \\ \cline{1-2} \cline{4-5} 
  	\textbf{Farbe} & braun (2) && \textbf{Vorkommen} & \brcell{Wald (häufig) \\ Höhle (mittel)} \\ \cline{1-2} \cline{4-5} 
  	\textbf{Geruch} & \brcell{flieder- \\ stachelbeere (2)} && \textbf{Region} & \brcell{gemäßigte \\ Temperaturen} \\ \cline{1-2} \cline{4-5} 
  	\textbf{Geschmack} & mehlig (2) && \textbf{Wert} & 15Kr \\ \cline{1-2} \cline{4-5} 
  	\textbf{Spezialeigenschaften} & Halluzinogen && \textbf{Utensilien} & (siehe Beschreibung) \\ \cline{1-2} \cline{4-5} 
\end{tabular} 
\end{center} 
\caption{Alraunenwurzel} 
\label{tab:alraunenwurzel} 
\end{table}


\subsection{Balissafrucht}
Balissafrucht ist eine essbare Frucht die am Balissastrauch wächst. Sie zeichnet sich durch eine zarte magische Resonanz aus. 

\begin{table}[h]
\begin{center}
\begin{tabular}{|l|l|p{1cm}|l|l|}
	\cline{1-2} \cline{4-5}
	\textbf{Hauptwirkstoff} & Quebrith && \textbf{Pflückprobe} & - \\ \cline{1-2} \cline{4-5}
	\textbf{Sekundärwirkstoff} & Rubedo && \textbf{Menge} & 2+1W6 pro Strauch \\ \cline{1-2} \cline{4-5}
	\textbf{Farbe} & violett (2) && \textbf{Vorkommen} & Feld (mittel), Wald (häufig) \\ \cline{1-2} \cline{4-5}
	\textbf{Geruch} & fruchtig (1) && \textbf{Region} & überall \\ \cline{1-2} \cline{4-5}
	\textbf{Geschmack} & süß (2) && \textbf{Wert} & 12Kr \\ \cline{1-2} \cline{4-5}
	\textbf{Spezialeigenschaften} & - && \textbf{Utensilien} & - \\ \cline{1-2} \cline{4-5}
\end{tabular}
\end{center}
\caption{Balissafrucht}
\label{tab:balissafrucht}
\end{table}


\subsection{Berberrohrfrucht}
Die Berberrohrfrucht schmeckt sauer und ist saftig. Sie wächst an einem Strauch mit Dornen. Der Strauch wird bis zu 1,5m hoch. 

Misslingt die \textit{Pflückprobe} (aber \textit{Alchemie}-Probe gelingt) bekommt der Held 1 Schaden pro QS die er zu niedrig ist. Misslingt die \textit{Alchemie}-Probe fällt der Held in den Strauch und bekommt zusätzlich 1 Schaden pro Wert den er unterschritten hat. 

Bsp.: Misslingt die Probe um 2 bekommt er 2 Schaden und zusätzlich 3 Schaden wegen der nicht erreichten QS IV.

\newpage

\begin{table}[h]
\begin{center}
\begin{tabular}{|l|l|p{1cm}|l|l|}
	\cline{1-2} \cline{4-5}
	\textbf{Hauptwirkstoff} & Äther && \textbf{Pflückprobe} & QS IV \\ \cline{1-2} \cline{4-5}
	\textbf{Sekundärwirkstoff} & Albedo && \textbf{Menge} & 1W12 pro Strauch \\ \cline{1-2} \cline{4-5}
	\textbf{Farbe} & grün (2) && \textbf{Vorkommen} & Feld (häufig) \\ \cline{1-2} \cline{4-5}
	\textbf{Geruch} & neutral && \textbf{Region} & überall \\ \cline{1-2} \cline{4-5}
	\textbf{Geschmack} & sauer (2) && \textbf{Wert} & 21Kr \\ \cline{1-2} \cline{4-5}
	\textbf{Spezialeigenschaften} & - && \textbf{Utensilien} & Handschuh \\ \cline{1-2} \cline{4-5}
\end{tabular}
\end{center}
\caption{Berberrohrfrucht}
\label{tab:berberrohrfrucht}
\end{table}


\subsection{Büschelkrautblüten}
Mit den auffällig roten Blüten ist Büschelkraut eine weitverbreitete Pflanze, die auf Wiesen und feuchten Böden wächst. Die Blütenblätter enthalten ein leichtes Halluzinogen, sind giftig und werden für Fisstech verwendet. 

\begin{table}[h] 
\begin{center} 
\begin{tabular}{|l|l|p{1cm}|l|l|} 
  	\cline{1-2} \cline{4-5} 
  	\textbf{Hauptwirkstoff} & Hydragenum && \textbf{Pflückprobe} & - \\ \cline{1-2} \cline{4-5} 
  	\textbf{Sekundärwirkstoff} & Rubedo && \textbf{Menge} & 1 pro Pflanze \\ \cline{1-2} \cline{4-5} 
  	\textbf{Farbe} & rot (3) && \textbf{Vorkommen} & \brcell{Sumpf (selten)\\Feld (selten)} \\ \cline{1-2} \cline{4-5} 
  	\textbf{Geruch} & büschelkraut (2) && \textbf{Region} & überall \\ \cline{1-2} \cline{4-5} 
  	\textbf{Geschmack} & bitter (3) && \textbf{Wert} & 15Kr \\ \cline{1-2} \cline{4-5} 
  	\textbf{Spezialeigenschaften} & Gift, Halluzinogen && \textbf{Utensilien} & - \\ \cline{1-2} \cline{4-5} 
\end{tabular} 
\end{center} 
\caption{Büschelkrautblüten} 
\label{tab:bueschelkrautblueten} 
\end{table}


\subsection{Echinopswurzel}


\subsection{Eisenkraut}
Eisenkraut ist eine weit verbreitete, blühende Pflanze. 

\begin{table}[h] 
\begin{center} 
\begin{tabular}{|l|l|p{1cm}|l|l|} 
  	\cline{1-2} \cline{4-5} 
  	\textbf{Hauptwirkstoff} & Quebrith && \textbf{Pflückprobe} & - \\ \cline{1-2} \cline{4-5} 
  	\textbf{Sekundärwirkstoff} & Albedo && \textbf{Menge} & 1 pro Pflanze \\ \cline{1-2} \cline{4-5} 
  	\textbf{Farbe} & rosa (1) && \textbf{Vorkommen} & Ödland (häufig) \\ \cline{1-2} \cline{4-5} 
  	\textbf{Geruch} & eisenkraut (1) && \textbf{Region} & überall \\ \cline{1-2} \cline{4-5} 
  	\textbf{Geschmack} & bitter (1) && \textbf{Wert} & 15Kr \\ \cline{1-2} \cline{4-5} 
  	\textbf{Spezialeigenschaften} & - && \textbf{Utensilien} & - \\ \cline{1-2} \cline{4-5} 
\end{tabular} 
\end{center} 
\caption{Eisenkraut} 
\label{tab:eisenkraut} 
\end{table}


\subsection{Feainnwedd}
Feainnwedd ist eine Blume mit wundervollem Duft, die einer Legende nach nur an Orten wächst, an dem \textit{Älteren-Blut} vergossen wurde sowie im \textit{Tal der Blumen}. Die Elfen erzählen, dass Feainnwedd erst zu wachsen begann, nachdem \textit{Lara Dorren} gestorben war.

In der alten Sprache bedeutet Faen "`Sonne"' und wedd "`Kind"', was hier möglicherweise "`Kind der Sonne"' oder "`Sonnenkind"' heißt. 

\begin{table}[h] 
\begin{center} 
\begin{tabular}{|l|l|p{1cm}|l|l|} 
  	\cline{1-2} \cline{4-5} 
  	\textbf{Hauptwirkstoff} & Karmin && \textbf{Pflückprobe} & - \\ \cline{1-2} \cline{4-5} 
  	\textbf{Sekundärwirkstoff} & Rubedo && \textbf{Menge} & 1 pro Pflanze \\ \cline{1-2} \cline{4-5} 
  	\textbf{Farbe} & lila (2) && \textbf{Vorkommen} & \brcell{siehe Beschreibung \\ (selten)} \\ \cline{1-2} \cline{4-5} 
  	\textbf{Geruch} & Feainnwedd (2) && \textbf{Region} & siehe Beschreibung \\ \cline{1-2} \cline{4-5} 
  	\textbf{Geschmack} & neutral && \textbf{Wert} & 50Kr \\ \cline{1-2} \cline{4-5} 
  	\textbf{Spezialeigenschaften} & - && \textbf{Utensilien} & - \\ \cline{1-2} \cline{4-5} 
\end{tabular} 
\end{center} 
\caption{Feainnwedd} 
\label{tab:feainnwedd} 
\end{table}


\subsection{Geisblatt}
Geisblatt nennen sich die Blätter einer widerstandsfähigen Pflanze die vorzugsweise auf Ödland wächst als auch auf Wiesen und Feldern. 

\begin{table}[h] 
\begin{center} 
\begin{tabular}{|l|l|p{1cm}|l|l|} 
  	\cline{1-2} \cline{4-5} 
  	\textbf{Hauptwirkstoff} & Quebrith && \textbf{Pflückprobe} & - \\ \cline{1-2} \cline{4-5} 
  	\textbf{Sekundärwirkstoff} & Albedo && \textbf{Menge} & 1 pro Pflanze \\ \cline{1-2} \cline{4-5} 
  	\textbf{Farbe} & grün (2) && \textbf{Vorkommen} & \brcell{Ödland (mittel) \\ Feld (mittel)} \\ \cline{1-2} \cline{4-5} 
  	\textbf{Geruch} & neutral && \textbf{Region} & \brcell{überall \\ Brokilon (häufig)} \\ \cline{1-2} \cline{4-5} 
  	\textbf{Geschmack} & neutral && \textbf{Wert} & 21Kr \\ \cline{1-2} \cline{4-5} 
  	\textbf{Spezialeigenschaften} & - && \textbf{Utensilien} & - \\ \cline{1-2} \cline{4-5} 
\end{tabular} 
\end{center} 
\caption{Geisblatt} 
\label{tab:geisblatt} 
\end{table}


\subsection{Grünschimmel}
Grünschimmel wächst an den Wänden dunkler und feuchter Orte, wie z. B. in natürlichen Höhlen oder auch in Abwasserkanälen. Es sieht ein auf den ersten Blick aus wie Moos ist aber haarig und nicht so weich.

Zum Pflücken benötigt man etwas zum abscharben, z.B. ein Messer.

\begin{table}[h] 
\begin{center} 
\begin{tabular}{|l|l|p{1cm}|l|l|} 
  	\cline{1-2} \cline{4-5} 
  	\textbf{Hauptwirkstoff} & Rebis && \textbf{Pflückprobe} & QS 3 \\ \cline{1-2} \cline{4-5} 
  	\textbf{Sekundärwirkstoff} & Rubedo && \textbf{Menge} & 2W4 pro Ort \\ \cline{1-2} \cline{4-5} 
  	\textbf{Farbe} & grün (2) && \textbf{Vorkommen} & \brcell{Höhle (häufig) \\ Sumpf (selten)} \\ \cline{1-2} \cline{4-5} 
  	\textbf{Geruch} & übelriechend (3) && \textbf{Region} & überall \\ \cline{1-2} \cline{4-5} 
  	\textbf{Geschmack} & schimmel (2) && \textbf{Wert} & 15Kr \\ \cline{1-2} \cline{4-5} 
  	\textbf{Spezialeigenschaften} & - && \textbf{Utensilien} & Messer \\ \cline{1-2} \cline{4-5} 
\end{tabular} 
\end{center} 
\caption{Grünschimmel} 
\label{tab:gruenschimmel} 
\end{table}


\subsection{Hanfasern}
Hanfasern sehen aus wie karge und blasse Grasbüschel und werden deshalb häufig für Unkraut gehalten. Sie sind eine weit verbreitete Pflanze auf Wiesen und Feldern mit gutem Nährboden. Hanfasern sind giftig. Wenn man sie mit einer Sichel pflückt, erhöht sich der Ertrag um 2. Häufig werden Bauernhof Tiere krank, weil sie sie essen und es nicht vom Bauern bemerkt wird.

\begin{table}[h] 
\begin{center} 
\begin{tabular}{|l|l|p{1cm}|l|l|} 
  	\cline{1-2} \cline{4-5} 
  	\textbf{Hauptwirkstoff} & Rebis && \textbf{Pflückprobe} & - \\ \cline{1-2} \cline{4-5} 
  	\textbf{Sekundärwirkstoff} & Nigredo && \textbf{Menge} & 1W4 \\ \cline{1-2} \cline{4-5} 
  	\textbf{Farbe} & grün (1) && \textbf{Vorkommen} & Feld (häufig) \\ \cline{1-2} \cline{4-5} 
  	\textbf{Geruch} & neutral && \textbf{Region} & überall \\ \cline{1-2} \cline{4-5} 
  	\textbf{Geschmack} & neutral && \textbf{Wert} & 21Kr \\ \cline{1-2} \cline{4-5} 
  	\textbf{Spezialeigenschaften} & Gift && \textbf{Utensilien} & Sichel \\ \cline{1-2} \cline{4-5} 
\end{tabular} 
\end{center} 
\caption{Hanfasern} 
\label{tab:hanfasern} 
\end{table}


\subsection{Hopfendolden}
Hopfendolden sind eine gebräuchliche alchemistische Zutat für Tränke. Sie wird auch zum Bier brauen verwendet.

\begin{table}[h] 
\begin{center} 
\begin{tabular}{|l|l|p{1cm}|l|l|} 
  	\cline{1-2} \cline{4-5} 
  	\textbf{Hauptwirkstoff} & Vitriol && \textbf{Pflückprobe} & - \\ \cline{1-2} \cline{4-5} 
  	\textbf{Sekundärwirkstoff} & Nigredo && \textbf{Menge} & 1 pro Pflanze \\ \cline{1-2} \cline{4-5} 
  	\textbf{Farbe} & grün (1) && \textbf{Vorkommen} & Feld (häufig) \\ \cline{1-2} \cline{4-5} 
  	\textbf{Geruch} & hopfen (2) && \textbf{Region} & überall \\ \cline{1-2} \cline{4-5} 
  	\textbf{Geschmack} & herb (2) && \textbf{Wert} & 21Kr \\ \cline{1-2} \cline{4-5} 
  	\textbf{Spezialeigenschaften} & - && \textbf{Utensilien} & - \\ \cline{1-2} \cline{4-5} 
\end{tabular} 
\end{center} 
\caption{Hopfendolden} 
\label{tab:hopfendolden} 
\end{table}


\subsection{Hundspetersilie}
Hundspetersilie ist eine Sumpfpflanze mit fleischigen, großen, grünen Blättern. Die Pflanze gedeiht üppig in sumpfigen Gebieten, wie zum Beispiel im Friedhofssumpf von Alt-Vizima und in den Sümpfen von Vizima. Die Blätter dieser Pflanze bestehen aus Fasern, mit denen sich ein magisches Hemd weben lässt. 

Bei der alchemistischen Zubereitung erhält man eine dickflüssige Substanz, die häufig für Öle verwendet wird. Wird außerdem gerne verwendet um Soßen dickflüssiger zu machen und um ihr einen frischen Kräutergeschmack zu verleihen.

\begin{table}[h] 
\begin{center} 
\begin{tabular}{|l|l|p{1cm}|l|l|} 
  	\cline{1-2} \cline{4-5} 
  	\textbf{Hauptwirkstoff} & Quebrith && \textbf{Pflückprobe} & - \\ \cline{1-2} \cline{4-5} 
  	\textbf{Sekundärwirkstoff} & Rubedo && \textbf{Menge} & 2+1W4 pro Pflanze \\ \cline{1-2} \cline{4-5} 
  	\textbf{Farbe} & grün (2) && \textbf{Vorkommen} & Sumpf (häufig) \\ \cline{1-2} \cline{4-5} 
  	\textbf{Geruch} & petersilie (2) && \textbf{Region} & überall \\ \cline{1-2} \cline{4-5} 
  	\textbf{Geschmack} & petersilie (2) && \textbf{Wert} & 15Kr \\ \cline{1-2} \cline{4-5} 
  	\textbf{Spezialeigenschaften} & Dickflüssig && \textbf{Utensilien} & - \\ \cline{1-2} \cline{4-5} 
\end{tabular} 
\end{center} 
\caption{Hundspetersilie} 
\label{tab:hundspetersilie} 
\end{table}

\subsection{Ignatia Blüten}
Ignatia ist ein Busch, der recht häufig in \textit{Temerien} und im \textit{Kaer Morhen Tal} vorkommt. Die Ignatia Blüten finden häufig Verwendung in der Alchemie.

Ignatia wird auch Mondrose genannt. In Legenden heißt es, die Pflanze wächst dort, wo eine Sternschnuppe auf die Erde gefallen ist.

\begin{table}[h] 
\begin{center} 
\begin{tabular}{|l|l|p{1cm}|l|l|} 
  	\cline{1-2} \cline{4-5} 
  	\textbf{Hauptwirkstoff} & Äther && \textbf{Pflückprobe} & - \\ \cline{1-2} \cline{4-5} 
  	\textbf{Sekundärwirkstoff} & Nigredo && \textbf{Menge} & 2+1W4 pro Busch \\ \cline{1-2} \cline{4-5} 
  	\textbf{Farbe} & rosa (2) && \textbf{Vorkommen} & \brcell{Feld (häufig) \\ Wald (häufig)} \\ \cline{1-2} \cline{4-5} 
  	\textbf{Geruch} & ignatia (2) && \textbf{Region} & \brcell{Temerien \\ Kear Morgen Tal} \\ \cline{1-2} \cline{4-5} 
  	\textbf{Geschmack} & neutral && \textbf{Wert} & 15Kr \\ \cline{1-2} \cline{4-5} 
  	\textbf{Spezialeigenschaften} & - && \textbf{Utensilien} & - \\ \cline{1-2} \cline{4-5} 
\end{tabular} 
\end{center} 
\caption{Ignatia Blüten} 
\label{tab:ignatia_blueten} 
\end{table}

\subsection{Krähenauge}
Die Wurzel eines zypressenartigen Busches, die häufig in der Alchemie verwendet wird. Um an die Knollenwurzel zu gelangen muss der gesamte Busch ausgegraben werden. Die Knolle ist so groß wie eine Zwiebel und hat die Form und Farbe einer Kartoffel.

\begin{table}[h] 
\begin{center} 
\begin{tabular}{|l|l|p{1cm}|l|l|} 
  	\cline{1-2} \cline{4-5} 
  	\textbf{Hauptwirkstoff} & Vitriol && \textbf{Pflückprobe} & QS IV \\ \cline{1-2} \cline{4-5} 
  	\textbf{Sekundärwirkstoff} & Negredo && \textbf{Menge} & 1 pro Busch \\ \cline{1-2} \cline{4-5} 
  	\textbf{Farbe} & braun (1) && \textbf{Vorkommen} & Feld (häufig) \\ \cline{1-2} \cline{4-5} 
  	\textbf{Geruch} & erdig (1) && \textbf{Region} & überall \\ \cline{1-2} \cline{4-5} 
  	\textbf{Geschmack} & bitter (2) && \textbf{Wert} & 15Kr \\ \cline{1-2} \cline{4-5} 
  	\textbf{Spezialeigenschaften} & - && \textbf{Utensilien} & Spaten, Schaufel \\ \cline{1-2} \cline{4-5} 
\end{tabular} 
\end{center} 
\caption{Krähenauge} 
\label{tab:kraehenauge} 
\end{table}

\subsection{Mistelzweig}
Mistelzweige sind heiligen Pflanzen der Druiden. Sie verwenden sie für Heilmittel gegen Krankheiten. Man findet sie als Opfergaben bei Altaren der Melitele. Es gibt nur wenige Orte an denen sie natürlich Wachsen und nur Druiden und manche Anhänger der Melitele kennen diese Orte.

\begin{table}[h] 
\begin{center} 
\begin{tabular}{|l|l|p{1cm}|l|l|} 
  	\cline{1-2} \cline{4-5} 
  	\textbf{Hauptwirkstoff} & Hydragenum && \textbf{Pflückprobe} & - \\ \cline{1-2} \cline{4-5} 
  	\textbf{Sekundärwirkstoff} & Nigredo && \textbf{Menge} & unbekannt \\ \cline{1-2} \cline{4-5} 
  	\textbf{Farbe} & grün (1) && \textbf{Vorkommen} & unbekannt (selten) \\ \cline{1-2} \cline{4-5} 
  	\textbf{Geruch} & tanne (1) && \textbf{Region} & \brcell{unbekannt \\ (siehe Beschreibung)} \\ \cline{1-2} \cline{4-5} 
  	\textbf{Geschmack} & neutral && \textbf{Wert} & 21Kr \\ \cline{1-2} \cline{4-5} 
  	\textbf{Spezialeigenschaften} & Gesund && \textbf{Utensilien} & - \\ \cline{1-2} \cline{4-5} 
\end{tabular} 
\end{center} 
\caption{Mistelzweig} 
\label{tab:mistelzweig} 
\end{table}

\subsection{Mutterkornsamen}
Mutterkornsamen ist die Saat einer rituellen Pflanze. Diese Pflanze, die als Unkraut bezeichnet wird, sieht aus wie ein kleiner Gerstenhalm. Aus der Blüte erhält man die Samen. Sie wird manchmal als Ersatz für Gerste verwendet, da sie fast die selben Eigenschaften hat.

\begin{table}[h] 
\begin{center} 
\begin{tabular}{|l|l|p{1cm}|l|l|} 
  	\cline{1-2} \cline{4-5} 
  	\textbf{Hauptwirkstoff} & Karmin && \textbf{Pflückprobe} & - \\ \cline{1-2} \cline{4-5} 
  	\textbf{Sekundärwirkstoff} & Rubedo && \textbf{Menge} & 1W4 pro Pflanze \\ \cline{1-2} \cline{4-5} 
  	\textbf{Farbe} & gelb-braun (2) && \textbf{Vorkommen} & Feld (häufig) \\ \cline{1-2} \cline{4-5} 
  	\textbf{Geruch} & getreide (2) && \textbf{Region} & überall \\ \cline{1-2} \cline{4-5} 
  	\textbf{Geschmack} & gerste (2) && \textbf{Wert} & 15Kr \\ \cline{1-2} \cline{4-5} 
  	\textbf{Spezialeigenschaften} & - && \textbf{Utensilien} & - \\ \cline{1-2} \cline{4-5} 
\end{tabular} 
\end{center} 
\caption{Mutterkornsamen} 
\label{tab:mutterkornsamen} 
\end{table}

\subsection{Nieswurzblüten}
Nieswurzblüten sind eine allgemeine und weit verbreitete Giftpflanze. Eine alte Frau aus dem Umland von Vizima erzählt, dass man aus Nieswurzblüten ein Tonikum gegen Schlaflosigkeit brauen kann.

Auch heute noch wird Nieswurz als Heilpflanze verwendet. 

\begin{table}[h] 
\begin{center} 
\begin{tabular}{|l|l|p{1cm}|l|l|} 
  	\cline{1-2} \cline{4-5} 
  	\textbf{Hauptwirkstoff} & Äther && \textbf{Pflückprobe} & - \\ \cline{1-2} \cline{4-5} 
  	\textbf{Sekundärwirkstoff} & Rubedo && \textbf{Menge} & 1 pro Pflanze \\ \cline{1-2} \cline{4-5} 
  	\textbf{Farbe} & lila (2) && \textbf{Vorkommen} & Feld (mittel) \\ \cline{1-2} \cline{4-5} 
  	\textbf{Geruch} & nieswurz (2) && \textbf{Region} & \brcell{gemäßigte und tro- \\ pische Temparaturen} \\ \cline{1-2} \cline{4-5} 
  	\textbf{Geschmack} & neutral && \textbf{Wert} & 9Kr \\ \cline{1-2} \cline{4-5} 
  	\textbf{Spezialeigenschaften} & \brcell{Gesund \\ Aufputschmittel} && \textbf{Utensilien} & - \\ \cline{1-2} \cline{4-5} 
\end{tabular} 
\end{center} 
\caption{Nieswurzblüten} 
\label{tab:nieswurzblueten} 
\end{table}

\subsection{Pimentwurzel}
Pimentwurzel ist eine Pflanze, die von den Druiden gezüchtet wird. Sie ist daher in der freien Natur nicht zu finden, sondern nur bei Druiden zu kaufen.

Heute werden vom Piment besonders die unreifen Früchte für Gewürze und Parfüms verwendet.

\begin{table}[h] 
\begin{center} 
\begin{tabular}{|l|l|p{1cm}|l|l|} 
  	\cline{1-2} \cline{4-5} 
  	\textbf{Hauptwirkstoff} & Äther && \textbf{Pflückprobe} & - \\ \cline{1-2} \cline{4-5} 
  	\textbf{Sekundärwirkstoff} & Nigredo && \textbf{Menge} & n.a. \\ \cline{1-2} \cline{4-5} 
  	\textbf{Farbe} & braun (1) && \textbf{Vorkommen} & (siehe Beschreibung) \\ \cline{1-2} \cline{4-5} 
  	\textbf{Geruch} & erdig (1) && \textbf{Region} & (siehe Bescheibung) \\ \cline{1-2} \cline{4-5} 
  	\textbf{Geschmack} & bitter (2) && \textbf{Wert} & 15Kr \\ \cline{1-2} \cline{4-5} 
  	\textbf{Spezialeigenschaften} & Gesund && \textbf{Utensilien} & - \\ \cline{1-2} \cline{4-5} 
\end{tabular} 
\end{center} 
\caption{Pimentwurzel} 
\label{tab:pimentwurzel} 
\end{table}

\subsection{Schöllkraut}

\subsection{Sewanten}

\subsection{Wolfsaloe}

\subsection{Wolfsbann}

\subsection{Zaunrübenwurzel}



\section{Mineralien}

\subsection{Ginaz-Säure}

\subsection{Kalzium Equum}

\subsection{Lunasplitter}

\subsection{Weißer Essig}