{\let\clearpage\relax\chapter{Herbarium}}
Hier sind alle Ingredienzen aufgeführt, die für die Herstellung von Gebräue wie Tränke, Öle oder Bomben benutzt werden können. Ingredienzen können eine \textit{Pflückprobe} und \textit{Utensilien} definieren. Die \textit{Pflückprobe} gibt an wie eine Probe auf \textit{Alchemie} erfüllt werden muss, damit die Ingredienz erfolgreich gepflückt wird. \textit{Utensilien} geben an was zum Pflücken benötigt wird. Wenn nicht anders angegeben, muss die \textit{Pflückprobe} ausgeführt werden, wenn kein angegebenes \textit{Utensil} verwendet wird. Wenn die \textit{Pflückprobe} misslingt wird die Ingredienz zerstört (wenn nicht anders angegeben).

\section{Wirkstoffe}
Alle Ingredienzen bestehen aus mindestens einem Wirkstoff. Gebräue benötigen häufig als Zutaten keine konkreten Ingredienzen sondern Wirkstoffe. In diesem Kapitel befindet sich eine Übersicht zu allen Hauptwirkstoffen. Dazu gehört auch eine Liste von allen Ingredienzen, die den entsprechenden Wirkstoff enthält. Häufig enthält ein Wirkstoff eines der drei alchemistischen Zusatzstoffe Albedo, Nigredo oder Rubedo. Diese werden auch Sekundärwirkstoffe genannt. 

\subsection{Hauptwirkstoffe}
Zu den Hauptwirkstoffen gehört Äther, Hydragenum, Karmin, Quebrith, Rebis und Vitriol. 

\subsubsection{Äther}
Äther kommt in folgenden Zutaten vor:

\begin{longtable}{p{5cm}p{2cm}p{5cm}}
\textbf{Pflanzen} & & \textbf{Mineralien} \\ \cline{1-1} \cline{3-3}
Berberrohrfrucht & & Pulverisierte Perle \\ 
Ignatia Blüten & & Königsstein \\ 
Nieswurzblüten  & & Quecksilberlösung \\ 
Pimentwurzel & &  \\ 

\caption{Ingredienzen mit Äther}
\label{tab:ingredienzen_mit_aether}
\end{longtable}

Äther mit Zusatzstoffen:
\begin{itemize}
\item \textbf{Albedo:} Berberrohrfrucht
\item \textbf{Nigredo:} Pimentwurzel, Ignatia Blüten, Quecksilberlösung
\item \textbf{Rubedo:} Nieswurzblüten, Pulverisierte Perle
\end{itemize}


\subsubsection{Hydragenum}
Äther kommt in folgenden Zutaten vor:
\begin{longtable}{p{5cm}p{2cm}p{5cm}}
\textbf{Pflanzen} & & \textbf{Mineralien} \\ \cline{1-1} \cline{3-3}
Büschelkrautblüten & & Kohle \\ 
Mistelzweig & &  \\ 
Wolfsaloe  & &  \\

\caption{Ingredienzen mit Hydragenum}
\label{tab:ingredienzen_mit_hydragenum}
\end{longtable}

Hydragenum mit Zusatzstoffen:
\begin{itemize}
\item \textbf{Albedo:} Wolfsaloe
\item \textbf{Nigredo:} Mistelzweig, Kohle
\item \textbf{Rubedo:} Büschelkrautblüten
\end{itemize}


\subsubsection{Karmin}
Karmin kommt in folgenden Zutaten vor:
\begin{longtable}{p{5cm}p{2cm}p{5cm}}
\textbf{Pflanzen} & & \textbf{Mineralien} \\ \cline{1-1} \cline{3-3}
Zaunrübenwurzel & & Phosphor \\
Zaunrübe & & \\
Mutterkornsamen & & \\
Feainnwedd & & \\
Wolfsbann  & & \\

\caption{Ingredienzen mit Karmin}
\label{tab:ingredienzen_mit_karmin}
\end{longtable}

Karmin mit Zusatzstoffen:
\begin{itemize}
\item \textbf{Albedo:} Wolfsbann
\item \textbf{Nigredo:} Zaunrübenwurzel, Zaunrübe
\item \textbf{Rubedo:} Mutterkornsamen, Feainnwedd
\end{itemize}


\subsubsection{Rebis}
Rebis kommt in folgenden Zutaten vor:
\begin{longtable}{p{5cm}p{2cm}p{5cm}}
\textbf{Pflanzen} & & \textbf{Mineralien} \\ \cline{1-1} \cline{3-3}
Schöllkraut & & Lunasplitter \\
Grünschimmel & & Weinstein \\
Hanfasern & & Kaliumnitrat \\

\caption{Ingredienzen mit Rebis}
\label{tab:ingredienzen_mit_rebis}
\end{longtable}

Rebis mit Zusatzstoffen:
\begin{itemize}
\item \textbf{Albedo:} Kaliumnitrat
\item \textbf{Nigredo:} Schöllkraut, Hanfasern
\item \textbf{Rubedo:} Grünschimmel, Lunasplitter
\end{itemize}


\subsubsection{Quebrith}
Quebrith kommt in folgenden Zutaten vor:
\begin{longtable}{p{5cm}p{2cm}p{5cm}}
\textbf{Pflanzen} & & \textbf{Mineralien} \\ \cline{1-1} \cline{3-3}
Balissafrucht & & Herzogswasser \\
Hundspetersilie & & Optima Mater \\
Geisblatt & & Schwefel \\
Alraunenwurzel & & \\
Eisenkraut & & \\

\caption{Ingredienzen mit Quebrith}
\label{tab:ingredienzen_mit_quebrith}
\end{longtable}

Quebrith mit Zusatzstoffen:
\begin{itemize}
\item \textbf{Albedo:} Geisblatt, Eisenkraut
\item \textbf{Nigredo:} Alraunenwurzel, Herzogswasser
\item \textbf{Rubedo:} Balissafrucht, Hundspetersilie, Optima Mater
\end{itemize}


\subsubsection{Vitriol}
Vitriol kommt in folgenden Zutaten vor:
\begin{longtable}{p{5cm}p{2cm}p{5cm}}
\textbf{Pflanzen} & & \textbf{Mineralien} \\ \cline{1-1} \cline{3-3}
Krähenauge & & Kalzium Equum \\
Hopfendolden & & Weißer Essig \\
Sewanten & & \\
Weiße Myrte & & \\

\caption{Ingredienzen mit Vitriol}
\label{tab:ingredienzen_mit_vitriol}
\end{longtable}

Vitriol mit Zusatzstoffen:
\begin{itemize}
\item \textbf{Albedo:} Weiße Myrte
\item \textbf{Nigredo:} Krähenauge, Hopfendolden
\item \textbf{Rubedo:} Sewanten, Kalzium Equum
\end{itemize}


\subsection{Sekundärwirkstoffe}
\label{chap:sekundaerwirkstoffe}
Es gibt drei Zusatzstoffe, auch Sekundärwirkstoffe genannt. Albedo, Nigredo und Rubedo. Einige Ingredienzen haben neben dem Hauptwirkstoff auch einen Sekundärwirkstoff. Jeder Zusatzstoff hat einen eigenen Effekt, den er auf den Trank überträgt, wenn bei dessen Herstellung nur Zusatzstoffe des entsprechenden Sekundärwirkstoffes verwendet wird. Werden für ein Trank nur Zutaten mit dem Sekundärwirkstoff \textit{Albedo} verwendet, wird daraus ein \textit{Albedo}-Trank mit zusätzlichen Effekten. Das selbe geht auch bei Ölen und Bomben, auch wenn die Effekt dort nicht immer sinn ergeben oder sogar kontraproduktiv sind.

\subsubsection{Albedo}
Albedo (lateinisch albedo = Weisheit) ist eine alchemistische Nebensubstanz. Enthalten ist sie in vielen Kräutern, Mineralien und auch in den Kadavern von Ungeheuern.

Albedo-Tränke weisen eine verminderte Toxizität auf. Zudem rufen sie den eine Stunde dauernden Effekt "Albedo dominant" hervor, welcher die Toxizität in dieser Zeit eingenonmmener Tränke senkt. D.h., dass Rausch-, Gift oder Halluzinogene Effekt neutralisiert werden. 

Vorsicht, \textit{Albedo}-Tränke/-Bomben können nicht gleichzeitig Gift-Tränke/-Bomben sein, da der Gift vom \textit{Albedo} neutralisiert wird.

\subsubsection{Nigredo}
Nigredo (lateinisch nigredo = Schwärze) ist eine alchemistische Nebensubstanz. Enthalten ist sie in vielen Kräutern, Mineralien und auch in den Kadavern von Ungeheuern.

Nigredo-Tränke erhöhen die Koordinationsfähigkeit des Hexers und somit die Chance, den Gegner zu treffen $\rightarrow$ +2TW. Der SL kann außerdem auf motorische Talente Erleichterungen geben. Die Wirkung von Nigredo hält 2 Stunden.

\subsubsection{Rudedo}
Rubedo (lateinisch rubedo = Rötung) ist eine alchemistische Nebensubstanz. Enthalten ist sie in vielen Kräutern, Mineralien und auch in den Kadavern von Ungeheuern.

Rubedo-Tränke regenerieren zusätzlich zu den normalen Effekten Lebenspunkte $\rightarrow$ +1W12 LP. Außerdem werden die maximalen LP für 2 Stunden um den selben Wert erhöht. Lässt sich gut mit Heiltränken kombinieren.


\section{Pflanzen}
Im ersten Teil des Herbariums befinden sich alle Pflanzen, die für die Herstellung von Tränken, Öle und Bomben verwendet werden können.

\subsection{Alraunenwurzel}
Die Alraunenwurzel oder auch Mandragora ist eine giftige Pflanze, enthält aber magische Eigenschaften. Druiden verwenden diese Wurzel häufig. Die gelben Beeren und die Blätter sind essbar. Die Wurzeln sondern bei Verletzung ein Toxin mit stark halluzinogener Wirkung ab. Die Wurzeln haben oft das Aussehen eines Menschen.

Mit Handschuhen und/oder mit Werkzeug zum Graben (z.B. ein Spaten) wird die \textit{Pflückprobe} um je 1 QS einfacher.

Bei Fehlschlag 1W4 Tage schwere Halluzinationen.

\begin{table}[H] 
\begin{center} 
\begin{tabular}{|l|l|p{1cm}|l|l|} 
  	\cline{1-2} \cline{4-5} 
  	\textbf{Hauptwirkstoff} & Quebrith && \textbf{Pflückprobe} & QS III \\ \cline{1-2} \cline{4-5} 
  	\textbf{Sekundärwirkstoff} & Nigredo && \textbf{Menge} & 1 pro Pflanze \\ \cline{1-2} \cline{4-5} 
  	\textbf{Farbe} & braun (2) && \textbf{Vorkommen} & \brcell{Wald (häufig) \\ Höhle (mittel)} \\ \cline{1-2} \cline{4-5} 
  	\textbf{Geruch} & \brcell{flieder- \\ stachelbeere (2)} && \textbf{Region} & \brcell{gemäßigte \\ Temperaturen} \\ \cline{1-2} \cline{4-5} 
  	\textbf{Geschmack} & mehlig (2) && \textbf{Wert} & 15Kr \\ \cline{1-2} \cline{4-5} 
  	\textbf{Spezialeigenschaften} & \brcell{Gift \\ Halluzinogen} && \textbf{Utensilien} & (siehe Beschreibung) \\ \cline{1-2} \cline{4-5} 
\end{tabular} 
\end{center} 
\caption{Alraunenwurzel} 
\label{tab:alraunenwurzel} 
\end{table}


\subsection{Balissafrucht}
Balissafrucht ist eine essbare Frucht die am Balissastrauch wächst. Sie zeichnet sich durch eine zarte magische Resonanz aus. 

\begin{table}[H]
\begin{center}
\begin{tabular}{|l|l|p{1cm}|l|l|}
	\cline{1-2} \cline{4-5}
	\textbf{Hauptwirkstoff} & Quebrith && \textbf{Pflückprobe} & - \\ \cline{1-2} \cline{4-5}
	\textbf{Sekundärwirkstoff} & Rubedo && \textbf{Menge} & 2+1W6 pro Strauch \\ \cline{1-2} \cline{4-5}
	\textbf{Farbe} & violett (2) && \textbf{Vorkommen} & Feld (mittel), Wald (häufig) \\ \cline{1-2} \cline{4-5}
	\textbf{Geruch} & fruchtig (1) && \textbf{Region} & überall \\ \cline{1-2} \cline{4-5}
	\textbf{Geschmack} & süß (2) && \textbf{Wert} & 12Kr \\ \cline{1-2} \cline{4-5}
	\textbf{Spezialeigenschaften} & - && \textbf{Utensilien} & - \\ \cline{1-2} \cline{4-5}
\end{tabular}
\end{center}
\caption{Balissafrucht}
\label{tab:balissafrucht}
\end{table}


\subsection{Berberrohrfrucht}
Die Berberrohrfrucht schmeckt sauer und ist saftig. Sie wächst an einem Strauch mit Dornen. Der Strauch wird bis zu 1,5m hoch. 

Misslingt die \textit{Pflückprobe} (aber \textit{Alchemie}-Probe gelingt) bekommt der Held 1 Schaden pro QS die er zu niedrig ist. Misslingt die \textit{Alchemie}-Probe fällt der Held in den Strauch und bekommt zusätzlich 1 Schaden pro Wert den er unterschritten hat. 

Bsp.: Misslingt die Probe um 2 bekommt er 2 Schaden und zusätzlich 3 Schaden wegen der nicht erreichten QS IV.

\begin{table}[H]
\begin{center}
\begin{tabular}{|l|l|p{1cm}|l|l|}
	\cline{1-2} \cline{4-5}
	\textbf{Hauptwirkstoff} & Äther && \textbf{Pflückprobe} & QS IV \\ \cline{1-2} \cline{4-5}
	\textbf{Sekundärwirkstoff} & Albedo && \textbf{Menge} & 1W12 pro Strauch \\ \cline{1-2} \cline{4-5}
	\textbf{Farbe} & grün (2) && \textbf{Vorkommen} & Feld (häufig) \\ \cline{1-2} \cline{4-5}
	\textbf{Geruch} & neutral && \textbf{Region} & überall \\ \cline{1-2} \cline{4-5}
	\textbf{Geschmack} & sauer (2) && \textbf{Wert} & 21Kr \\ \cline{1-2} \cline{4-5}
	\textbf{Spezialeigenschaften} & - && \textbf{Utensilien} & Handschuh \\ \cline{1-2} \cline{4-5}
\end{tabular}
\end{center}
\caption{Berberrohrfrucht}
\label{tab:berberrohrfrucht}
\end{table}


\subsection{Büschelkrautblüten}
Mit den auffällig roten Blüten ist Büschelkraut eine weitverbreitete Pflanze, die auf Wiesen und feuchten Böden wächst. Die Blütenblätter enthalten ein leichtes Halluzinogen, sind giftig und werden für Fisstech verwendet. 

\begin{table}[H] 
\begin{center} 
\begin{tabular}{|l|l|p{1cm}|l|l|} 
  	\cline{1-2} \cline{4-5} 
  	\textbf{Hauptwirkstoff} & Hydragenum && \textbf{Pflückprobe} & - \\ \cline{1-2} \cline{4-5} 
  	\textbf{Sekundärwirkstoff} & Rubedo && \textbf{Menge} & 1 pro Pflanze \\ \cline{1-2} \cline{4-5} 
  	\textbf{Farbe} & rot (3) && \textbf{Vorkommen} & \brcell{Sumpf (selten)\\Feld (selten)} \\ \cline{1-2} \cline{4-5} 
  	\textbf{Geruch} & büschelkraut (2) && \textbf{Region} & überall \\ \cline{1-2} \cline{4-5} 
  	\textbf{Geschmack} & bitter (3) && \textbf{Wert} & 15Kr \\ \cline{1-2} \cline{4-5} 
  	\textbf{Spezialeigenschaften} & Gift, Halluzinogen && \textbf{Utensilien} & - \\ \cline{1-2} \cline{4-5} 
\end{tabular} 
\end{center} 
\caption{Büschelkrautblüten} 
\label{tab:bueschelkrautblueten} 
\end{table}


\subsection{Eisenkraut}
Eisenkraut ist eine weit verbreitete, blühende Pflanze. 

\begin{table}[H] 
\begin{center} 
\begin{tabular}{|l|l|p{1cm}|l|l|} 
  	\cline{1-2} \cline{4-5} 
  	\textbf{Hauptwirkstoff} & Quebrith && \textbf{Pflückprobe} & - \\ \cline{1-2} \cline{4-5} 
  	\textbf{Sekundärwirkstoff} & Albedo && \textbf{Menge} & 1 pro Pflanze \\ \cline{1-2} \cline{4-5} 
  	\textbf{Farbe} & rosa (1) && \textbf{Vorkommen} & Ödland (häufig) \\ \cline{1-2} \cline{4-5} 
  	\textbf{Geruch} & eisenkraut (1) && \textbf{Region} & überall \\ \cline{1-2} \cline{4-5} 
  	\textbf{Geschmack} & bitter (1) && \textbf{Wert} & 15Kr \\ \cline{1-2} \cline{4-5} 
  	\textbf{Spezialeigenschaften} & - && \textbf{Utensilien} & - \\ \cline{1-2} \cline{4-5} 
\end{tabular} 
\end{center} 
\caption{Eisenkraut} 
\label{tab:eisenkraut} 
\end{table}


\subsection{Feainnwedd}
Feainnwedd ist eine Blume mit wundervollem Duft, die einer Legende nach nur an Orten wächst, an dem \textit{Älteren-Blut} vergossen wurde sowie im \textit{Tal der Blumen}. Die Elfen erzählen, dass Feainnwedd erst zu wachsen begann, nachdem \textit{Lara Dorren} gestorben war.

In der alten Sprache bedeutet Faen "`Sonne"' und wedd "`Kind"', was hier möglicherweise "`Kind der Sonne"' oder "`Sonnenkind"' heißt. 

\begin{table}[H] 
\begin{center} 
\begin{tabular}{|l|l|p{1cm}|l|l|} 
  	\cline{1-2} \cline{4-5} 
  	\textbf{Hauptwirkstoff} & Karmin && \textbf{Pflückprobe} & - \\ \cline{1-2} \cline{4-5} 
  	\textbf{Sekundärwirkstoff} & Rubedo && \textbf{Menge} & 1 pro Pflanze \\ \cline{1-2} \cline{4-5} 
  	\textbf{Farbe} & lila (2) && \textbf{Vorkommen} & \brcell{siehe Beschreibung \\ (selten)} \\ \cline{1-2} \cline{4-5} 
  	\textbf{Geruch} & Feainnwedd (2) && \textbf{Region} & siehe Beschreibung \\ \cline{1-2} \cline{4-5} 
  	\textbf{Geschmack} & neutral && \textbf{Wert} & 50Kr \\ \cline{1-2} \cline{4-5} 
  	\textbf{Spezialeigenschaften} & - && \textbf{Utensilien} & - \\ \cline{1-2} \cline{4-5} 
\end{tabular} 
\end{center} 
\caption{Feainnwedd} 
\label{tab:feainnwedd} 
\end{table}


\subsection{Geisblatt}
Geisblatt nennen sich die Blätter einer widerstandsfähigen Pflanze die vorzugsweise auf Ödland wächst als auch auf Wiesen und Feldern. 

\begin{table}[H] 
\begin{center} 
\begin{tabular}{|l|l|p{1cm}|l|l|} 
  	\cline{1-2} \cline{4-5} 
  	\textbf{Hauptwirkstoff} & Quebrith && \textbf{Pflückprobe} & - \\ \cline{1-2} \cline{4-5} 
  	\textbf{Sekundärwirkstoff} & Albedo && \textbf{Menge} & 1 pro Pflanze \\ \cline{1-2} \cline{4-5} 
  	\textbf{Farbe} & grün (2) && \textbf{Vorkommen} & \brcell{Ödland (mittel) \\ Feld (mittel)} \\ \cline{1-2} \cline{4-5} 
  	\textbf{Geruch} & neutral && \textbf{Region} & \brcell{überall \\ Brokilon (häufig)} \\ \cline{1-2} \cline{4-5} 
  	\textbf{Geschmack} & neutral && \textbf{Wert} & 21Kr \\ \cline{1-2} \cline{4-5} 
  	\textbf{Spezialeigenschaften} & - && \textbf{Utensilien} & - \\ \cline{1-2} \cline{4-5} 
\end{tabular} 
\end{center} 
\caption{Geisblatt} 
\label{tab:geisblatt} 
\end{table}


\subsection{Grünschimmel}
Grünschimmel wächst an den Wänden dunkler und feuchter Orte, wie z. B. in natürlichen Höhlen oder auch in Abwasserkanälen. Es sieht ein auf den ersten Blick aus wie Moos ist aber haarig und nicht so weich.

Zum Pflücken benötigt man etwas zum abscharben, z.B. ein Messer.

\begin{table}[H] 
\begin{center} 
\begin{tabular}{|l|l|p{1cm}|l|l|} 
  	\cline{1-2} \cline{4-5} 
  	\textbf{Hauptwirkstoff} & Rebis && \textbf{Pflückprobe} & QS 3 \\ \cline{1-2} \cline{4-5} 
  	\textbf{Sekundärwirkstoff} & Rubedo && \textbf{Menge} & 2W4 pro Ort \\ \cline{1-2} \cline{4-5} 
  	\textbf{Farbe} & grün (2) && \textbf{Vorkommen} & \brcell{Höhle (häufig) \\ Sumpf (selten)} \\ \cline{1-2} \cline{4-5} 
  	\textbf{Geruch} & übelriechend (3) && \textbf{Region} & überall \\ \cline{1-2} \cline{4-5} 
  	\textbf{Geschmack} & schimmel (2) && \textbf{Wert} & 15Kr \\ \cline{1-2} \cline{4-5} 
  	\textbf{Spezialeigenschaften} & - && \textbf{Utensilien} & Messer \\ \cline{1-2} \cline{4-5} 
\end{tabular} 
\end{center} 
\caption{Grünschimmel} 
\label{tab:gruenschimmel} 
\end{table}


\subsection{Hanfasern}
Hanfasern sehen aus wie karge und blasse Grasbüschel und werden deshalb häufig für Unkraut gehalten. Sie sind eine weit verbreitete Pflanze auf Wiesen und Feldern mit gutem Nährboden. Hanfasern sind giftig. Wenn man sie mit einer Sichel pflückt, erhöht sich der Ertrag um 2. Häufig werden Bauernhof Tiere krank, weil sie sie essen und es nicht vom Bauern bemerkt wird.

\begin{table}[H] 
\begin{center} 
\begin{tabular}{|l|l|p{1cm}|l|l|} 
  	\cline{1-2} \cline{4-5} 
  	\textbf{Hauptwirkstoff} & Rebis && \textbf{Pflückprobe} & - \\ \cline{1-2} \cline{4-5} 
  	\textbf{Sekundärwirkstoff} & Nigredo && \textbf{Menge} & 1W4 \\ \cline{1-2} \cline{4-5} 
  	\textbf{Farbe} & grün (1) && \textbf{Vorkommen} & Feld (häufig) \\ \cline{1-2} \cline{4-5} 
  	\textbf{Geruch} & neutral && \textbf{Region} & überall \\ \cline{1-2} \cline{4-5} 
  	\textbf{Geschmack} & neutral && \textbf{Wert} & 21Kr \\ \cline{1-2} \cline{4-5} 
  	\textbf{Spezialeigenschaften} & Gift && \textbf{Utensilien} & Sichel \\ \cline{1-2} \cline{4-5} 
\end{tabular} 
\end{center} 
\caption{Hanfasern} 
\label{tab:hanfasern} 
\end{table}


\subsection{Hopfendolden}
Hopfendolden sind eine gebräuchliche alchemistische Zutat für Tränke. Sie wird auch zum Bier brauen verwendet.

\begin{table}[H] 
\begin{center} 
\begin{tabular}{|l|l|p{1cm}|l|l|} 
  	\cline{1-2} \cline{4-5} 
  	\textbf{Hauptwirkstoff} & Vitriol && \textbf{Pflückprobe} & - \\ \cline{1-2} \cline{4-5} 
  	\textbf{Sekundärwirkstoff} & Nigredo && \textbf{Menge} & 1 pro Pflanze \\ \cline{1-2} \cline{4-5} 
  	\textbf{Farbe} & grün (1) && \textbf{Vorkommen} & Feld (häufig) \\ \cline{1-2} \cline{4-5} 
  	\textbf{Geruch} & hopfen (2) && \textbf{Region} & überall \\ \cline{1-2} \cline{4-5} 
  	\textbf{Geschmack} & herb (2) && \textbf{Wert} & 21Kr \\ \cline{1-2} \cline{4-5} 
  	\textbf{Spezialeigenschaften} & - && \textbf{Utensilien} & - \\ \cline{1-2} \cline{4-5} 
\end{tabular} 
\end{center} 
\caption{Hopfendolden} 
\label{tab:hopfendolden} 
\end{table}


\subsection{Hundspetersilie}
Hundspetersilie ist eine Sumpfpflanze mit fleischigen, großen, grünen Blättern. Die Pflanze gedeiht üppig in sumpfigen Gebieten, wie zum Beispiel im Friedhofssumpf von Alt-Vizima und in den Sümpfen von Vizima. Die Blätter dieser Pflanze bestehen aus Fasern, mit denen sich ein magisches Hemd weben lässt. 

Bei der alchemistischen Zubereitung erhält man eine dickflüssige Substanz, die häufig für Öle verwendet wird. Wird außerdem gerne verwendet um Soßen dickflüssiger zu machen und um ihr einen frischen Kräutergeschmack zu verleihen.

\begin{table}[H] 
\begin{center} 
\begin{tabular}{|l|l|p{1cm}|l|l|} 
  	\cline{1-2} \cline{4-5} 
  	\textbf{Hauptwirkstoff} & Quebrith && \textbf{Pflückprobe} & - \\ \cline{1-2} \cline{4-5} 
  	\textbf{Sekundärwirkstoff} & Rubedo && \textbf{Menge} & 2+1W4 pro Pflanze \\ \cline{1-2} \cline{4-5} 
  	\textbf{Farbe} & grün (2) && \textbf{Vorkommen} & Sumpf (häufig) \\ \cline{1-2} \cline{4-5} 
  	\textbf{Geruch} & petersilie (2) && \textbf{Region} & überall \\ \cline{1-2} \cline{4-5} 
  	\textbf{Geschmack} & petersilie (2) && \textbf{Wert} & 15Kr \\ \cline{1-2} \cline{4-5} 
  	\textbf{Spezialeigenschaften} & Dickflüssig && \textbf{Utensilien} & - \\ \cline{1-2} \cline{4-5} 
\end{tabular} 
\end{center} 
\caption{Hundspetersilie} 
\label{tab:hundspetersilie} 
\end{table}

\subsection{Ignatia Blüten}
Ignatia ist ein Busch, der recht häufig in \textit{Temerien} und im \textit{Kaer Morhen Tal} vorkommt. Die Ignatia Blüten finden häufig Verwendung in der Alchemie.

Ignatia wird auch Mondrose genannt. In Legenden heißt es, die Pflanze wächst dort, wo eine Sternschnuppe auf die Erde gefallen ist.

\begin{table}[H] 
\begin{center} 
\begin{tabular}{|l|l|p{1cm}|l|l|} 
  	\cline{1-2} \cline{4-5} 
  	\textbf{Hauptwirkstoff} & Äther && \textbf{Pflückprobe} & - \\ \cline{1-2} \cline{4-5} 
  	\textbf{Sekundärwirkstoff} & Nigredo && \textbf{Menge} & 2+1W4 pro Busch \\ \cline{1-2} \cline{4-5} 
  	\textbf{Farbe} & rosa (2) && \textbf{Vorkommen} & \brcell{Feld (häufig) \\ Wald (häufig)} \\ \cline{1-2} \cline{4-5} 
  	\textbf{Geruch} & ignatia (2) && \textbf{Region} & \brcell{Temerien \\ Kear Morgen Tal} \\ \cline{1-2} \cline{4-5} 
  	\textbf{Geschmack} & neutral && \textbf{Wert} & 15Kr \\ \cline{1-2} \cline{4-5} 
  	\textbf{Spezialeigenschaften} & - && \textbf{Utensilien} & - \\ \cline{1-2} \cline{4-5} 
\end{tabular} 
\end{center} 
\caption{Ignatia Blüten} 
\label{tab:ignatia_blueten} 
\end{table}

\subsection{Krähenauge}
Die Wurzel eines zypressenartigen Busches, die häufig in der Alchemie verwendet wird. Um an die Knollenwurzel zu gelangen muss der gesamte Busch ausgegraben werden. Die Knolle ist so groß wie eine Zwiebel und hat die Form und Farbe einer Kartoffel.

\begin{table}[H] 
\begin{center} 
\begin{tabular}{|l|l|p{1cm}|l|l|} 
  	\cline{1-2} \cline{4-5} 
  	\textbf{Hauptwirkstoff} & Vitriol && \textbf{Pflückprobe} & QS IV \\ \cline{1-2} \cline{4-5} 
  	\textbf{Sekundärwirkstoff} & Negredo && \textbf{Menge} & 1 pro Busch \\ \cline{1-2} \cline{4-5} 
  	\textbf{Farbe} & braun (1) && \textbf{Vorkommen} & Feld (häufig) \\ \cline{1-2} \cline{4-5} 
  	\textbf{Geruch} & erdig (1) && \textbf{Region} & überall \\ \cline{1-2} \cline{4-5} 
  	\textbf{Geschmack} & bitter (2) && \textbf{Wert} & 15Kr \\ \cline{1-2} \cline{4-5} 
  	\textbf{Spezialeigenschaften} & - && \textbf{Utensilien} & Spaten, Schaufel \\ \cline{1-2} \cline{4-5} 
\end{tabular} 
\end{center} 
\caption{Krähenauge} 
\label{tab:kraehenauge} 
\end{table}

\subsection{Mistelzweig}
Mistelzweige sind heiligen Pflanzen der Druiden. Sie verwenden sie für Heilmittel gegen Krankheiten. Man findet sie als Opfergaben bei Altaren der Melitele. Es gibt nur wenige Orte an denen sie natürlich Wachsen und nur Druiden und manche Anhänger der Melitele kennen diese Orte.

\begin{table}[H] 
\begin{center} 
\begin{tabular}{|l|l|p{1cm}|l|l|} 
  	\cline{1-2} \cline{4-5} 
  	\textbf{Hauptwirkstoff} & Hydragenum && \textbf{Pflückprobe} & - \\ \cline{1-2} \cline{4-5} 
  	\textbf{Sekundärwirkstoff} & Nigredo && \textbf{Menge} & unbekannt \\ \cline{1-2} \cline{4-5} 
  	\textbf{Farbe} & grün (1) && \textbf{Vorkommen} & unbekannt (selten) \\ \cline{1-2} \cline{4-5} 
  	\textbf{Geruch} & tanne (1) && \textbf{Region} & \brcell{unbekannt \\ (siehe Beschreibung)} \\ \cline{1-2} \cline{4-5} 
  	\textbf{Geschmack} & neutral && \textbf{Wert} & 21Kr \\ \cline{1-2} \cline{4-5} 
  	\textbf{Spezialeigenschaften} & Gesund && \textbf{Utensilien} & - \\ \cline{1-2} \cline{4-5} 
\end{tabular} 
\end{center} 
\caption{Mistelzweig} 
\label{tab:mistelzweig} 
\end{table}

\subsection{Mutterkornsamen}
Mutterkornsamen ist die Saat einer rituellen Pflanze. Diese Pflanze, die als Unkraut bezeichnet wird, sieht aus wie ein kleiner Gerstenhalm. Aus der Blüte erhält man die Samen. Sie wird manchmal als Ersatz für Gerste verwendet, da sie fast die selben Eigenschaften hat.

\begin{table}[H] 
\begin{center} 
\begin{tabular}{|l|l|p{1cm}|l|l|} 
  	\cline{1-2} \cline{4-5} 
  	\textbf{Hauptwirkstoff} & Karmin && \textbf{Pflückprobe} & - \\ \cline{1-2} \cline{4-5} 
  	\textbf{Sekundärwirkstoff} & Rubedo && \textbf{Menge} & 1W4 pro Pflanze \\ \cline{1-2} \cline{4-5} 
  	\textbf{Farbe} & gelb-braun (2) && \textbf{Vorkommen} & Feld (häufig) \\ \cline{1-2} \cline{4-5} 
  	\textbf{Geruch} & getreide (2) && \textbf{Region} & überall \\ \cline{1-2} \cline{4-5} 
  	\textbf{Geschmack} & gerste (2) && \textbf{Wert} & 15Kr \\ \cline{1-2} \cline{4-5} 
  	\textbf{Spezialeigenschaften} & - && \textbf{Utensilien} & - \\ \cline{1-2} \cline{4-5} 
\end{tabular} 
\end{center} 
\caption{Mutterkornsamen} 
\label{tab:mutterkornsamen} 
\end{table}

\subsection{Nieswurzblüten}
Nieswurzblüten sind eine allgemeine und weit verbreitete Giftpflanze. Eine alte Frau aus dem Umland von Vizima erzählt, dass man aus Nieswurzblüten ein Tonikum gegen Schlaflosigkeit brauen kann.

Auch heute noch wird Nieswurz als Heilpflanze verwendet. 

\begin{table}[H] 
\begin{center} 
\begin{tabular}{|l|l|p{1cm}|l|l|} 
  	\cline{1-2} \cline{4-5} 
  	\textbf{Hauptwirkstoff} & Äther && \textbf{Pflückprobe} & - \\ \cline{1-2} \cline{4-5} 
  	\textbf{Sekundärwirkstoff} & Rubedo && \textbf{Menge} & 1 pro Pflanze \\ \cline{1-2} \cline{4-5} 
  	\textbf{Farbe} & lila (2) && \textbf{Vorkommen} & Feld (mittel) \\ \cline{1-2} \cline{4-5} 
  	\textbf{Geruch} & nieswurz (2) && \textbf{Region} & \brcell{gemäßigte und tro- \\ pische Temparaturen} \\ \cline{1-2} \cline{4-5} 
  	\textbf{Geschmack} & neutral && \textbf{Wert} & 9Kr \\ \cline{1-2} \cline{4-5} 
  	\textbf{Spezialeigenschaften} & \brcell{Gesund \\ Aufputschmittel} && \textbf{Utensilien} & - \\ \cline{1-2} \cline{4-5} 
\end{tabular} 
\end{center} 
\caption{Nieswurzblüten} 
\label{tab:nieswurzblueten} 
\end{table}

\subsection{Pimentwurzel}
Pimentwurzel ist eine Pflanze, die von den Druiden gezüchtet wird. Sie ist daher in der freien Natur nicht zu finden, sondern nur bei Druiden zu kaufen.

Heute werden vom Piment besonders die unreifen Früchte für Gewürze und Parfüms verwendet.

\begin{table}[H] 
\begin{center} 
\begin{tabular}{|l|l|p{1cm}|l|l|} 
  	\cline{1-2} \cline{4-5} 
  	\textbf{Hauptwirkstoff} & Äther && \textbf{Pflückprobe} & - \\ \cline{1-2} \cline{4-5} 
  	\textbf{Sekundärwirkstoff} & Nigredo && \textbf{Menge} & n.a. \\ \cline{1-2} \cline{4-5} 
  	\textbf{Farbe} & braun (1) && \textbf{Vorkommen} & (siehe Beschreibung) \\ \cline{1-2} \cline{4-5} 
  	\textbf{Geruch} & erdig (1) && \textbf{Region} & (siehe Bescheibung) \\ \cline{1-2} \cline{4-5} 
  	\textbf{Geschmack} & bitter (2) && \textbf{Wert} & 15Kr \\ \cline{1-2} \cline{4-5} 
  	\textbf{Spezialeigenschaften} & Gesund && \textbf{Utensilien} & - \\ \cline{1-2} \cline{4-5} 
\end{tabular} 
\end{center} 
\caption{Pimentwurzel} 
\label{tab:pimentwurzel} 
\end{table}


\subsection{Schöllkraut}
Schöllkraut ist eine weit verbreitete allgemeine Pflanze, deren Wirkstoffe viel in der Medizin verwendet wird. In der alten Sprache wird Schöllkraut herba zireael genannt.

Das Schöllkraut ist bis heute auf der nördlichen Halbkugel anzutreffen. 

\begin{table}[H] 
\begin{center} 
\begin{tabular}{|l|l|p{1cm}|l|l|} 
  	\cline{1-2} \cline{4-5} 
  	\textbf{Hauptwirkstoff} & Rebis && \textbf{Pflückprobe} & - \\ \cline{1-2} \cline{4-5} 
  	\textbf{Sekundärwirkstoff} & Nigredo && \textbf{Menge} & 1 pro Pflanze \\ \cline{1-2} \cline{4-5} 
  	\textbf{Farbe} & grün (1) && \textbf{Vorkommen} & \brcell{Feld (selten) \\ Sumpf (mittel)} \\ \cline{1-2} \cline{4-5} 
  	\textbf{Geruch} & kraut (1) && \textbf{Region} & \brcell{überall \\ (siehe Beschreibung)} \\ \cline{1-2} \cline{4-5} 
  	\textbf{Geschmack} & herb (1) && \textbf{Wert} & 12Kr \\ \cline{1-2} \cline{4-5} 
  	\textbf{Spezialeigenschaften} & Gesund && \textbf{Utensilien} & - \\ \cline{1-2} \cline{4-5} 
\end{tabular} 
\end{center} 
\caption{Schöllkraut} 
\label{tab:schoellkraut} 
\end{table}

\subsection{Sewanten}
Sewanten sind große graue Pilze die oft in Höhlen wachsen. Man findet sie in fast allen Höhlen und Grüften und haben einen einzigartigen Geschmack, der von den meisten als angenehm bitter-süß bezeichnet wird. Sie machen einen müde, weshalb sie gerne zum Abendessen verzehrt werden.

Sie sondern Poren ab, die schläfrig und erschöpft machen können. Dementsprechend sollte man sich nicht sehr lange in der Nähe von Sewanten aufhalten.

\begin{table}[H] 
\begin{center} 
\begin{tabular}{|l|l|p{1cm}|l|l|} 
  	\cline{1-2} \cline{4-5} 
  	\textbf{Hauptwirkstoff} & Vitriol && \textbf{Pflückprobe} & - \\ \cline{1-2} \cline{4-5} 
  	\textbf{Sekundärwirkstoff} & Rubedo && \textbf{Menge} & 1W4 pro Ort \\ \cline{1-2} \cline{4-5} 
  	\textbf{Farbe} & grau (1) && \textbf{Vorkommen} & Höhle (häufig) \\ \cline{1-2} \cline{4-5} 
  	\textbf{Geruch} & erdig (1) && \textbf{Region} & überall \\ \cline{1-2} \cline{4-5} 
  	\textbf{Geschmack} & sewanten (2) && \textbf{Wert} & 9Kr \\ \cline{1-2} \cline{4-5} 
  	\textbf{Spezialeigenschaften} & Schläfrig && \textbf{Utensilien} & - \\ \cline{1-2} \cline{4-5} 
\end{tabular} 
\end{center} 
\caption{Sewanten} 
\label{tab:sewanten} 
\end{table}


\subsection{Weiße Myrte}
Die Weiße Myrte ist eine weit verbreiteten Feldpflanze, welche man sehr leicht an ihren großen weißen Blüten erkennen kann. Die Pflanze ist ein importiertes Gewächs und außerdem eine hervorragende Grundlage für die Weinkelterei. Sie hat einen munter und wach-machenden Effekt.

\begin{table}[H] 
\begin{center} 
\begin{tabular}{|l|l|p{1cm}|l|l|} 
  	\cline{1-2} \cline{4-5} 
  	\textbf{Hauptwirkstoff} & Vitriol && \textbf{Pflückprobe} & - \\ \cline{1-2} \cline{4-5} 
  	\textbf{Sekundärwirkstoff} & Albedo && \textbf{Menge} & 1 pro Pflanze \\ \cline{1-2} \cline{4-5} 
  	\textbf{Farbe} & transparent && \textbf{Vorkommen} & Feld (häufig) \\ \cline{1-2} \cline{4-5} 
  	\textbf{Geruch} & myrte (1) && \textbf{Region} & überall \\ \cline{1-2} \cline{4-5} 
  	\textbf{Geschmack} & neutral && \textbf{Wert} & 9Kr \\ \cline{1-2} \cline{4-5} 
  	\textbf{Spezialeigenschaften} & Aufputschmittel && \textbf{Utensilien} & - \\ \cline{1-2} \cline{4-5} 
\end{tabular} 
\end{center} 
\caption{Weiße Myrte} 
\label{tab:weisse_myrte} 
\end{table}


\subsection{Wolfsaloe}
Wolfsaloe stammt ursprünglich aus dem \textit{Kaer Morhen} und ist als Zimmerpflanze weit verbreitet. Die Pflanze versprüht einen angenehmen Duft, der an Flieder erinnert. Man kann aus ihr ein Konzentrat gewinnen, dass einene Rauschlindernden Effekt hat. Magier machen daraus häufig einen Trank gegen Kater.

\begin{table}[H] 
\begin{center} 
\begin{tabular}{|l|l|p{1cm}|l|l|} 
  	\cline{1-2} \cline{4-5} 
  	\textbf{Hauptwirkstoff} & Hydragenum && \textbf{Pflückprobe} & - \\ \cline{1-2} \cline{4-5} 
  	\textbf{Sekundärwirkstoff} & Albedo && \textbf{Menge} & 1W4 pro Pflanze \\ \cline{1-2} \cline{4-5} 
  	\textbf{Farbe} & grün (1) && \textbf{Vorkommen} & Gebirge (mittel) \\ \cline{1-2} \cline{4-5} 
  	\textbf{Geruch} & flieder (1) && \textbf{Region} & überall \\ \cline{1-2} \cline{4-5} 
  	\textbf{Geschmack} & bitter (1) && \textbf{Wert} & 18Kr \\ \cline{1-2} \cline{4-5} 
  	\textbf{Spezialeigenschaften} & Rauschlindernd && \textbf{Utensilien} & - \\ \cline{1-2} \cline{4-5} 
\end{tabular} 
\end{center} 
\caption{Wolfsaloe} 
\label{tab:wolfsaloe} 
\end{table}

\subsection{Wolfsbann}
Wolfsbann ist eine rituelle Pflanze der Druiden und wird auch "`Mönchshut"' genannt. Wolfsbann gibt es in der Natur nicht zu ernten, da sie nur von Druiden und Kräuterkundlern gezüchtet wird. Sie rieht sehr ähnlich wie \textit{Wolfsaloe}, also nach Flieder, schmeckt dafür aber sehr bitter. Sie hat einen betäubenden Effekt.

\begin{table}[H] 
\begin{center} 
\begin{tabular}{|l|l|p{1cm}|l|l|} 
  	\cline{1-2} \cline{4-5} 
  	\textbf{Hauptwirkstoff} & Karmin && \textbf{Pflückprobe} & - \\ \cline{1-2} \cline{4-5} 
  	\textbf{Sekundärwirkstoff} & Albedo && \textbf{Menge} & n.a. \\ \cline{1-2} \cline{4-5} 
  	\textbf{Farbe} & grün (1) && \textbf{Vorkommen} & (siehe Beschreibung) \\ \cline{1-2} \cline{4-5} 
  	\textbf{Geruch} & flieder (1) && \textbf{Region} & (siehe Beschreibung) \\ \cline{1-2} \cline{4-5} 
  	\textbf{Geschmack} & bitter (2) && \textbf{Wert} & 15Kr \\ \cline{1-2} \cline{4-5} 
  	\textbf{Spezialeigenschaften} & Betäubungsgift && \textbf{Utensilien} & - \\ \cline{1-2} \cline{4-5} 
\end{tabular} 
\end{center} 
\caption{Wolfsbann} 
\label{tab:wolfsbann} 
\end{table}

\subsection{Zaunrübe}
Die Stängel der Zaunrübe finden Verwendung in der Alchemie. Zaunrüben wachsen in unfruchtbaren Gegenden und Wiesen. Die Zaunrübe hat eine weiße Blüte.

\begin{table}[H] 
\begin{center} 
\begin{tabular}{|l|l|p{1cm}|l|l|} 
  	\cline{1-2} \cline{4-5} 
  	\textbf{Hauptwirkstoff} & Karmin && \textbf{Pflückprobe} & - \\ \cline{1-2} \cline{4-5} 
  	\textbf{Sekundärwirkstoff} & Nigredo && \textbf{Menge} & 1 pro Pflanze \\ \cline{1-2} \cline{4-5} 
  	\textbf{Farbe} & transparent && \textbf{Vorkommen} & Ödland (mittel) \\ \cline{1-2} \cline{4-5} 
  	\textbf{Geruch} & neutral && \textbf{Region} & überall \\ \cline{1-2} \cline{4-5} 
  	\textbf{Geschmack} & neutral && \textbf{Wert} & 12Kr \\ \cline{1-2} \cline{4-5} 
  	\textbf{Spezialeigenschaften} & Giftneutralisierend && \textbf{Utensilien} & - \\ \cline{1-2} \cline{4-5} 
\end{tabular} 
\end{center} 
\caption{Zaunrübe} 
\label{tab:zaunruebe} 
\end{table}


\subsection{Zaunrübenwurzel}
Die Wurzeln der Zaunrübe wurden oft als Menschentalismane verwendet. Außerdem wird es verwendet um Kinder mit z.B. Lebensmittelvergiftungen zu heilen, da sie es wegen ihrem Geschmack gerne essen.

\begin{table}[H] 
\begin{center} 
\begin{tabular}{|l|l|p{1cm}|l|l|} 
  	\cline{1-2} \cline{4-5} 
  	\textbf{Hauptwirkstoff} & Karmin && \textbf{Pflückprobe} & QS II \\ \cline{1-2} \cline{4-5} 
  	\textbf{Sekundärwirkstoff} & Nigredo && \textbf{Menge} & 1 pro Pflanze \\ \cline{1-2} \cline{4-5} 
  	\textbf{Farbe} & braun (1) && \textbf{Vorkommen} & Ödland (mittel) \\ \cline{1-2} \cline{4-5} 
  	\textbf{Geruch} & neutral && \textbf{Region} & überall \\ \cline{1-2} \cline{4-5} 
  	\textbf{Geschmack} & herb-süß (2) && \textbf{Wert} & 9Kr \\ \cline{1-2} \cline{4-5} 
  	\textbf{Spezialeigenschaften} & Giftneutralisierend && \textbf{Utensilien} & (Hand)spaten \\ \cline{1-2} \cline{4-5} 
\end{tabular} 
\end{center} 
\caption{Zaunrübenwurzel} 
\label{tab:zaunruebenwurzel} 
\end{table}


\section{Mineralien}
In diesem Teil des Herbariums befinden sich alle Mineralien. Sie werden hauptsächlich für die Herstellung von Bomben verwendet, können aber auch für andere Gebräue verwendet werden.


\subsection{Herzogswasser}
Herzogswasser ist ein hochwertiges alchemistisches ätzendes Lösungsmittel. Es ist sehr wertvoll und lohnt sich daher zu verkaufen. Muss in einer speziellen säureresistenten Fiole aufbewahrt werden. Es wird aus dem seltenen Königsstein extrahiert (siehe Kapitel \ref{chap:koenigsstein} und in Wasser aufgelöst. 

Bei der Herstellung müssen zwingend Handschuhe getragen werden. Und selbst mit Handschuhen besteht die Gefahr, dass die Säure auf die Haut gelangt und schwere Verätzungen verursacht. 

Der Spieler muss zur Herstellung die \textit{Pflückprobe} bestehen. Trägt er keine Handschuhe ist die Probe um 4 erschwert. Säureresistente Handschuhe erleichtern die Probe um 4. 

Gelangt die Säure auf die Haut bekommt der Held 2+1W6 Schaden. Wenn versucht wird, die Säure mit Wasser wegzuspülen verschlimmert sich die Verätzung und der Spieler bekommt zusätzlich 1W4 Schaden.

\begin{table}[H] 
\begin{center} 
\begin{tabular}{|l|l|p{1cm}|l|l|} 
  	\cline{1-2} \cline{4-5} 
  	\textbf{Hauptwirkstoff} & Quebrith && \textbf{Pflückprobe} & QS II \\ \cline{1-2} \cline{4-5} 
  	\textbf{Sekundärwirkstoff} & Nigredo && \textbf{Menge} & 1 pro Königsstein \\ \cline{1-2} \cline{4-5} 
  	\textbf{Farbe} & neutral && \textbf{Vorkommen} & (siehe Beschreibung) \\ \cline{1-2} \cline{4-5} 
  	\textbf{Geruch} & essig (2) && \textbf{Region} & (siehe Beschreibung) \\ \cline{1-2} \cline{4-5} 
  	\textbf{Geschmack} & essig (1) && \textbf{Wert} & 50Kr \\ \cline{1-2} \cline{4-5} 
  	\textbf{Spezialeigenschaften} & Ätzend && \textbf{Utensilien} & Handschuhe \\ \cline{1-2} \cline{4-5} 
\end{tabular} 
\end{center} 
\caption{Herzogswasser} 
\label{tab:herzogswasser} 
\end{table}


\subsection{Kaliumnitrat}
\label{chap:kaliumnitrat}
Kaliumnitrat oder Bengalsalpeter, im allgemeinen Sprachgebrauch oft bezeichnet als Salpeter, ist ein farbloser bis weißer Mineral, das als Ausblühung auf Böden vorkommt. Es kann auch synthetisch aus Salpetersäure hergestellt werden. Es wird häufig dazu verwendet Schwarzpulver herzustellen. 

Ausblühungen sollten mit einem Utensil, z.B. einem Messer abgeschabt werden.

\begin{table}[H] 
\begin{center} 
\begin{tabular}{|l|l|p{1cm}|l|l|} 
  	\cline{1-2} \cline{4-5} 
  	\textbf{Hauptwirkstoff} & Rebis && \textbf{Pflückprobe} & QS I \\ \cline{1-2} \cline{4-5} 
  	\textbf{Sekundärwirkstoff} & Albedo && \textbf{Menge} & 1W4 pro Ausblühung \\ \cline{1-2} \cline{4-5} 
  	\textbf{Farbe} & transparent && \textbf{Vorkommen} & \brcell{Ödland (selten) \\ Höhle (selten)} \\ \cline{1-2} \cline{4-5} 
  	\textbf{Geruch} & neutral && \textbf{Region} & überall \\ \cline{1-2} \cline{4-5} 
  	\textbf{Geschmack} & neutral && \textbf{Wert} & 15Kr \\ \cline{1-2} \cline{4-5} 
  	\textbf{Spezialeigenschaften} & Explosiv && \textbf{Utensilien} & (siehe Beschreibung) \\ \cline{1-2} \cline{4-5} 
\end{tabular} 
\end{center} 
\caption{Kaliumnitrat} 
\label{tab:kaliumnitrat} 
\end{table}


\subsection{Kalzium Equum}
Kalzium Equum ist ein weitverbreitetes Mineral, das der Volksmund auch "´Pferdekalzium"' nennt. Es kommt in Höhlen vor und muss mit einer Spitzhacke abgebaut werden. Siehe Kapitel \ref{chap:abbau_von_mineralien} über den Abbau von Mineralien.

\begin{table}[H] 
\begin{center} 
\begin{tabular}{|l|l|p{1cm}|l|l|} 
  	\cline{1-2} \cline{4-5} 
  	\textbf{Hauptwirkstoff} & Vitriol && \textbf{Pflückprobe} & (siehe Beschreibung) \\ \cline{1-2} \cline{4-5} 
  	\textbf{Sekundärwirkstoff} & Rubedo && \textbf{Menge} & 1W6 pro Ader \\ \cline{1-2} \cline{4-5} 
  	\textbf{Farbe} & grau (1) && \textbf{Vorkommen} & Höhle (mittel) \\ \cline{1-2} \cline{4-5} 
  	\textbf{Geruch} & neutral && \textbf{Region} & überall \\ \cline{1-2} \cline{4-5} 
  	\textbf{Geschmack} & neutral && \textbf{Wert} & 30Kr \\ \cline{1-2} \cline{4-5} 
  	\textbf{Spezialeigenschaften} & - && \textbf{Utensilien} & Spitzhacke \\ \cline{1-2} \cline{4-5} 
\end{tabular} 
\end{center} 
\caption{Kalzium Equum} 
\label{tab:kalzium_equum} 
\end{table}


\subsection{Kohle}
\label{chap:kohle}
Kohle kann aus Höhlen mit Spitzhacken abgebaut werden. Ohne Spitzhacke kann die Kohle nicht abgebaut werden. Kohlepulver wird zur Herstellung von Schwarzpulver verwendet. Siehe Kapitel \ref{chap:abbau_von_mineralien} über den Abbau von Mineralien.

\begin{table}[H] 
\begin{center} 
\begin{tabular}{|l|l|p{1cm}|l|l|} 
  	\cline{1-2} \cline{4-5} 
  	\textbf{Hauptwirkstoff} & Hydragenum && \textbf{Pflückprobe} & (siehe Beschreibung) \\ \cline{1-2} \cline{4-5} 
  	\textbf{Sekundärwirkstoff} & Nigredo && \textbf{Menge} & 1W6 pro Ader \\ \cline{1-2} \cline{4-5} 
  	\textbf{Farbe} & schwarz (2) && \textbf{Vorkommen} & \brcell{Höhle (mittel) \\ Gebirge (selten)} \\ \cline{1-2} \cline{4-5} 
  	\textbf{Geruch} & asche (4) && \textbf{Region} & überall \\ \cline{1-2} \cline{4-5} 
  	\textbf{Geschmack} & asche (3) && \textbf{Wert} & 5Kr \\ \cline{1-2} \cline{4-5} 
  	\textbf{Spezialeigenschaften} & - && \textbf{Utensilien} & Spitzhacke \\ \cline{1-2} \cline{4-5} 
\end{tabular} 
\end{center} 
\caption{Kohle} 
\label{tab:kohle} 
\end{table}


\subsection{Königsstein}
\label{chap:koenigsstein}
Königsstein ist ein seltenes Mineral, das in Höhlen vorkommt. Es wird für die Herstellung von hochätzendem Herzogswasser hergestellt, ist aber selber harmlos. Zum Abbau von Königsstein wird eine Spitzhacke benötigt. Siehe Kapitel \ref{chap:abbau_von_mineralien} für den Abbau von Mineralien mit Spitzhacken.

\begin{table}[H] 
\begin{center} 
\begin{tabular}{|l|l|p{1cm}|l|l|} 
  	\cline{1-2} \cline{4-5} 
  	\textbf{Hauptwirkstoff} & Äther && \textbf{Pflückprobe} & (siehe Beschreibung) \\ \cline{1-2} \cline{4-5} 
  	\textbf{Sekundärwirkstoff} & - && \textbf{Menge} & 1W4 pro Ader \\ \cline{1-2} \cline{4-5} 
  	\textbf{Farbe} & blau (1) && \textbf{Vorkommen} & Höhle (selten) \\ \cline{1-2} \cline{4-5} 
  	\textbf{Geruch} & neutral && \textbf{Region} & überall \\ \cline{1-2} \cline{4-5} 
  	\textbf{Geschmack} & neutral && \textbf{Wert} & 30Kr \\ \cline{1-2} \cline{4-5} 
  	\textbf{Spezialeigenschaften} & - && \textbf{Utensilien} & Spitzhacke \\ \cline{1-2} \cline{4-5} 
\end{tabular} 
\end{center} 
\caption{Königsstein} 
\label{tab:koenigsstein} 
\end{table}


\subsection{Lunasplitter}
Luna Splitter bestehen aus silbrigem Metall, das im Dunkeln leuchtet. Sie wachsen in Höhlen an Wänden und Decken und müssen vorsichtig mit einer Spitzhacke ausgegraben werden. Beim Abbauen ist besondere Vorsicht geboten, da man die Splitter sonst beschädigen könnte. 

Zum Abbauen wird ein Utensil zum Ablösen der Splitter benötigt. Z.B. ein Messer, Handbeil oder Spitzhacke. Die \textit{Pflückprobe} muss auf jeden Fall bestanden werden. Anstatt auf \textit{Alchemie} (wie für die \textit{Pflückprobe} üblich) muss auf \textit{Motorische Talente} geworfen werden. Benutzt man ein Utensil, ist die \textit{Pflückprobe} um 2 QS erleichtert.

\begin{table}[H] 
\begin{center} 
\begin{tabular}{|l|l|p{1cm}|l|l|} 
  	\cline{1-2} \cline{4-5} 
  	\textbf{Hauptwirkstoff} & Rebis && \textbf{Pflückprobe} & QS III \\ \cline{1-2} \cline{4-5} 
  	\textbf{Sekundärwirkstoff} & Rubedo && \textbf{Menge} & 1 pro Splitter \\ \cline{1-2} \cline{4-5} 
  	\textbf{Farbe} & silber (1) && \textbf{Vorkommen} & Höhle (selten) \\ \cline{1-2} \cline{4-5} 
  	\textbf{Geruch} & neutral && \textbf{Region} & überall \\ \cline{1-2} \cline{4-5} 
  	\textbf{Geschmack} & neutral && \textbf{Wert} & 40Kr \\ \cline{1-2} \cline{4-5} 
  	\textbf{Spezialeigenschaften} & Leuchtend && \textbf{Utensilien} & \brcell{Messer \\ Handbeil \\ Spitzhacke \\ (siehe Beschreibung)} \\ \cline{1-2} \cline{4-5} 
\end{tabular} 
\end{center} 
\caption{Lunasplitter} 
\label{tab:lunasplitter} 
\end{table}


\subsection{Optima Mater}
Optima Mater ist ein Katalysator, der von Alchemisten hoch geschätzt wird. Es lohnt sich daher, diese Substanz zu verkaufen. Sie findet vor allem in hochwertigen Tränken Verwendung. 

Optima Mater enthält einige Wirkstoffe die einen rauschlindernden Effekt haben. Diese Wirkstoffe sind jedoch hoch konzentriert und sollten nur für spezielle rauschlindernde Elixiere verwendet werden. Pur können sie schwere Übelkeit auslösen.

Optima Mater sind bunte Gesteinsbrocken die in Höhlen zu finden sind. Manchmal sind sie auch festgewachsen. In diesem Fall müssen sie mit einem Hammer oder einer Spitzhacke abgebrochen werden. Dazu muss die \textit{Pflückprobe} auf \textit{Kraftakt} bestanden werden.

\begin{table}[H] 
\begin{center} 
\begin{tabular}{|l|l|p{1cm}|l|l|} 
  	\cline{1-2} \cline{4-5} 
  	\textbf{Hauptwirkstoff} & Quebrith && \textbf{Pflückprobe} & QS I \\ \cline{1-2} \cline{4-5} 
  	\textbf{Sekundärwirkstoff} & Rubedo && \textbf{Menge} & 1W4 pro Ader \\ \cline{1-2} \cline{4-5} 
  	\textbf{Farbe} & bunt (2) && \textbf{Vorkommen} & Höhle (selten) \\ \cline{1-2} \cline{4-5} 
  	\textbf{Geruch} & neutral && \textbf{Region} & überall \\ \cline{1-2} \cline{4-5} 
  	\textbf{Geschmack} & neutral && \textbf{Wert} & 70Kr \\ \cline{1-2} \cline{4-5} 
  	\textbf{Spezialeigenschaften} & \brcell{(siehe \\ Beschreibung)} && \textbf{Utensilien} & \brcell{Hammer \\ Spitzhacke \\ (siehe Beschreibung)} \\ \cline{1-2} \cline{4-5} 
\end{tabular} 
\end{center} 
\caption{Optima Mater} 
\label{tab:optima_mater} 
\end{table}


\subsection{Schwefel}
\label{chap:schwefel}
Schwefel kann aus schwefelhaltigen Verbindungen in Kohlenwasserstoffquellen wie Kohle gewonnen werden. An Vulkanen und in ihrer Nähe kommen Fumarolen vor, die gasförmigen elementaren Schwefel ausstoßen, der beim Abkühlen an der Austrittsstelle kondensiert und Kristalle bildet. Schwefel wird häufig verwendet um Schwarzpulver herzustellen.

Diese können mit z.B. einem Messer abgeschabt werden. Zur Kohlegewinnung siehe Kapitel \ref{chap:kohle}. Um aus Kohle Schwefel herzustellen muss eine Probe auf \textit{Alchemie} abgelegt und bestanden werden. Ohne Utensilien kann Schwefel bzw. Kohle nicht abgebaut werden.

\begin{table}[H] 
\begin{center} 
\begin{tabular}{|l|l|p{1cm}|l|l|} 
  	\cline{1-2} \cline{4-5} 
  	\textbf{Hauptwirkstoff} & Quebrith && \textbf{Pflückprobe} & (siehe Beschreibung) \\ \cline{1-2} \cline{4-5} 
  	\textbf{Sekundärwirkstoff} & - && \textbf{Menge} & 1 pro Kohle/Kristall \\ \cline{1-2} \cline{4-5} 
  	\textbf{Farbe} & gelb (1) && \textbf{Vorkommen} & Fumarolen (häufig) \\ \cline{1-2} \cline{4-5} 
  	\textbf{Geruch} & übelriechend (3) && \textbf{Region} & überall \\ \cline{1-2} \cline{4-5} 
  	\textbf{Geschmack} & neutral && \textbf{Wert} & 60Kr \\ \cline{1-2} \cline{4-5} 
  	\textbf{Spezialeigenschaften} & - && \textbf{Utensilien} & - \\ \cline{1-2} \cline{4-5} 
\end{tabular} 
\end{center} 
\caption{Schwefel} 
\label{tab:schwefel} 
\end{table}


\subsection{Phosphor}
Phosphor ist eine leicht entflammbare alchemische Substanz. In der Natur kommt Phosphor in Form der Phosphate in der Erdkruste vor. 

Zur Gewinnung muss Erde ausgegraben und untersucht werden. Dazu kann der SL, je nach Boden eine Probe auf \textit{Kraftakt} verlangen. Zur Untersuchung muss eine Probe auf \textit{Alchemie} bestanden werden. Gelingt die Probe ebenfalls entscheidet der Zufall, ob die Erde Phosphor enthält. Die Chance hängt dabei von dem Seltenheitsgrad, abhängig vom Ort, ab. Mittel = 33\% und Selten = 20\%. Die Untersuchung dauert ca. 5 Stunden. Allerdings können mehrere Erdproben parallel untersucht werden.

\begin{table}[H] 
\begin{center} 
\begin{tabular}{|l|l|p{1cm}|l|l|} 
  	\cline{1-2} \cline{4-5} 
  	\textbf{Hauptwirkstoff} & Karmin && \textbf{Pflückprobe} & (siehe Beschreibung) \\ \cline{1-2} \cline{4-5} 
  	\textbf{Sekundärwirkstoff} & - && \textbf{Menge} & 1 pro Erdprobe \\ \cline{1-2} \cline{4-5} 
  	\textbf{Farbe} & beige (1) && \textbf{Vorkommen} & \brcell{Gebirge (mittel) \\ Ödland (mittel) \\ Feld (selten) \\ Wald (selten)} \\ \cline{1-2} \cline{4-5} 
  	\textbf{Geruch} & neutral && \textbf{Region} & \brcell{Nördliche Köngreiche} \\ \cline{1-2} \cline{4-5} 
  	\textbf{Geschmack} & neutral && \textbf{Wert} & 60Kr \\ \cline{1-2} \cline{4-5} 
  	\textbf{Spezialeigenschaften} & Explosiv && \textbf{Utensilien} & Schaufel \\ \cline{1-2} \cline{4-5} 
\end{tabular} 
\end{center} 
\caption{Phosphor} 
\label{tab:phosphor} 
\end{table}


\subsection{Pulverisierte Perle}
Pulverisierte Perle ist eine alchimistische Substanz, die aus zerstoßenen weißen Perlen hergestellt wird. Große Perlen können auch genug Pulver für zwei Tränke hergeben ($\rightarrow$ Menge: 2 pro Perle).

\begin{table}[H] 
\begin{center} 
\begin{tabular}{|l|l|p{1cm}|l|l|} 
  	\cline{1-2} \cline{4-5} 
  	\textbf{Hauptwirkstoff} & Äther && \textbf{Pflückprobe} & - \\ \cline{1-2} \cline{4-5} 
  	\textbf{Sekundärwirkstoff} & Rubedo && \textbf{Menge} & 1 pro Perle \\ \cline{1-2} \cline{4-5} 
  	\textbf{Farbe} & transparent && \textbf{Vorkommen} & \brcell{Strand (mittel) \\ Meeresboden (mittel)} \\ \cline{1-2} \cline{4-5} 
  	\textbf{Geruch} & neutral && \textbf{Region} & überall \\ \cline{1-2} \cline{4-5} 
  	\textbf{Geschmack} & neutral && \textbf{Wert} & 40Kr \\ \cline{1-2} \cline{4-5} 
  	\textbf{Spezialeigenschaften} & - && \textbf{Utensilien} & Mörser und Stößel \\ \cline{1-2} \cline{4-5} 
\end{tabular} 
\end{center} 
\caption{Pulverisierte Perle} 
\label{tab:pulverisierte_perle} 
\end{table}


\subsection{Weinstein}
Weinstein ist eine Substanz, die Hexen aus Wein herstellen. Dazu muss ein magiebegabter eine Probe auf \textit{Magiekunde} bestehen. Je nach Erfahrungsgrad in der Zauberkunst kann der SL Erschwernisse oder Erleichterungen aussprechen. Der Prozess dauert ca. eine halbe Stunde.

Weinstein wird verwendet um Brei für medizinische Umschläge zu verwenden.

\begin{table}[H] 
\begin{center} 
\begin{tabular}{|l|l|p{1cm}|l|l|} 
  	\cline{1-2} \cline{4-5} 
  	\textbf{Hauptwirkstoff} & Rebis && \textbf{Pflückprobe} & (siehe Beschreibung) \\ \cline{1-2} \cline{4-5} 
  	\textbf{Sekundärwirkstoff} & - && \textbf{Menge} & 1 pro Wein \\ \cline{1-2} \cline{4-5} 
  	\textbf{Farbe} & rot (1) && \textbf{Vorkommen} & (siehe Beschreibung) \\ \cline{1-2} \cline{4-5} 
  	\textbf{Geruch} & wein (1) && \textbf{Region} & (siehe Beschreibung) \\ \cline{1-2} \cline{4-5} 
  	\textbf{Geschmack} & wein (1) && \textbf{Wert} & 30Kr \\ \cline{1-2} \cline{4-5} 
  	\textbf{Spezialeigenschaften} & Giftneutralisierend && \textbf{Utensilien} & - \\ \cline{1-2} \cline{4-5} 
\end{tabular} 
\end{center} 
\caption{Weinstein} 
\label{tab:weinstein} 
\end{table}


\subsection{Weißer Essig}
Weißer Essig ist ein alchemischer Katalysator, gewonnen aus den Kadavern von Ghulen, Alghulen, Zemetauren und Graveiren. 

Weißer Essig ist sehr instabil, weshalb er häufig für Bomben verwendet wird. Er muss mit vorsichtig eingesammelt und aufbewahrt werden, da er sonst explodieren kann. Außerdem sollten nicht zu viele Mengen auf einmal gelagert werden.

\begin{table}[H] 
\begin{center} 
\begin{tabular}{|l|l|p{1cm}|l|l|} 
  	\cline{1-2} \cline{4-5} 
  	\textbf{Hauptwirkstoff} & Vitriol && \textbf{Pflückprobe} & QS II \\ \cline{1-2} \cline{4-5} 
  	\textbf{Sekundärwirkstoff} & - && \textbf{Menge} & 1W6 pro Kadaver \\ \cline{1-2} \cline{4-5} 
  	\textbf{Farbe} & transparent && \textbf{Vorkommen} & (siehe Beschreibung) \\ \cline{1-2} \cline{4-5} 
  	\textbf{Geruch} & essig (2) && \textbf{Region} & überall \\ \cline{1-2} \cline{4-5} 
  	\textbf{Geschmack} & essig (2) && \textbf{Wert} & 40Kr \\ \cline{1-2} \cline{4-5} 
  	\textbf{Spezialeigenschaften} & Instabil && \textbf{Utensilien} & - \\ \cline{1-2} \cline{4-5} 
\end{tabular} 
\end{center} 
\caption{Weißer Essig} 
\label{tab:weisser_essig} 
\end{table}


\subsection{Quecksilber}
Ein natürlich vorkommendes Mineral. Kommt meist in festem Zustand in Zinn oder anderen Gesteinen vor. Kommt primär in vulkanischen Gebieten vor, aber auch an ehemaligen Vulkangebieten. Kann aber auch im flüssigen Zustand in Höhlen gefunden werden. D.h., es kann flüssig oder fest sein. In Höhlen kommt es nur selten vor (egal ob flüssig oder fest) und in vulkanischen Gebieten mittel-häufig.

Flüssiges Quecksilber kann durch gezielte und unbestimmte Suche gefunden und mit einem Gefäß aufgenommen werden. Festes Quecksilber muss jedoch mit einer Spitzhacke abgebaut werden. Siehe Kapitel \ref{chap:abbau_von_mineralien} über den Abbau von Mineralien.

Flüssiges Quecksilber sondert ab 24°C Dämpfe ab. Das Einatmen dieser Dämpfe kann im Extremfall zum Tode führen.

\begin{table}[H] 
\begin{center} 
\begin{tabular}{|l|l|p{1cm}|l|l|} 
  	\cline{1-2} \cline{4-5} 
  	\textbf{Hauptwirkstoff} & Äther && \textbf{Pflückprobe} & (siehe Beschreibung) \\ \cline{1-2} \cline{4-5} 
  	\textbf{Sekundärwirkstoff} & Nigredo && \textbf{Menge} & 1W4 pro Ader \\ \cline{1-2} \cline{4-5} 
  	\textbf{Farbe} & silber (2) && \textbf{Vorkommen} & Höhlen (selten) \\ \cline{1-2} \cline{4-5} 
  	\textbf{Geruch} & quecksilber (1) && \textbf{Region} & \brcell{Vulkane (mittel) \\ (siehe Beschreibung)} \\ \cline{1-2} \cline{4-5} 
  	\textbf{Geschmack} & quecksilber (1) && \textbf{Wert} & 50Kr \\ \cline{1-2} \cline{4-5} 
  	\textbf{Spezialeigenschaften} & Gift && \textbf{Utensilien} & Spitzhacke \\ \cline{1-2} \cline{4-5} 
\end{tabular} 
\end{center} 
\caption{Quecksilber} 
\label{tab:quecksilber} 
\end{table}