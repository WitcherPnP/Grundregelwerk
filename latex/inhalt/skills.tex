{\let\clearpage\relax\chapter{Magie}}
Fähigkeiten können nur von Magiebegabten eingesetzt werden. Megiebegabte werden auch "`Quelle"' genannt. Damit sie ihre Kräfte gezielt einsetzen können bedarf es jahrelanger Übung und einer entsprechenden Ausbildung. Um Zauber zu wirken muss eine Zauberformel aufgesagt und eine Geste gemacht werden. Die Geste wird häufig mit einem Magierstab oder Stock ausgeführt. Um überhaupt einen Zauber verwenden zu können, muss außerdem magische Energie von außen gesammelt werden. Wie viel Energie aus welchem Element geschöpft werden kann, hängt von der Erfahrung und Ausbildungsgrad des Zauberers ab.

\section{Erfahrungs- und Ausbildungsgrad}
\label{chap:ausbildungsgrad}
Wird bei einem Kind das Talent zur Magie entdeckt, wird es mit ziemlicher Wahrscheinlichkeit früher oder später eine Ausbildung zum Magier erhalten. Mädchen erhalten diese Ausbildung an der hoch angesehenen Akademie für Zauberinnen auf der Thanedd-Insel. Die Akademie für Zauberer ist in Ban Ard in Kaedwen. Kinder, die gerade erst mit der Ausbildung beginnen haben der Erfahrungs- bzw. Ausbildungsgrad \textit{Novize} bzw. \textit{Novizin}. Mit abgeschlossener Ausbildung oder bereits längerem Studium werden sie zu \textit{Adepten}. Durch weitere Ausbildung, z.B. indem sie Lehrlinge von erfahrenen Magiern werden, werden sie selbst zu \textit{erfahrenen Zauberern}. Nur durch weiteres Training und ein extrem ausgeprägtes Talent für Magie, kann der \textit{erfahrene Magier} zum \textit{Meister} werden.

Es gibt auch Magier, die keine Ausbildung an einer Akademie erhalten haben. Dabei handelt es sich häufig um Priester oder Druiden. Sie beherrschen die Energieschöpfung, indem sie durch Meditation in eine Art Trance versetzt werden. Sie können (schwere) Zauber also nur durch das Meditieren wirken. 

Aber nun wieder zurück zu den Akademikern. Je nach Erfahrungs- bzw. Ausbildungsgrad erhält der Spieler bestimmte Vorzüge.

\begin{table}[H]
\begin{center}
\begin{tabular}{|l|l|l|l|l|}
\hline
\textbf{Ausbildungsgrad} & \textbf{Mana} & \textbf{SK} & \brcell{\textbf{Magiepunkte}\\\textbf{(max. Zauber)}} & \brcell{\textbf{Erleichterung}\\\textbf{beim Schöpfen}} \\ \hline
\textbf{Novize} & 12 & 0 & 12 (6) & 0 \\ \hline
\textbf{Adept} & 17 & 1 & 20 (9) & +1 \\ \hline
\textbf{Erfahren} & 22 & 2 & 29 (12) & +2 \\ \hline
\textbf{Meister} & 50+ & 3 & 39 (16) & +3 \\ \hline
\end{tabular}
\end{center}
\caption{Magier-Attribute abhängig durch den Erfahrungs- bzw. Ausbildungsgrad}
\label{tab:magier-attribute}
\end{table}

Die Abbildung \ref{tab:magier-attribute} gibt die Seelenkraft des Spielers in Abhängigkeit des Ausbildungsgrades an. Der Wert in der Spalte \textit{Magiepunkte (max. Zauber)}, der nicht in Klammern steht gibt an, wie viele Punkte der Spieler zur Verfügung hat um Fähigkeiten zu erwerben (siehe Kapitel \ref{chap:magiebegabte_klassen} zur Charaktererstellung von Magiern). Spieler können unter Umständen weitere Fähigkeiten erlernen. Die maximale Anzahl an Zaubersprüchen (die Zahl in Klammern) darf jedoch nicht überschritten werden. Die \textit{Mana}-Spalte gibt an, wie viel Energie maximal geschöpft werden kann, bevor es zu einer magischen Energieüberlastung kommt.

Ein Meister ist in der Lage aus jedem Element Energie zu ziehen, ohne eine Probe auf \textit{Magiekunde} ablegen zu müssen. Jedoch kann er nicht mehr Energie aus einem Element pro Aktion schöpfen als in Tabelle \ref{tab:energie_schoepfen} angegeben.

\section{Magische Energie}
Die Natur wird von zu jeder Zeit und an jedem Ort von magischer Energie durchlaufen. Genauer gesagt befindet sie sich in den Elementen Feuer, Erde, Wasser und Luft. Um einen Zauber zu wirken muss genügend Energie aus einem der Elemente gesammelt werden. Dies müssen junge Zauberer (Novizen und Adepten) als erstes lernen. Je nach Element kann das äußerst gefährlich sein und in Extremfällen sogar zum Tode führen. Aus diesem Grund können bzw. sollten nur erfahrene Zauberer Energie aus Feuer schöpfen. Grundsätzlich können Zauber auch ohne geschöpfte Magie gewirkt werden. In diesem Fall zerrt die Ausführung allerdings an der körperlichen Gesundheit des Anwenders. Das kann zu einem Schock, Schwindel, Ohnmacht oder sogar zum Tode führen. Das selbe kann auch passieren, wenn zu viel Energie auf einmal geschöpft wird. \textit{Meister} sind in der Lage beliebig viel Energie aus einer beliebigen Quelle bzw. beliebigem Element zu schöpfen. Sie müssen keine Probe auf \textit{Magiekunde} ablegen.

\subsection{Wasser}
Energie aus Wasser zu schöpfen ist relativ einfach. Aus diesem Grund lernen Novizen es als erstes. Aus ruhigen Gewässern kann leichter Energie gesammelt werden als an schnellen Strömungen oder auf offenem Meer.

\subsection{Erde}
Energie aus Erde zu schöpfen ist zwar schwer aber ungefährlich. Da Erde die Energie nicht gut leitet wird von statischer Energie gesprochen. Es ist sehr mühselig Energie aus der Erde zu schöpfen. Nur sehr \textit{erfahrene Magier} oder \textit{Meister} können aus der Erde problemlos genug Energie schöpfen um mächtige Zauber zu wirken.

\subsection{Luft}
Aus der Luft kann viel Energie geschöpft werden. Aufgrund seiner unbeständigen Natur verlangt dieses Element großes Können und Wissen. Unerfahrene Zauberer werden kaum in der Lage sein eine zufriedenstellende Menge von Energie zu absorbieren.

\subsection{Feuer}
Die Kraft des Feuers ist ebenso gewaltig wie unstet und sie lässt sich leicht und schnell beschwören. Das liegt daran, dass man die Energie, die durch das Feuer kanalisiert wird, relativ mühelos "`greifen"' kann und ein grüner Adept zehrt schon mal zu heftig oder zu gierig daran. Darum müssen Quellen ganz besonders vorsichtig sein, denn gelangen sie mit den Energien des Feuers in Kontakt, kann das ihre speziellen Fähigkeiten aktivieren, meist auf äußerst destruktive Weise.


\section{Orte der Macht}
Ein Ort der Macht ist ein Punkt, an dem die magische Energie zusammenläuft und besonders präsent ist. Häufig werden dort Riten abgehalten oder Kultstätten errichtet. Außerdem werden Drachen und Katzen von solchen Orten der Macht angelockt. Andere Tiere wiederum meiden sie. An solchen Orten ist das Schöpfen von Energie um 2 erleichtert.

\section{Priester und Druiden}
Manche Priester und Druiden können genauso mit Magie umgehen, wie ausgebildete Magier. Indem sie beten sind sie in der Lage die Kraft ebenso zu manipulieren wie Zauberer. Und das ohne jedes formelle Training, ohne Studium und ohne Vorbereitung. Manche Zauberer versuchen das Phänomen zu erklären. Sie kommen zu  dem Schluss, dass Priester durch Gebete in eine Art Trance verfallen oder sich selbst hypnotisieren und so ihre Fähigkeit aktivieren, die Energie zu lenken und zu nutzen wie es ausgebildete Zauberer tun.

Priester und Novizen müssen regelmäßig beten um Zauber ausführen zu können. \textit{Beten} dauert 4 Aktionen. Seine maximale Energie (der Standardwert) wird um 1+1W4 erhöht. Die Erhöhung bezieht sich immer auf den Standardwert, auch wenn dieser noch nicht erreicht ist. Sinken kann der aktuelle Wert jedoch nicht. Wenn die maximale Energie auf dem Standardwert ist, kann kein Zauber gewirkt werden. Jede Stunde sinkt die maximale Energie um 1 bis der Standardwert erreicht ist.

Der SL darf den Wurf, zur temporären Erhöhung der maximalen Energie, verbessern oder verschlechtern, je nachdem wo der Held betet. Es macht also einen Unterschied ob er an einer Kathedrale des eigenen Kultes oder in einer Taverne betet. Außerdem ist zum Beten zwingend Ruhe nötig.


\section{Zauberaktionen}
Wenn der Spieler einen Zauber ausführen will, muss er eine fest vorgegebene Reihenfolge einhalten.

\begin{enumerate}
\item \textbf{Ziel benennen} - siehe Kapitel \ref{chap:ziel_benennen}
\item \textbf{Modifikatoren auswählen} - siehe Kapitel \ref{chap:zauber_modifikationen}
\item \textbf{Energiekosten festlegen} - Berechnen der benötigten Mana/Energie
\item \textbf{Energiequelle benennen} - siehe Kapitel \ref{chap:energiequelle_benennen}
\item \textbf{Energie schöpfen} - siehe Kapitel \ref{chap:energie_schoepfen}
\item \textbf{Zauberprobe ablegen} - siehe Kapitel \ref{chap:magieprobe}
\item \textbf{Wirkung des Zaubers ausführen} - bei erfolgreicher Zauberprobe
\end{enumerate}

\subsection{Ziel benennen}
\label{chap:ziel_benennen}
Der Spieler wählt als erstes das Ziel für seinen Zauber. Ist das Ziel nicht mehr gültig, z.B. weil es außer Reichweite geht oder stirbt, während der Zauber gewirkt wird, darf der Spieler ein neues gültiges Ziel wählen.

\subsection{Modifikationen}
\label{chap:zauber_modifikationen}
Zauber können weiterhin modifiziert werde, um die Zauberzeit/-dauer, Zauberwirkung, Reichweite oder Wirkungsradius zu verändern. Für eine Modifikation (das heißt, eine Stufe in einer Modifikation) ist ein KL/IN-Wert von 13 nötig, für die zweite 14, für die dritte 15, und so weiter. Bei Magiern, zählt die Klugheit, bei Priestern und Druiden die Intuition. Modifikationen sind nur bei Zaubern möglich, deren Spezifikation der Zaubernde beherrscht.

\subsubsection{Zauberdauer}
\begin{table}[H]
\begin{center}
\begin{tabular}{|l|l|l|l|l|}
\hline
\textbf{Zauberzeit} & x2 & x1 & x$\frac{1}{2}$ & x$\frac{1}{4}$ \\ \hline
\textbf{Erschwernis} & -2 & 0 & +2 & +4 \\ \hline

\end{tabular}
\end{center}
\caption{Modifikator: Zauberdauer}
\label{tab:modifikator_zauberdauer}
\end{table}
Die Zauberzeit sinkt nicht unter eine Aktion. Die Veränderung der Zauberzeit erfolgt nach allen anderen Modifikationen. Die Zauberzeit darf nur einmal verdoppelt werden.


\subsubsection{Zauberwirkung erzwingen}
\begin{table}[H]
\begin{center}
\begin{tabular}{|l|l|l|l|l|l|l|}
\hline
\textbf{QS Erhöhung} & +1 & +2 & +3 & +4 & +5 & +6 \\ \hline
\textbf{Energiekosten (MP)} & +2 & +4 & +6 & +8 & +10 & +12 \\ \hline
\textbf{Zauberdauer} & +1 & +2 & +3 & +4 & +5 & +6 \\ \hline

\end{tabular}
\end{center}
\caption{Modifikator: Zauberwirkung}
\label{tab:modifikator_zauberwirkung}
\end{table}
Misslingt der Zauber, entstehen chaotische Effekte und das Opfer bemerkt den Zauber.


\subsubsection{Verändern der Reichweite oder des Wirkungsradius}
\begin{table}[H]
\begin{center}
\begin{tabular}{|l|l|l|l|l|l|l|l|l|l|}
\hline
\textbf{Strecke (in Meter)} & Selbst & Berührung & 1 & 3 & 7 & 21 & 49 & Horizont & Außer Sicht \\ \hline
\brcell{\textbf{Erschwernis ab}\\\textbf{Reichweite Selbst}} & 0 & 1 & 2 & 4 & 6 & 8 & 10 & 12 & 14 \\ \hline

\end{tabular}
\end{center}
\caption{Modifikator: Reichweite oder Wirkungsradius}
\label{tab:modifikator_reichweite_oder_wirkungsradius}
\end{table}
Die endgültige Erschwernis berechnet sich aus der positiven Differenz der angegeben Erschwernisse bei ursprünglicher und neuer Strecke. Das Verringern des Wirkungsradius erhöht die Zauberdauer um eine Aktion pro Reichweiten-Stufe.


\subsection{Benennung der Energiequelle}
\label{chap:energiequelle_benennen}
Jetzt muss die Quelle bzw. das Element genannt werden, aus der magische Energie geschöpft werden soll um den Zauber zu wirken. Die Spieler kann eines der Elemente Feuer, Wasser, Erde oder Luft in unmittelbarer Umgebung wählen.

\subsection{Energie schöpfen}
\label{chap:energie_schoepfen}
Um Energie aus der gewählten Quelle zu schöpfen, muss der Spieler eine Probe auf \textit{Magiekunde} bestehen. Diese Probe kann erschwert oder erleichtert werden. Je nach Ausbildungsgrad und gewähltem Element. Dieser Schritt dauert eine Aktion. Auch wenn das Schöpfen von Energie separat ausgewürfelt wird, geschieht es parallel zum Aufsagen des Zauberspruches. D.h., dass die Zauberzeit des Zaubers dadurch nicht verlängert wird. Dieser Schritt kann beliebig oft wiederholt werden, um  Ist die Zauberzeit abgelaufen, bevor genügend Energie (Mana) geschöpft wurde, wird das fehlende Mana von den LP abgezogen. Sobald zu viel Energie geschöpft wird, muss in der Tabelle \ref{tab:energieueberlastung} die Auswirkung nachgeschaut werden. Die Wirkung hängt dabei von der Menge an Energie ab, die über der maximalen Energie des Helden liegt.

\begin{table}[H]
\begin{center}
\begin{tabular}{|l|l|}
\hline
\textbf{Menge Energie über Maximum (n)} & \textbf{Auswirkung} \\ \hline
\textbf{1-6} & Schock (Paralyse +1 für nAK, n-2 Schaden) \\ \hline
\textbf{7-15} & Ohnmacht (Betäubung IV, n-13 Schaden) \\ \hline
\textbf{16+} & Tod \\ \hline

\end{tabular}
\end{center}
\caption{Folgen von Energieüberlastung}
\label{tab:energieueberlastung}
\end{table}

Schaden den man durch eine magische Überbelastung bekommt kann nicht negativ werden. Es ist dadurch also keine Heilung möglich. Die Dauer des Schocks hängt außerdem von der stärke der Überlastung ab.

Die zweite bis fünfte Spalte in Tabelle \ref{tab:energie_schoepfen} gibt den Ausbildungsgrad des Zauberers an. Die Zahl hinter dem Ausbildungsgrad gibt an, um wie viel \textit{Magiekunde}-Proben zum Schöpfen von Energie erleichtert sind. Die Werte, die nicht in Klammern stehen geben die \textbf{maximale Menge Energie (max.M)} an, die pro Aktion aus dem Element geschöpft werden kann. \textbf{Zahlen in Klammern} repräsentieren eine Probe auf \textit{Magiekunde}, die um die entsprechende Zahl vereinfacht (+) bzw. erschwert (-) ist. 

\textbf{Runde Klammern (RK)} bedeuten, dass die Probe nur abgelegt werden muss, wenn versucht wird weniger Energie zu schöpfen als maximal möglich. Die Zahl gibt die \textbf{Erschwernis} (oder Erleichterung) \textbf{pro Energie} (EpE) die weniger geschöpft werden soll an. Beispiel: Möchte ein \textit{Adept} 2 Energie aus wilden Gewässern schöpfen, ist die Probe um -1 (siehe runde Klammer) * 2 (4-2) = -2 erschwert. Misslingt die Probe schlägt der Zauber oder das Schöpfen nicht fehl. Stattdessen wird mehr Energie geschöpft als gewollt. +1M pro Punkt mit dem die Probe fehlschlug. Dabei kann es zu einer Überlastung kommen. Kommt es zu keiner Überlastung, zählt die Probe als Erfolg.

\textbf{Eckige Klammern (EK)} bedeuten, dass die Probe abgelegt werden muss, damit überhaupt Energie geschöpft werden kann. Die \textbf{linke Zahl (EKL)} gibt wie bei runden Klammern auch eine Erleichterung bzw. Erschwernis an. Für jede Energie, die weniger als das Maximum geschöpft werden soll, erschwert sich die Probe zusätzlich um die \textbf{rechte Zahl (EKR)}. Misslingt die Probe passiert das selbe wie mit Runden Klammern.

Wenn der Zauberer es schafft genau so viel Energie zu schöpfen, wie der Zauber benötigt, erhöht sich die QS des Zaubers um 2. Weicht die Menge Energie um 1 ab, erhöht sich die QS des Zaubers um 1.

\begin{table}[H]
\begin{center}
\begin{tabular}{|l|l|l|l|}
\hline
\textbf{Element} & \textbf{Novize 0} & \textbf{Adept +1} & \textbf{Erfahren +2} \\ \hline
\textbf{Wasser (ruhig)} & 4 (-1) & 4 (-1) & 5 (-1) \\ \hline
\textbf{Wasser (wild)} & 4 (-1) & 4 (-1) & 5 (-1) \\ \hline
\textbf{Luft (windstill)} & 5 [-2/-1] & 5 [-2/-1] & 6 (-1) \\ \hline
\textbf{Luft (windig)} & 5 [-3/-1] & 5 [-3/-1] & 6 (-1) \\ \hline
\textbf{Luft (stürmisch)} & 5 [-4/-2] & 5 [-4/-1] & 6 (-2) \\ \hline
\textbf{Erde} & 6 [-6/-2] & 6 [-5/-1] & 7 [-4/-1] \\ \hline
\textbf{Feuer} & 9 [-5/-2] & 9 [-4/-2] & 10 [-3/-2] \\ \hline

\end{tabular}
\end{center}
\caption{Energie schöpfen}
\label{tab:energie_schoepfen}
\end{table}

Aus der Tabelle \ref{tab:energie_schoepfen} können folgende Formeln zur Berechnung der benötigten Aktionen und der Erschwernisse für Proben abgeleitet werden. Die Formeln hängen immer davon ab, wie viel \textbf{Energie (M)} der Zauber benötigt.

\textbf{Anzahl Aktionen = M / max.M (aufgerundet)} 

\textbf{Probe (bei runden Klammern) = (Anzahl Aktionen * max.M - M) * EpE + Ausbildungsgrad-Bonus} 
(Wenn das Ergebnis 0 oder höher ist, ist keine Probe nötig.) 

\textbf{Probe (bei eckigen Klammern) = (Anzahl Aktionen * max.M - M) * EKR + EKL + Ausbildungsgrad-Bonus} \\
(Probe muss immer durchgeführt werden) 

\subsection{Magieprobe}
\label{chap:magieprobe}
Für den eigentlichen Zauber muss eine Probe abgelegt werden. Das funktioniert genau so wie bei einem Talent. Jede Fähigkeit basiert auf drei Grundattributen. Die Wirkung mancher Zauber werden durch die erreichte QS verändert.

\section{Spezialisierungen}
\label{chap:spezialisierungen}
Ein Magier kann sich je nach Erfahrung auf eine oder mehrere Zauberkategorien spezialisieren. Zauber der eigenen Spezialisierung bekommen einen erhöhten Fertigkeitswert (FW), abhängig von der Erfahrung.

\subsection{Dämonisch}
\begin{longtable}{|p{5cm}|p{10cm}|}
\hline
\textbf{Zauber} & \textbf{Kurzbeschreibung} \\ \hline

\textbf{Ecliptifactus} & Dämonenbeschwörung (sein eigener Schatten). Kapitel \ref{chap:ecliptifactus} \\ \hline

\textbf{Herzschlag ruhe} & Lässt Herzschlag eines Gegners aussetzen. Kapitel \ref{chap:herzschlag_ruhe} \\ \hline

\textbf{Krabbelnder Schrecken} & Bedeckt Gegner mit Insekten/Kleintieren. Kapitel \ref{chap:krabbelnder_schrecken} \\ \hline

\caption{Dämonenzauber}
\label{tab:daemonenzauber}
\end{longtable}


\subsection{Einfluss}
\begin{longtable}{|p{5cm}|p{10cm}|}
\hline
\textbf{Zauber} & \textbf{Kurzbeschreibung} \\ \hline

\textbf{Bannbaladin} & Freundschaftszauber. Kapitel \ref{chap:bannbaladin} \\ \hline

\textbf{Blitz dich find} & Der Verzauberte wird geblendet. Kapitel \ref{chap:blitz_dich_find} \\ \hline

\textbf{Hilfreiche Pfote} & Beeinflusst ein kleines Säugetier. Kapitel \ref{chap:hilfreiche_pfote} \\ \hline

\textbf{Imperavi} & Jemanden manipulieren, damit er eine Sache für der Zaubernden macht. Kapitel \ref{chap:imperavi} \\ \hline

\caption{Einflusszauber}
\label{tab:einflusszauber}
\end{longtable}


\subsection{Elementar}
\begin{longtable}{|p{5cm}|p{10cm}|}
\hline
\textbf{Zauber} & \textbf{Kurzbeschreibung} \\ \hline

\textbf{Aeolito} & Stürmischer Wind der Schaden verursacht. Kapitel \ref{chap:aeolito} \\ \hline

\textbf{Ignifaxius} & Feuer aus dem Finger. Kapitel 
\ref{chap:ignifaxius} \\ \hline

\textbf{Ignisphaero} & Feuerball. Kapitel \ref{chap:ignisphaero} \\ \hline

\caption{Elementarzauber}
\label{tab:elementarzauber}
\end{longtable}



\subsection{Heilung}

\begin{longtable}{|p{5cm}|p{10cm}|}
\hline
\textbf{Zauber} & \textbf{Kurzbeschreibung} \\ \hline

\textbf{Armatrutz} & Erhöht RS. Kapitel \ref{chap:armatrutz} \\ \hline

\textbf{Balsam Salabunde} & Heilzauber. Kapitel \ref{chap:balsam_salabunde} \\ \hline

\textbf{Lunge des Leviatan} & Erhöht Ausdauer. Kapitel \ref{chap:lunge_des_leviatan} \\ \hline

\textbf{Klarum Purum} & Heilt Vergiftung. Kapitel \ref{chap:klarum_purum} \\ \hline

\textbf{Regeneratio} & Heilung über Zeit. Kapitel \ref{chap:regeneratio} \\ \hline

\caption{Heilungszauber}
\label{tab:heilungszauber}
\end{longtable}


\subsection{Illusion}
\begin{longtable}{|p{5cm}|p{10cm}|}
\hline
\textbf{Zauber} & \textbf{Kurzbeschreibung} \\ \hline

\textbf{Duplicatus} & Doppelgänger erschaffen. Kapitel \ref{chap:duplicatus} \\ \hline

\textbf{Manus Illusionis} & Täuscht den Tastsinn. Kapitel \ref{chap:manus_illusionis} \\ \hline

\textbf{Menetekel} & Lässt Text/Schriftzeichen erscheinen. Kapitel \ref{chap:menetekel} \\ \hline

\textbf{Penetrizzel} & Durch Wände blicken. Kapitel \ref{chap:penetrizzel} \\ \hline

\textbf{Projectimago} & Erschafft Projektion seiner selbst an einem weit entfernten Ort. Kapitel \ref{chap:projectimago} \\ \hline

\textbf{Vogelzwitschern} & Geräusche austauschen. Kapitel \ref{chap:vogelzwitschern} \\ \hline

\caption{Illusionszauber}
\label{tab:illusionszauber}
\end{longtable}


\subsection{Objekt}
\begin{longtable}{|p{5cm}|p{10cm}|}
\hline
\textbf{Zauber} & \textbf{Kurzbeschreibung} \\ \hline

\textbf{Claudibus} & Schlösser schließen/versperren. Kapitel \ref{chap:claudibus} \\ \hline

\textbf{Desintegratus} & Zerstört Objekte. Kapitel \ref{chap:desintegratus} \\ \hline

\textbf{Foramen} & Schlösser öffnen. Kapitel \ref{chap:foramen} \\ \hline

\textbf{Objectobscuro} & Unsichtbarkeitszauber für Objekte. Kapitel \ref{chap:objectobscuro} \\ \hline

\textbf{Objectovoco} & Objekt etwas fragen. Kapitel \ref{chap:objectovoco} \\ \hline

\textbf{Runenzauber} & Ermöglicht die Verwendung von Runen. Kapitel \ref{chap:runenzauber} \\ \hline

\textbf{Sumus Elixiere} & Verbessert Alchemieskills. Verlängert u.A. die Haltbarkeit von Zutaten und Lebensmitteln. Kapitel \ref{chap:sumus_elixiere} \\ \hline

\caption{Objektzauber}
\label{tab:objektzauber}
\end{longtable}


\subsection{Psychokinetik}
\begin{longtable}{|p{5cm}|p{10cm}|}
\hline
\textbf{Zauber} & \textbf{Kurzbeschreibung} \\ \hline

\textbf{Fortifex} & Errichtet undurchdringliche Wand. Kapitel \ref{chap:fortifex} \\ \hline

\textbf{Motoricus} & Kann Gegenstände bewegen. Kapitel \ref{chap:motoricus} \\ \hline

\textbf{Radau} & Lässt Stock für sich kämpfen. Kapitel \ref{chap:radau} \\ \hline

\textbf{Silentium} & Erzeugt Aura in der man nichts hört. Kapitel \ref{chap:silentium} \\ \hline

\textbf{Solidirid} & Erzeugt Brücke aus Licht. Kapitel \ref{chap:solidirid} \\ \hline

\caption{Psychokinetische-Zauber}
\label{tab:psychokinetische-zauber}
\end{longtable}


\subsection{Sphären}
\begin{longtable}{|p{5cm}|p{10cm}|}
\hline
\textbf{Zauber} & \textbf{Kurzbeschreibung} \\ \hline

\textbf{Aen evall} & Ruft ein Pferd aus der Welt der Aen Seidhe. Kapitel \ref{chap:aen_evall} \\ \hline

\textbf{Nuntiovolo} & Beschwört Dämonenbote. Kapitel \ref{chap:nuntiovolo} \\ \hline

\textbf{Transversalis} & Teleportation. Kapitel \ref{chap:transversalis} \\ \hline

\caption{Sphärenzauber}
\label{tab:sphaerenzauber}
\end{longtable}


\subsection{Verwandlung}
\begin{longtable}{|p{5cm}|p{10cm}|}
\hline
\textbf{Zauber} & \textbf{Kurzbeschreibung} \\ \hline

\textbf{Adlerschwinge} & Der Zauberer verwandelt sich in ein kleines Fluglebewesen seiner Wahl. Kapitel \ref{chap:adlerschwinge} \\ \hline

\textbf{Fulminictus} & Der Verzauberte erleidet innere Verletzungen. Kapitel \ref{chap:fulminictus} \\ \hline

\textbf{Katzenaugen} & Ermöglicht das Sehen im Dunkeln. Kapitel \ref{chap:katzenaugen} \\ \hline

\textbf{Kraft des Tieres} & Erhöht eigene oder fremde Körperkraft. Kapitel \ref{chap:kraft_des_tieres} \\ \hline

\textbf{Paralysis} & Paralysiert Gegner. Kapitel \ref{chap:paralysis} \\ \hline

\caption{Verwandlungszauber}
\label{tab:verwandlungszauber}
\end{longtable}


\subsection{Wahrsagerei}
\begin{longtable}{|p{5cm}|p{10cm}|}
\hline
\textbf{Zauber} & \textbf{Kurzbeschreibung} \\ \hline

\textbf{Blick in die Gedanken} & Gedanken lesen. Kapitel \ref{chap:blick_in_die_gedanken} \\ \hline

\textbf{Exposami} & Der Zauberer kann Lebewesen in der Nähe spüren. Kapitel \ref{chap:exposami} \\ \hline

\textbf{Gefunden} & Objekte wiederfinden. Kapitel \ref{chap:gefunden} \\ \hline

\textbf{Sensibar} & Gefühle lesen. Kapitel \ref{chap:sensibar} \\ \hline

\caption{Wahrsagerei-Zauber}
\label{tab:wahrsagerei-zauber}
\end{longtable}


\section{Fähigkeiten}

\subsection{Adlerschwinge}
\label{chap:adlerschwinge}
\textbf{Probe (MTW):} MU/IN/GE \\
\textbf{Wirkung:} Der Zauberer verwandelt sich in ein Fluglebewesen seiner Wahl, das jedoch maximal die Größenkategorie klein aufweisen darf. Während der Wirkungsdauer verwendet der Zauberer die körperlichen Werte des Tieres, einschließlich seiner Eigenschaften, Talente und Kampfwerte. Niedrigere LP werden bei der Verwandlung anteilsmäßig angerechnet, gleiches geschieht bei der Rückverwandlung. Die geistigen Werte des Zauberers bleiben ebenso wie sein Bewusstsein erhalten. Die Verwandlung betrifft nur den Zauberer selbst, nicht seine Kleidung oder Ausrüstung. Es ist keine Verwandlung in ein kulturschaffendes Wesen möglich.\\
\textbf{Zauberdauer (ZD/ZZ):} 8 Aktionen \\
\textbf{Mana-Kosten (MK):} 8 MP (Aktivierung des Zaubers) + 4 MP pro Stunde \\
\textbf{Reichweite (RW):} selbst \\
\textbf{Wirkungsdauer (WD):} aufrechterhaltend \\
\textbf{Zielkategorie:} Kulturschaffende \\
\textbf{Spezialisierung:} Verwandlung \\
\textbf{Steigungsfaktor (SF):} C


\subsection{Aen evall}
\label{chap:aen_evall}
\textbf{Probe (MTW):} KL/IN/CH \\
\textbf{Wirkung:} Neben dem Zauberer manifestiert sich ein Zauberpferd von der Welt der Aen Elle. Das Tier steht zwar nicht unter magischer Kontrolle, ist aber zutraulich und kann ganz normal geritten oder für andere Zwecke eingesetzt werden. Das Pferd wird vom Zauberer nicht erschaffen, sondern an seine Seite teleportiert. Nach QS Stunden wird das Tier wieder an seinen Ursprungsort transportiert, selbst wenn es in der Zwischenzeit verstorben ist. Gleiches gilt für eventuell abgetrennte oder entnommene Körperteile. Sollten sich am Ende der Wirkungsdauer noch Reiter oder Ausrüstung auf dem Pferd befinden, fallen diese zu Boden. Der Zauberer ruft kein bestimmtes Pferd von der Parallelwelt, sondern ein zufällig bestimmtes.\\
\textbf{Zauberdauer (ZD/ZZ):} 16 Aktionen \\
\textbf{Mana-Kosten (MK):} 16 MP \\
\textbf{Reichweite (RW):} 1 Meter \\
\textbf{Wirkungsdauer (WD):} sofort \\
\textbf{Zielkategorie:} Lebewesen (Zauberpferd) \\
\textbf{Spezialisierung:} Sphären \\
\textbf{Steigungsfaktor (SF):} B


\subsection{Aeolito}
\label{chap:aeolito}
\textbf{Probe (MTW):} KL/CH/KO (modifiziert um ZK) \\
\textbf{Wirkung:} Der Aeolito erschafft einen stürmischen Wind, der beim Ziel je nach QS Schaden und den Status Liegend verursacht.\\
QS 1: 1W6 TP\\
QS 2: 1W6+2 TP\\
QS 3: 1W6+2 TP, Liegend\\
QS 4: 1W6+4 TP, Liegend, 1 Meter zurückgeworfen\\
QS 5: 1W6+6 TP, Liegend, 2 Meter zurückgeworfen\\
QS 6: 1W6+6 TP, Liegend, 3 Meter zurückgeworfen\\
Rüstungen schützen nicht vor diesem Zauber. Der Zauberspruch trifft sein Ziel automatisch. Der Gegner kann weder ausweichen noch blocken.
 \\
\textbf{Zauberdauer (ZD/ZZ):} 1 Aktion \\
\textbf{Mana-Kosten (MK):} 8 MP (Kosten sind nicht modifizierbar) \\
\textbf{Reichweite (RW):} 16 Meter \\
\textbf{Wirkungsdauer (WD):} sofort \\
\textbf{Zielkategorie:} Lebewesen \\
\textbf{Spezialisierung:} Elementar \\
\textbf{Steigungsfaktor (SF):} B


\subsection{Armatrutz}
\label{chap:armatrutz}
\textbf{Probe (MTW):} KL/IN/FF \\
\textbf{Wirkung:} Die Haut des Zaubernden verhärtet sich, ohne ihre Flexibilität zu verlieren. So erhält er einen natürlichen Rüstungsschutz, der sich auf getragene Rüstung aufaddiert, ohne die Belastung zu erhöhen. Die Höhe des gewünschten zusätzlichen Rüstungsschutzes muss der Zaubernde vor der Probe festlegen. Maximal ist ein zusätzlicher RS von 3 möglich. \\
\textbf{Zauberdauer (ZD/ZZ):} 1 Aktion \\
\textbf{Mana-Kosten (MK):} 4 MP für RS 1, 8 MP für RS 2, 16 MP für RS 3 (Kosten sind nicht modifizierbar) \\
\textbf{Reichweite (RW):} selbst \\
\textbf{Wirkungsdauer (WD):} QS x 3 Minuten \\
\textbf{Zielkategorie:} Wesen \\
\textbf{Spezialisierung:} Heilung \\
\textbf{Steigungsfaktor (SF):} C


\subsection{Balsam Salabunde}
\label{chap:balsam_salabunde}
\textbf{Probe (MTW):} KL/IN/FF \\
\textbf{Wirkung:} Der Verzauberte erhält innerhalb von 6 Minuten nach dem Wirken des Zaubers verlorene LP in Höhe der verwendeten MP zurück. Der Zauberer kann maximal so viele MP einsetzen, wie er FW hat. Für jede QS sinkt die Zeit um 1 Minute. Wird der Zauber vor dem Ablauf der durch den Konstitutionswert angegebenen Frist für den Tod eines Helden begonnen, kann er gerettet werden. Wird der Zauber jedoch unterbrochen, überlebt der Patient danach nur noch die verbliebenen Kampfrunden. \\
\textbf{Zauberdauer (ZD/ZZ):} 16 Aktionen \\
\textbf{Mana-Kosten (MK):} 1 MP pro LP, mindestens jedoch 4 MP (Kosten sind nicht modifizierbar) \\
\textbf{Reichweite (RW):} Berührung \\
\textbf{Wirkungsdauer (WD):} sofort \\
\textbf{Zielkategorie:} Kulturschaffende \\
\textbf{Spezialisierung:} Heilung \\
\textbf{Steigungsfaktor (SF):} B


\subsection{Bannbaladin}
\label{chap:bannbaladin}
\textbf{Probe (MTW):} MU/IN/CH (modifiziert um SK) \\
\textbf{Wirkung:} Dieser ursprünglich elfische Freundschaftszauber weckt in seinem Ziel Sympathien und Freundschaft für den Zaubernden. Die genaue Wirkung ist davon abhängig, wie das Ziel zu dem Zaubernden steht. Pro QS verbessert sich die Verbindung zwischen den beiden um eine Stufe. Ist der Zaubernde dem Ziel vorher unbekannt, bewegt sich die Einstellung des Ziels meist zwischen Stufe 4 (Abneigung) und Stufe 6 (Zuneigung). Vorteile wie Gutaussehend oder Nachteile wie Vorurteile gegenüber dem Zaubernden können diesen ersten Eindruck je nach Situation um eine oder mehr Stufen modifizieren.\\
Stufe 1: Das Ziel steht dem Zaubernden voller unstillbarem Hass gegenüber.\\
Stufe 2: Das Ziel ist der erklärte Feind des Zaubernden.\\
Stufe 3: Das Ziel verspürt dem Zaubernden gegenüber Abscheu und Ablehnung.\\
Stufe 4: Das Ziel verspürt eine gewisse Abneigung für den Zaubernden.\\
Stufe 5: Das Ziel ist dem Zaubernden gegenüber neutral.\\
Stufe 6: Das Ziel verspürt eine gewisse Zuneigung für den Zaubernden.\\
Stufe 7: Das Ziel empfindet Vertrauen und Freundschaft zum Zaubernden.\\
Stufe 8: Das Ziel ist dem Zaubernden eng und loyal verbunden.\\
Stufe 9: Das Ziel ist dem Zaubernden verfallen (geht für ihn jedoch nicht in den Tod).\\
Der Zauberspruch verändert nicht das Erinnerungsvermögen. Dem Verzauberten ist nach Ende des Zaubers bewusst, was er empfunden und getan hat. Während der Zauberwirkung mag er sich des Zaubers bewusst sein, das beeinflusst aber seine neue Einstellung nicht. Handlungen des Zauberers hingegen können während der Wirkung das Verhältnis zum Verzauberten positiv oder negativ beeinflussen. \\
\textbf{Zauberdauer (ZD/ZZ):} 4 Aktionen \\
\textbf{Mana-Kosten (MK):} 8 MP \\
\textbf{Reichweite (RW):} 4 Meter \\
\textbf{Wirkungsdauer (WD):} QS x 3 Minuten \\
\textbf{Zielkategorie:} Kulturschaffende, übernatürliche Wesen \\
\textbf{Spezialisierung:} Einfluss \\
\textbf{Steigungsfaktor (SF):} B


\subsection{Blick in die Gedanken}
\label{chap:blick_in_die_gedanken}
\textbf{Probe (MTW):} MU/KL/IN (modifiziert um SK) \\
\textbf{Wirkung:} Der Zaubernde kann die Gedanken des Verzauberten lesen. Er kann nur erkennen, was der Verzauberte gerade denkt, nicht jedoch gezielt in seinen Erinnerungen forschen. Der Verzauberte kann mit einer Erfolgsprobe auf Sinnesschärfe erschwert um QS des Zaubers entdecken, dass etwas Fremdes in seine Gedanken blickt. Mit einer um QS der Zauberprobe erschwerte Probe auf Willenskraft kann er gezielt Gedanken mit irreführenden Informationen aussenden oder seine Gedanken mit nutzlosem Durcheinander, inneren Monologen oder Gesang füllen. \\
\textbf{Zauberdauer (ZD/ZZ):} 4 Aktionen \\
\textbf{Mana-Kosten (MK):} 8 MP (Aktivierung des Zaubers), 4 MP pro 30 Sekunden \\
\textbf{Reichweite (RW):} 4 Meter \\
\textbf{Wirkungsdauer (WD):} aufrechterhaltend \\
\textbf{Zielkategorie:} Kulturschaffende, übernatürliche Wesen \\
\textbf{Spezialisierung:} Wahrsagerei \\
\textbf{Steigungsfaktor (SF):} C


\subsection{Blitz dich find}
\label{chap:blitz_dich_find}
\textbf{Probe (MTW):} MU/IN/CH (modifiziert um SK) \\
\textbf{Wirkung:} Der Verzauberte wird geblendet. Er erhält eine Stufe des Zustands Verwirrung. \\
\textbf{Zauberdauer (ZD/ZZ):} 1 Aktion \\
\textbf{Mana-Kosten (MK):} 4 MP \\
\textbf{Reichweite (RW):} 8 Meter \\
\textbf{Wirkungsdauer (WD):} QS in Kampfrunden \\
\textbf{Zielkategorie:} Lebewesen \\
\textbf{Spezialisierung:} Einfluss \\
\textbf{Steigungsfaktor (SF):} A


\subsection{Claudibus}
\label{chap:claudibus}
\textbf{Probe (MTW):} KL/IN/FF \\
\textbf{Wirkung:} Durch diesen Zauberspruch verschließen sich Schlösser automatisch und es wird schwerer, sie wieder zu öffnen oder zu knacken. Pro QS werden Proben auf Schlösserknacken (Bartschlösser oder Kombinationsschlösser) und Kraftakt (Eintreten \& Zertrümmern) gegen das Schloss um 1 erschwert. \\
\textbf{Zauberdauer (ZD/ZZ):} 2 Aktionen \\
\textbf{Mana-Kosten (MK):} 4 MP \\
\textbf{Reichweite (RW):} Berührung \\
\textbf{Wirkungsdauer (WD):} 30 Minuten \\
\textbf{Zielkategorie:} Objekte (Schlösser) \\
\textbf{Spezialisierung:} Objekt \\
\textbf{Steigungsfaktor (SF):} B


\subsection{Desintegratus}
\label{chap:desintegratus}
\textbf{Probe (MTW):} KL/CH/KK \\
\textbf{Wirkung:} Das verzauberte Objekt verliert QS x 10 Strukturpunkte. Sinken die Strukturpunkte dadurch unter 1, zerfällt das Objekt zu Staub. Um Rüstungen zu zerstören, benötigt der Zauberer so viele QS, wie die Rüstung Rüstungsschutz hat. Bei Waffen der Kampftechniken Dolche und Raufen muss 1 QS erzielt werden, bei andere Waffen mindestens 2 QS. Zweihandwaffe können nur mit mindestens 3 QS zerstört werden. Erzielte QS gegen Waffen summieren sich auf.\\
\textbf{Zauberdauer (ZD/ZZ):} 8 Aktionen \\
\textbf{Mana-Kosten (MK):} 16 MP \\
\textbf{Wirkungsdauer (WD):} sofort \\
\textbf{Zielkategorie:} Profane Objekte \\
\textbf{Spezialisierung:} Objekt \\
\textbf{Steigungsfaktor (SF):} B


\subsection{Duplicatus}
\label{chap:duplicatus}
\textbf{Probe (MTW):} KL/IN/CH \\
\textbf{Wirkung:} Der Zaubernde erzeugt einen oder mehrere illusionäre Doppelgänger des Ziels, die sich synchron zu ihm bewegen. Je nach QS erscheinen mehrere Doppelgänger. Der Zaubernde kann sich entscheiden, weniger Doppelgänger als maximal möglich erscheinen zu lassen:\\
QS 1: 1 Doppelgänger, aber nur 2 Kampfrunden\\
QS 2: 1 Doppelgänger\\
QS 3: 2 Doppelgänger\\
QS 4: 3 Doppelgänger\\
QS 5: 4 Doppelgänger\\
QS 6: 4 Doppelgänger, aber doppelte Wirkungsdauer\\
Da die Doppelgänger nicht vom Zaubernden zu unterscheiden sind, haben Angreifer mit Nah- und Fernkampfangriffen oder Zaubern eine entsprechende Chance, einen der Doppelgänger statt des Zaubernden zu treffen. Sollte der Zaubernde statt eines Doppelgängers getroffen werden, muss er den Angriff blocken oder ihm ausweichen. Flächenangriffe sind hiervon nicht betroffen. \\
\textbf{Zauberdauer (ZD/ZZ):} 2 Aktionen \\
\textbf{Mana-Kosten (MK):} 4 MP pro Doppelgänger (bei Misslingen entsprechend 2 MP) \\
\textbf{Reichweite (RW):} Berührung \\
\textbf{Wirkungsdauer (WD):} QS x 3 Kampfrunden \\
\textbf{Zielkategorie:} Lebewesen \\
\textbf{Spezialisierung:} Illusion \\
\textbf{Steigungsfaktor (SF):} C


\subsection{Ecliptifactus}
\label{chap:ecliptifactus}
\textbf{Probe (MTW):} MU/IN/CH \\
\textbf{Wirkung:} Der Schatten des Zauberers wird durch diesen Zauberspruch belebt und kämpft für ihn. Mit dem Ecliptifactus kann der Magier immer nur einen Schatten heraufbeschwören. Erst nach Ablauf der Wirkungsdauer kann er den Zauber erneut wirken. Der Schatten besitzt die unten aufgeführten Werte. Zusätzlich erhält der Schatten folgende Boni:\\
QS 1: keine Änderungen\\
QS 2: +1TW, +5 LP\\
QS 3: +1TW, +1 BL, +1 AW, +1 TP, +5 LP\\
QS 4: +2TW, +1 BL, +1 AW, +1 TP, +1 RS, +10 LP\\
QS 5: +2TW, +2 BL, +2 AW, +1 TP, +1 RS, +10 LP\\
QS 6: +3TW, +2 BL, +2 AW, +2 TP, +1 RS, +15 LP\\
Der Schatten ist nicht durch Magie beeinflussbar und erleidet keine Zustände. Sinken seine LP auf 0, so verliert der Zaubernde schlagartig seine gesamten Manapunkte. Es dauert 7 Wochen, bis der Schatten sich regeneriert hat. Bis dahin besitzt der Zaubernde keinen Schatten und kann den Zauberspruch entsprechend auch nicht anwenden. \\
\textbf{Zauberdauer (ZD/ZZ):} 2 Aktionen \\
\textbf{Mana-Kosten (MK):} 4 MP (Aktivierung des Zaubers) + 2 MP pro Kampfrunde/2 MP pro Stunde \\
\textbf{Reichweite (RW):} selbst \\
\textbf{Wirkungsdauer (WD):} aufrechterhaltend \\
\textbf{Zielkategorie:} Lebewesen \\
\textbf{Spezialisierung:} Dämonisch \\
\textbf{Steigungsfaktor (SF):} C

\subsubsection{Schatten}

\textbf{MU} 15 \textbf{KL} 12 \textbf{IN} 13 \textbf{CH} 12 \\
\textbf{FF} 12 \textbf{GE} 13 \textbf{KO} 13 \textbf{KK} 14

\textbf{LP} 25 \textbf{MP} – \textbf{INI} 14+1W6

\textbf{AW} 7 \textbf{SK} 3 \textbf{ZK} 1 \textbf{GS} 8

Angriff: \\
\textbf{TW} 13 \textbf{BL} 8 \textbf{TP} 1W6+3 \textbf{RW} 1m

\textbf{RS/BE} 2/0

\textbf{Aktionen:} 1

\textbf{Sonderfertigkeiten:} keine

\textbf{Talente:} Klettern 6, Körperbeherrschung 7, Kraftakt 7, Selbstbeherrschung 10, Sinnesschärfe 7, Verbergen 12, Einschüchtern 10, Willenskraft 7

\textbf{Größenkategorie:} mittel

\textbf{Typus:} übernatürliches Wesen, humanoid

\textbf{Kampfverhalten:} Der Schatten verteidigt den Zaubernden, so gut er kann und geht, wenn dieser nicht direkt angegriffen wird, auch von sich aus zum Angriff auf potentielle Feinde über oder folgt dem Zauberer. Er führt jedoch keine Befehle des Zaubernden aus. Der Schatten versucht sich immer in einem Radius von 13 Schritten um den Zauberer herum aufzuhalten. Wird er durch irgendwelche Handlungen aus diesem Bereich gedrängt, versucht er den Radius wieder zu erreichen.

\textbf{Flucht:} Der Schatten flieht nie.


\subsection{Exposami}
\label{chap:exposami}
\textbf{Probe (MTW):} KL/IN/CH \\
\textbf{Wirkung:} Der Zauberer kann alle Lebewesen innerhalb der Reichweite des Zauberspruchs spüren, sofern sie mindestens so groß sind wie eine Ratte. Die Auren nimmt der Zauberer als grün-leuchtende Flecken wahr. Er ist jedoch nicht darauf angewiesen diese zu sehen, um Lebewesen zu spüren. Pflanzen, Pilze und Lebewesen, die kleiner als eine Ratte sind, können durch den Zauberspruch nicht erkannt werden. Bauwerke aus den Elementen Erz oder Eis stören die Wirkung des Spruches. Hinter einer Mauer aus Stein oder in einem Iglu kann der Zauberer also keine Lebewesen orten. Der Zauberer kann erkennen, um was für einen Typus von Lebewesen es sich handelt.\\
\textbf{Zauberdauer (ZD/ZZ):} 2 Aktionen \\
\textbf{Mana-Kosten (MK):} 4 MP \\
\textbf{Reichweite (RW):} QS x 20 Meter \\
\textbf{Wirkungsdauer (WD):} 10 Kampfrunden \\
\textbf{Zielkategorie:} Lebewesen (außer Pflanzen, Pilze und Wesen, die kleiner als eine Ratte sind) \\
\textbf{Spezialisierung:} Wahrsagerei \\
\textbf{Steigungsfaktor (SF):} A


\subsection{Foramen}
\label{chap:foramen}
\textbf{Probe (MTW):} KL/IN/FF \\
\textbf{Wirkung:} Durch diesen Zauberspruch öffnen sich einfache Schlösser automatisch, bei komplizierten Schlössern wird es leichter, sie zu öffnen. Pro QS werden Proben auf Schlösserknacken (Bartschlösser oder Kombinationsschlösser) und Kraftakt (Eintreten \& Zertrümmern) gegen das Schloss um 1 erleichtert. Sollte der Gesamtmodifikator zum Knacken eines Schlosses dadurch zu einer Erleichterung werden, springt das Schloss augenblicklich auf. Die Erleichterung gilt binnen der Wirkungsdauer sowohl für den Zauberer als auch für andere Personen. \\
\textbf{Zauberdauer (ZD/ZZ):} 2 Aktionen \\
\textbf{Mana-Kosten (MK):} 8 MP \\
\textbf{Reichweite (RW):} Berührung \\
\textbf{Wirkungsdauer (WD):} 5 Minuten \\
\textbf{Zielkategorie:} Objekte (Schlösser) \\
\textbf{Spezialisierung:} Objekt \\
\textbf{Steigungsfaktor (SF):} C


\subsection{Fortifex}
\label{chap:fortifex}
\textbf{Probe (MTW):} MU/IN/KO \\
\textbf{Wirkung:} Durch den Fortifex entsteht eine durchscheinende, allerdings leicht flimmernde (und damit sichtbare) Wand, die für physische Objekte undurchdringlich ist. Die Wand kann beispielsweise Pfeile, Bolzen, einen Ignifaxius oder Lebewesen aufhalten. Sie schützt nicht vor Hitze, Kälte, Licht, Gasen, körperlosen Wesen oder Zaubern, die keine physische Komponente aufweisen. Der Zauberer muss einen Mittelpunkt innerhalb der Reichweite des Zauberspruchs benennen. Von dort aus wirkt gradlinig eine Barriere, die QS x 2 Meter lang ist. Den exakten Verlauf durch den Mittelpunkt kann der Zauberer festlegen. Die Höhe der Barriere beträgt 3 Meter. Eine Tiefe besitzt die Wand nicht. Man kann sie nur aufgerichtet platzieren, nicht flach hinlegen oder schräg stellen. Die Wand besitzt 100 Strukturpunkte. Sollten die Strukturpunkte auf 0 oder darunter sinken, löst sie sich auf. \\
\textbf{Zauberdauer (ZD/ZZ):} 4 Aktionen \\
\textbf{Mana-Kosten (MK):} 16 MP \\
\textbf{Reichweite (RW):} 8 Meter \\
\textbf{Wirkungsdauer (WD):} 30 Kampfrunden \\
\textbf{Zielkategorie:} Zone \\
\textbf{Spezialisierung:} Psychokinetik \\
\textbf{Steigungsfaktor (SF):} B


\subsection{Fulminictus}
\label{chap:fulminictus}
\textbf{Probe (MTW):} KL/IN/KO (modifiziert um ZK) \\
\textbf{Wirkung:} Der Verzauberte erleidet Schaden, der auf seine Aura einwirkt und eine Reihe kleiner innerer Verletzungen erzeugt. Er erleidet 2W6+(QS x 2) Trefferpunkte. Rüstungen schützen nicht vor diesem Zauber. Der Zauberspruch trifft sein Ziel automatisch. Der Gegner kann weder ausweichen noch blocken. \\
\textbf{Zauberdauer (ZD/ZZ):} 1 Aktion \\
\textbf{Mana-Kosten (MK):} 8 MP (Kosten sind nicht modifizierbar) \\
\textbf{Reichweite (RW):} 8 Meter \\
\textbf{Wirkungsdauer (WD):} sofort \\
\textbf{Zielkategorie:} Lebewesen \\
\textbf{Spezialisierung:} Verwandlung \\
\textbf{Steigungsfaktor (SF):} C


\subsection{Gefunden}
\label{chap:gefunden}
\textbf{Probe (MTW):} KL/IN/GE \\
\textbf{Wirkung:} Der Zauberer kann die Richtung bestimmen, in der sich ein Objekt befindet. Er muss dieses insgesamt mindestens einen Monat lang bei sich getragen haben oder über eine magische Verbindung zu ihm verfügen. Ist das Objekt weiter als QS x 3 Meter von ihm entfernt, scheitert der Zauber.\\
\textbf{Zauberdauer (ZD/ZZ):} 8 Aktionen \\
\textbf{Mana-Kosten (MK):} 2 MP \\
\textbf{Reichweite (RW):} selbst \\
\textbf{Wirkungsdauer (WD):} sofort \\
\textbf{Zielkategorie:} Objekte \\
\textbf{Spezialisierung:} Wahrsagerei \\
\textbf{Steigungsfaktor (SF):} A


\subsection{Herzschlag ruhe}
\label{chap:herzschlag_ruhe}
\textbf{Probe (MTW):} MU/CH/KK (modifiziert um ZK) \\
\textbf{Wirkung:} Der Herzschlag des Ziels wird ausgesetzt, wodurch es jede Kampfrunde 1W4 LP verliert. Mit Hilfe einer Probe auf \textit{Selbstbeherrschung} erschwert um die QS oder auf \textit{Heilkunde} (Stabilisieren) erschwert um QS/2 kann die Wirkung des Zaubers gestoppt werden. Eine Probe zählt als Aktion. Die \textit{Heilkunde}-Probe kann sowohl das Opfer des Zaubers selbst als auch eine dritte Person einsetzen.\\
\textbf{Zauberdauer (ZD/ZZ):} 4 Aktionen \\
\textbf{Mana-Kosten (MK):} 16 MP \\
\textbf{Reichweite (RW):} Berührung \\
\textbf{Wirkungsdauer (WD):} QS x 3 Kampfrunden \\
\textbf{Zielkategorie:} Lebewesen \\
\textbf{Spezialisierung:} Dämonisch \\
\textbf{Steigungsfaktor (SF):} C


\subsection{Hilfreiche Pfote}
\label{chap:hilfreiche_pfote}
\textbf{Probe (MTW):} MU/IN/CH \\
\textbf{Wirkung:} Der Zauberer wählt eine Säugetierart aus und konzentriert sich beim Zaubern auf diese. Das Tier darf maximal die Größenkategorie klein aufweisen. Ist innerhalb der Reichweite des Zaubers ein solches Tier anwesend, bewegt es sich zum Zauberer. Sind mehrere Tiere dieser Art anwesend, so wählt der Meister zufällig eines davon aus. Ist kein solches Tier vorhanden, misslingt der Zauber automatisch. Das Tier ist bereit, dem Zauberer einen einzigen, kleinen Dienst zu erweisen (eine Maus kann ein Fesselseil durchnagen, ein Eichhörnchen kann auf einen Baum klettern und von dort etwas herunterholen usw.). Der Dienst muss für den Zauberer erfolgen und nicht für eine dritte Person, er darf der Natur des Tiers nicht widersprechen und muss für das Tier erfüllbar sein. Kein Dienst darf einen Kampf oder Angriff auf andere Wesen beinhalten.\\
\textbf{Zauberdauer (ZD/ZZ):} 16 Aktionen \\
\textbf{Mana-Kosten (MK):} 8 MP (Kosten sind nicht modifizierbar) \\
\textbf{Reichweite (RW):} 64 Meter \\
\textbf{Wirkungsdauer (WD):} QS x 10 Minuten \\
\textbf{Zielkategorie:} Tiere (Säugetiere) \\
\textbf{Spezialisierung:} Einfluss \\
\textbf{Steigungsfaktor (SF):} A


\subsection{Ignifaxius}
\label{chap:ignifaxius}
\textbf{Probe (MTW):} MU/KL/CH \\
\textbf{Wirkung:} Aus den Fingern des Zaubernden schießt einen Flammenstrahl, der in gerader Linie sein Ziel trifft. Der Magier muss keine zusätzliche Aktion aufwenden, um nach dem Wirken des Zaubers zu treffen. Das Treffen ist in der Zauberdauer inbegriffen. Das getroffene Ziel erleidet 2W6+(QS x 2) Trefferpunkte Schaden. Der Flammenstrahl zählt als Fernkampfangriff mit einer Schusswaffe und kann entsprechend geblockt werden, und auch ein Ausweichen ist möglich. An Schilden erzeugt er Strukturschaden, wenn er auf sie trifft. Der Strahl trifft automatisch, wenn man sich nicht verteidigt. Trifft der Flammenstrahl sein Ziel, werden die TP durch den RS des Ziels vermindert. Entflammbare Ziele erhalten bei 1-3 auf 1W6 den Status Brennend. \\
\textbf{Zauberdauer (ZD/ZZ):} 2 Aktionen \\
\textbf{Mana-Kosten (MK):} 8 MP (Kosten sind nicht modifizierbar) \\
\textbf{Reichweite (RW):} 16 Meter \\
\textbf{Wirkungsdauer (WD):} sofort \\
\textbf{Zielkategorie:} alle \\
\textbf{Spezialisierung:} Elementar \\
\textbf{Steigungsfaktor (SF):} C


\subsection{Ignisphaero}
\label{chap:ignisphaero}
\textbf{Probe (MTW):} MU/KL/KO \\
\textbf{Wirkung:} Aus den Händen des Zaubernden schießt ein Feuerball, der in gerader Linie sein Ziel trifft. Der Treffer des Feuerballs erfolgt am Ende der Zauberdauer, der Magier muss keine zusätzliche Aktion dafür aufwenden. Der Feuerball explodiert, wenn er sein Ziel oder auf ein großes, festes Hindernis trifft (z. B. eine Wand oder einen Schild). Verlässt der Feuerball die Reichweite, ohne ein Ziel zu treffen, so löst er sich auf. Auch die Gefährten des Zauberers können von dem Feuerball getroffen werden, wenn er explodiert. Der Radius der Explosion beträgt 5 Meter. Der Schaden beträgt bis zu einem Meter vom Zentrum entfernt 2W6+(QS x 3) Trefferpunkte (auch gegen leblose Objekte als Strukturschaden).\\
Pro Meter Entfernung von diesem Bereich verringert sich der Schaden um jeweils 3 TP. Der Feuerball trifft automatisch, wenn man ihm nicht aktiv zu entgehen versucht. Dazu muss eine Verteidigung eingesetzt und eine Probe auf Körperbeherrschung (Kampfmanöver) abgelegt werden. Pro QS hat der Held sich 1 Meter vom Zentrum der Explosion entfernt. Trifft der Feuerball sein Ziel, werden die TP durch den RS des Ziels vermindert. Entflammbare Ziele fangen bei 1-3 auf 1W6 Feuer. \\
\textbf{Zauberdauer (ZD/ZZ):} 4 Aktionen \\
\textbf{Mana-Kosten (MK):} 29 MP (Kosten sind nicht modifizierbar) \\
\textbf{Reichweite (RW):} 32 Meter \\
\textbf{Wirkungsdauer (WD):} sofort \\
\textbf{Zielkategorie:} Zone \\
\textbf{Spezialisierung:} Elementar \\
\textbf{Steigungsfaktor (SF):} D


\subsection{Imperavi}
\label{chap:imperavi}
\textbf{Probe (MTW):} MU/KL/CH (modifiziert um SK) \\
\textbf{Wirkung:} Das Ziel befolgt einen einzigen Befehl des Zauberers, der seinem Selbsterhaltungstrieb nicht offensichtlich zuwiderläuft. Es wird also einen Diebstahl begehen, eine einzige Frage beantworten oder dem Zauberer in einem Kampf beistehen, nicht aber von einer Klippe springen oder seine eigenen Hände verspeisen.\\
Der Zauber kann nicht dazu eingesetzt werden, dass das Opfer während der kompletten Wirkungsdauer Befehle entgegennimmt („Ich befehle dir, allen meinen Befehlen zu gehorchen.“) oder mehrere Fragen wahrheitsgemäß zu beantworten. Auch lange Befehlsketten sind nicht erlaubt („Öffne die Tür, gehe hinein, mache das Licht an und zerschlage die Kiste.“), sondern nur einzelne, knappe Befehle. Der SL hat hierbei das letzte Wort, ob ein Befehl möglich ist oder nicht.\\
\textbf{Zauberdauer (ZD/ZZ):} 16 Aktionen \\
\textbf{Mana-Kosten (MK):} 16 MP \\
\textbf{Reichweite (RW):} Berührung \\
\textbf{Wirkungsdauer (WD):} QS x 3 Minuten \\
\textbf{Zielkategorie:} Kulturschaffende \\
\textbf{Spezialisierung:} Einfluss \\
\textbf{Steigungsfaktor (SF):} C


\subsection{Katzenaugen}
\label{chap:katzenaugen}
\textbf{Probe (MTW):} KL/IN/KO \\
\textbf{Wirkung:} Die Lichtempfindlichkeit der Augen des Zieles erhöht sich ungemein. So kann es auch in fast völliger Dunkelheit deutlich besser sehen. Pro QS sinkt die Stufe der Sichterschwernis um 1. In völliger Dunkelheit nützt dieser Zauberspruch nichts. \\
\textbf{Zauberdauer (ZD/ZZ):} 4 Aktionen \\
\textbf{Mana-Kosten (MK):} 2 MP (Aktivierung) + 1 MP pro 10 Minuten \\
\textbf{Reichweite (RW):} selbst \\
\textbf{Wirkungsdauer (WD):} aufrechterhaltend \\
\textbf{Zielkategorie:} Wesen \\
\textbf{Spezialisierung:} Verwandlung \\
\textbf{Steigungsfaktor (SF):} A

\subsection{Klarum Purum}
\label{chap:klarum_purum}
\textbf{Probe (MTW):} KL/IN/CH \\
\textbf{Wirkung:} Der Zauberspruch hebt die Wirkung eines Giftes auf. Die stärke des Gifts darf die QS nicht übersteigen, sonst wirkt der Zauber nicht und gilt als misslungen. Ob die QS ausreicht entscheidet der SL. \\
\textbf{Zauberdauer (ZD/ZZ):} 4 Aktionen \\
\textbf{Mana-Kosten (MK):} 3 MP pro Giftstufe (Kosten nicht modifizierbar) \\
\textbf{Reichweite (RW):} 4 Meter \\
\textbf{Wirkungsdauer (WD):} sofort \\
\textbf{Zielkategorie:} Lebewesen \\
\textbf{Spezialisierung:} Heilung \\
\textbf{Steigungsfaktor (SF):} B


\subsection{Krabbelnder Schrecken}
\label{chap:krabbelnder_schrecken}
\textbf{Probe (MTW):} MU/CH/FF \\
\textbf{Wirkung:} Haut und Kleidung des Ziels werden von unzähligen Kleintieren bedeckt. Werden sie abgeschüttelt oder anderweitig entfernt, erscheinen sofort neue Exemplare. Alle Proben sind um QS/2 erschwert. Am Anfang jeder KR muss eine Probe auf Selbstbeherrschung (Störungen ignorieren) erschwert um 2 abgelegt werden. Bei Misslingen erleidet das Opfer eine 1 Stufe Furcht. Die Furcht ist kumulativ, das Opfer kann also über mehrere KR und misslungene Proben immer mehr Furchtstufen erhalten.\\
\textbf{Zauberdauer (ZD/ZZ):} 2 Aktionen \\
\textbf{Mana-Kosten (MK):} 16 MP \\
\textbf{Reichweite (RW):} 4 Meter \\
\textbf{Wirkungsdauer (WD):} 10 KR \\
\textbf{Zielkategorie:} Lebewesen \\
\textbf{Spezialisierung:} Dämonisch \\
\textbf{Steigungsfaktor (SF):} C


\subsection{Kraft des Tieres}
\label{chap:kraft_des_tieres}
\textbf{Probe (MTW):} MU/IN/KO \\
\textbf{Wirkung:} Durch diesen Zauberspruch erlangt das Ziel für einen kurzen Zeitraum große Kräfte. Der FW im Talent \textit{Kraftakt} erhöht sich um die QS. Maximal können bei einer solchen Probe aber nur 18 FW erzielt werden.\\
\textbf{Zauberdauer (ZD/ZZ):} 2 Aktionen \\
\textbf{Mana-Kosten (MK):} 4 MP \\
\textbf{Reichweite (RW):} selbst \\
\textbf{Wirkungsdauer (WD):} 5 KR \\
\textbf{Zielkategorie:} Kulturschaffende \\
\textbf{Spezialisierung:} Verwandlung \\
\textbf{Steigungsfaktor (SF):} A


\subsection{Lunge des Leviatan}
\label{chap:lunge_des_leviatan}
\textbf{Probe (MTW):} MU/IN/KO \\
\textbf{Wirkung:} Während der Wirkungsdauer kann der Zauberer sprinten, ohne dass ihn dies erschöpft. Sein FW in Körperbeherrschung (Laufen) wird für die Wirkungsdauer verdoppelt. Bei Verfolgungsjagden gilt diese Verdopplung jedoch vor der Einberechnung der effektiven GS.\\
\textbf{Zauberdauer (ZD/ZZ):} 2 Aktionen \\
\textbf{Mana-Kosten (MK):} 4 MP \\
\textbf{Reichweite (RW):} selbst \\
\textbf{Wirkungsdauer (WD):} QS Kampfrunden \\
\textbf{Zielkategorie:} Lebewesen \\
\textbf{Spezialisierung:} Heilung \\
\textbf{Steigungsfaktor (SF):} A


\subsection{Manus Illusionis}
\label{chap:manus_illusionis}
\textbf{Probe (MTW):} KL/IN/CH \\
\textbf{Wirkung:} Mit diesem Illusionszauber ist die Täuschung des Tastsinnes möglich. Die Illusion muss dabei auf eine bereits vorhandene Oberfläche gelegt werden, da sie lediglich die Art des Fühlens ändert, jedoch nicht etwas Fassbares aus dem Nichts zu erschaffen vermag. Nur die Oberfläche fühlt sich anders an, nicht jedoch das gesamte Objekt.\\
\textbf{Zauberdauer (ZD/ZZ):} 4 Aktionen \\
\textbf{Mana-Kosten (MK):} 4 MP (Aktivierung des Zaubers) + 2 MP pro 5 Minuten \\
\textbf{Reichweite (RW):} 8 Meter \\
\textbf{Wirkungsdauer (WD):} aufrechterhaltend \\
\textbf{Zielkategorie:} Zone (max. 3 x 3 Meter) \\
\textbf{Spezialisierung:} Illusion \\
\textbf{Steigungsfaktor (SF):} A


\subsection{Menetekel}
\label{chap:menetekel}
\textbf{Probe (MTW):} KL/CH/FF \\
\textbf{Wirkung:} Der Zauber ermöglicht es, illusionäre Schriftzüge erscheinen zu lassen. Alles, was der Zauberer sagt, erscheint an einer imaginären Wand (max. $10m^2$), die sich beliebig in Breite und Länge ausformen kann. Die Wand kann auch schräg sein und in der Luft "`schweben"'. Die Buchstaben können in beliebiger Größe und Anzahl dargestellt werden, gelten aber nicht als Sichthindernis. Die Buchstaben können auch auf eine Buchseite dargestellt werden.\\
\textbf{Zauberdauer (ZD/ZZ):} 4 Aktionen \\
\textbf{Mana-Kosten (MK):} 2 MP \\
\textbf{Reichweite (RW):} 8 Meter \\
\textbf{Wirkungsdauer (WD):} QS x 30 Minuten \\
\textbf{Zielkategorie:} Zone \\
\textbf{Spezialisierung:} Illusion \\
\textbf{Steigungsfaktor (SF):} A


\subsection{Motoricus}
\label{chap:motoricus}
\textbf{Probe (MTW):} KL/FF/KK \\
\textbf{Wirkung:} Der Zaubernde kann unbelebte Gegenstände telekinetisch anheben und bewegen. Sie bewegen sich mit einer Geschwindigkeit von maximal QS + 2 Meter pro Aktion und dürfen maximal QS x 20 Kilogramm wiegen. Pro 5 Kilogramm Gewicht muss der Zaubernde dafür 1 MP aufbringen. Befinden sich andere Objekte auf dem zu bewegenden Gegenstand, zählt deren Gewicht mit hinzu. Die Bewegungen des Objekts sind träge, weshalb sie für Angriffe oder Paraden ungeeignet sind. Um ein so bewegtes Objekt durch Festhalten oder Drücken aufzuhalten, ist eine um QS erschwerte Probe auf Kraftakt (Ziehen \& Zerren) nötig.\\
\textbf{Zauberdauer (ZD/ZZ):} 2 Aktionen \\
\textbf{Mana-Kosten (MK):} mindestens 4 MP (Aktivierung) + Hälfte der notwendigen MP pro 5 Minuten (Kosten sind nicht modifizierbar) \\
\textbf{Reichweite (RW):} 8 Meter \\
\textbf{Wirkungsdauer (WD):} aufrechterhaltend \\
\textbf{Zielkategorie:} Objekt \\
\textbf{Spezialisierung:} Psychokinetik \\
\textbf{Steigungsfaktor (SF):} B


\subsection{Nuntiovolo}
\label{chap:nuntiovolo}
\textbf{Probe (MTW):} MU/KL/CH \\
\textbf{Wirkung:} Der Zauberer beschwört eine Rauchgestalt in Form eines Vogels oder einer Fledermaus herauf, die ein maximal 250 Gramm wiegendes Objekt an einen vom Zauberer bestimmten Ort bringen kann. Die Entfernung zum Zielort darf höchstens 20 Meilen + 10 Meilen pro 1 MP betragen. Sobald die Maximalentfernung überschritten wird, löst sich die Gestalt auf und lässt den Gegenstand fallen. Die Kreatur bewegt sich mit einer GS von 10 fort, ohne dabei vom Wetter beeinflusst zu werden. Sie besitzt eine Lebensenergie in Höhe der QS und verfügt über keinen Verteidigungswert, allerdings ist sie von profanen Waffen nicht zu verletzen.\\
\textbf{Zauberdauer (ZD/ZZ):} 8 Aktionen \\
\textbf{Mana-Kosten (MK):} 8 MP \\
\textbf{Reichweite (RW):} 4 Meter \\
\textbf{Wirkungsdauer (WD):} sofort \\
\textbf{Zielkategorie:} Dämonen \\
\textbf{Spezialisierung:} Sphären \\
\textbf{Steigungsfaktor (SF):} B


\subsection{Objectobscuro}
\label{chap:objectobscuro}
\textbf{Probe (MTW):} KL/FF/KO \\
\textbf{Wirkung:} Das Zielobjekt wird innerhalb von 20 – (QS x 3) Kampfrunden unsichtbar. Während dieser Zeit wird es immer durchscheinender, bis es schließlich nicht mehr zu sehen ist. Objekte und Personen in oder hinter dem Objekt bleiben weiterhin sichtbar. Das verzauberte Objekt darf maximal 5 Kilogramm wiegen.\\
\textbf{Zauberdauer (ZD/ZZ):} 4 Aktionen \\
\textbf{Mana-Kosten (MK):} 8 MP (Aktivierung des Zaubers) + 4 MP pro 5 Minuten \\
\textbf{Reichweite (RW):} Brührung \\
\textbf{Wirkungsdauer (WD):} aufrechterhaltend \\
\textbf{Zielkategorie:} Objekte \\
\textbf{Spezialisierung:} Objekt \\
\textbf{Steigungsfaktor (SF):} B


\subsection{Objectovoco}
\label{chap:objectovoco}
\textbf{Probe (MTW):} KL/IN/CH \\
\textbf{Wirkung:} Der Gegenstand muss auf QS Fragen wahrheitsgemäß antworten. Die Fragen müssen mit Ja oder Nein zu beantworten sein. Auf andere Fragen kann der Gegenstand schweigen. Die Wahrnehmung von Gegenständen ist begrenzt, ebenso deren Intelligenz. Sie nehmen nur wahr, was in einer Reichweite von ca. 3 Meter um sie herum geschieht. Ein einfaches Objekt wie ein Kilogramm ist zudem nicht so klug wie ein komplexer Gegenstand, etwa eine Uhr.\\
\textbf{Zauberdauer (ZD/ZZ):} 4 Aktionen \\
\textbf{Mana-Kosten (MK):} 4 MP \\
\textbf{Reichweite (RW):} Berührung \\
\textbf{Wirkungsdauer (WD):} sofort \\
\textbf{Zielkategorie:} Objekte \\
\textbf{Spezialisierung:} Objekt \\
\textbf{Steigungsfaktor (SF):} B


\subsection{Paralysis}
\label{chap:paralysis}
\textbf{Probe (MTW):} KL/IN/KO (modifiziert um ZK) \\
\textbf{Wirkung:} Der  Körper  des  Verzauberten  versteift und verhärtet sich. Wird er vollständig paralysiert (Stufe IV), verwandelt sich sein Körper in eine unzerstörbare Substanz, ohne sein Gewicht zu verändern. Ein so versteinerter Körper ist durch gewöhnliche Waffen, Feuer oder Stürze nicht zu verletzen. Das  Opfer  kann  sich  nicht  bewegen,  fühlen  oder hören, aber innerhalb seines Sichtfeldes sehen. Die Wirkung  von  Giften  und  Krankheiten  kommt  zum Stillstand.\\
QS 1:1 Stufe Paralyse, aber nur für 2 KR\\
QS 2: 1 Stufe Paralyse\\
QS 3: 2 Stufen Paralyse\\
QS 4: 3 Stufen Paralyse\\
QS 5: 4 Stufen Paralyse\\
QS 6: 4 Stufen Paralyse, aber für die doppelte Wirkungsdauer\\
\textbf{Zauberdauer (ZD/ZZ):} 2 Aktionen \\
\textbf{Mana-Kosten (MK):} 8 MP \\
\textbf{Reichweite (RW):} 8 Meter \\
\textbf{Wirkungsdauer (WD):} QS x 2 in Minuten \\
\textbf{Zielkategorie:} Lebewesen \\
\textbf{Spezialisierung:} Verwandlung \\
\textbf{Steigungsfaktor (SF):} B


\subsection{Penetrizzel}
\label{chap:penetrizzel}
\textbf{Probe (MTW):} MU/KL/IN \\
\textbf{Wirkung:} Der Zaubernde kann durch feste Materie blicken. Pro QS kann der Zaubernde 20cm durchblicken. Magiestörendes Material kann die Probe erschweren (z. B. Silber –2). Magische Objekte können nicht durchblickt werden. Herrscht auf der anderen Seite des Hindernisses Dunkelheit oder Nebel, wirkt sich dies ganz normal auf die Sicht des Zaubernden aus. \\
\textbf{Zauberdauer (ZD/ZZ):} 2 Aktionen \\
\textbf{Mana-Kosten (MK):} 4 MP (Aktivierung des Zaubers) + 2 MP pro Minute \\
\textbf{Reichweite (RW):} selbst \\
\textbf{Wirkungsdauer (WD):} aufrechterhaltend \\
\textbf{Zielkategorie:} alle \\
\textbf{Spezialisierung:} Wahrsagerei \\
\textbf{Steigungsfaktor (SF):} B


\subsection{Projectimago}
\label{chap:projectimago}
\textbf{Probe (MTW):} MU/IN/CH \\
\textbf{Wirkung:} Der Zauberer projiziert ein optisches Ebenbild seiner selbst an einen gewünschten Ort (inklusive Kleidung und Gegenständen, die er am Körper trägt). Für die dortigen Beobachter sieht es so aus, als stünde eine reale Person vor ihnen, die alle Handlungen vollzieht, die der Zauberer durch Bewegung vorgibt. Dreht er sich also im Kreis und springt in die Luft, wird die Projektion das Gleiche tun. Der Ort der Erscheinung darf maximal 5 Meilen + 1 Meile pro vom Zauberer zusätzlich freiwillig investierten MP entfernt sein, und dieser muss ihn schon einmal aufgesucht haben oder ihn sehen können. Eine Übertragung der Sinne findet nicht statt, der Zauberer kann also im Normalfall nur vermuten, wie die Reaktionen am Zielort ausfallen.\\
\textbf{Zauberdauer (ZD/ZZ):} 8 Aktionen \\
\textbf{Mana-Kosten (MK):} 8 MP + 1 MP pro zusätzlicher Meile \\
\textbf{Reichweite (RW):} selbst \\
\textbf{Wirkungsdauer (WD):} QS x 3 Minuten \\
\textbf{Zielkategorie:} Kulturschaffende \\
\textbf{Spezialisierung:} Illusion \\
\textbf{Steigungsfaktor (SF):} B


\subsection{Radau}
\label{chap:radau}
\textbf{Probe (MTW):} KL/FF/KK \\
\textbf{Wirkung:} Der Besen oder ein anderer Stab oder Stock nach Wahl des Zaubernden greift von sich aus einen von der Hexe bestimmten Gegner innerhalb von 8 Meter an (nach dem Wirken des Zaubers können Ziel und Besen sich weiter entfernen). Er gilt während des Zaubers als unzerbrechlich und magisch und schlägt eine Attacke pro Runde zu. Er kann dabei keine Kampfmanöver ausführen. Seine Werte sind INI 12+1W6, TP 10+(QS), TP 1W6+3, GS 12. Sollte der Besen vor Ablauf der Wirkungsdauer den Gegner getötet haben oder keinen Gegner mehr vor sich haben, dann greift er ein weiteres zufälliges Opfer innerhalb von 16 Meter an (eventuell auch die Hexe selbst). Ist kein Opfer in dieser Reichweite, endet der Zauber. Um den Besen festzuhalten, ist ein erfolgreicher Angriff mit Raufen und der Sonderfertigkeit Haltegriff nötig. Für diese Zwecke verfügt der Stab über eine BL in Höhe der Hälfte der TP und einen Wert in Kraftakt von 10 Punkten mit Eigenschaften von 14 für die Probe. \\
\textbf{Zauberdauer (ZD/ZZ):} 2 Aktionen \\
\textbf{Mana-Kosten (MK):} 4 MP (Aktivierung des Zaubers) + 2 MP pro Kampfrunde \\
\textbf{Reichweite (RW):} 16 Meter \\
\textbf{Wirkungsdauer (WD):} aufrechterhaltend \\
\textbf{Zielkategorie:} Objekt \\
\textbf{Spezialisierung:} Psychokinetik \\
\textbf{Steigungsfaktor (SF):} B


\subsection{Regeneratio}
\label{chap:regeneratio}
\textbf{Probe (MTW):} IN/CH/KO \\
\textbf{Wirkung:} Der Zauberer regeneriert jede Kampfrunde LP in Höhe der QS/2. Maximal kann der Zauberspruch FW Kampfrunden aufrechterhalten werden.\\
\textbf{Zauberdauer (ZD/ZZ):} 4 Aktionen (Zauberdauer ist nicht modifizierbar) \\
\textbf{Mana-Kosten (MK):} 4 MP (Aktivierung des Zaubers) + 2 MP pro KR (Kosten nicht modifizierbar) \\
\textbf{Reichweite (RW):} selbst \\
\textbf{Wirkungsdauer (WD):} aufrechterhaltend \\
\textbf{Zielkategorie:} Lebewesen \\
\textbf{Spezialisierung:} Heilung \\
\textbf{Steigungsfaktor (SF):} D


\subsection{Runenzauber}
\label{chap:runenzauber}
\textbf{Probe (MTW):} KL/CH/IN \\
\textbf{Wirkung:} Ermöglicht die Herstellung von Glyphen und das Einbetten/Entfernen von Glyphen in/aus Ausrüstungsgegenständen. Siehe Kapitel \ref{chap:glyphen}. \\
\textbf{Zauberdauer (ZD/ZZ):} 60 Aktionen (Zauberdauer ist nicht modifizierbar) \\
\textbf{Mana-Kosten (MK):} 6 MP \\
\textbf{Reichweite (RW):} 1 Meter \\
\textbf{Wirkungsdauer (WD):} aufrechterhaltend \\
\textbf{Zielkategorie:} Objekt (Waffe/Schild/Rüstung) \\
\textbf{Spezialisierung:} Objekt \\
\textbf{Steigungsfaktor (SF):} C


\subsection{Sensibar}
\label{chap:sensibar}
\textbf{Probe (MTW):} MU/IN/CH (modifiziert um SK) \\
\textbf{Wirkung:} Der Zaubernde kann die Gefühle des Verzauberten ergründen. Der Zauber offenbart nur die Gefühle selbst, nicht jedoch deren Grund oder Ursprung. Für den Zaubernden sind Proben auf Menschenkenntnis (Lügen durchschauen oder Motivation erkennen) gegen das Ziel des Sensibar um die erreichte QS erleichtert. Der Verzauberte kann bemerken, dass etwas Fremdes in seinen Geist blickt. Hierzu muss ihm eine Erfolgsprobe auf Sinnesschärfe erschwert um die QS des Zaubers gelingen. Er kann aber auch dann nicht zweifelsfrei sagen, wer dafür verantwortlich ist. Mit einer Probe auf Willenskraft, ebenfalls erschwert um die QS des Zaubers, kann er seine Gefühle gezielt unterdrücken. Ist der Verzauberte emotional stark aufgewühlt, ist dies jedoch nach SLentscheid um bis zu 3 Punkte zusätzlich erschwert. \\
\textbf{Zauberdauer (ZD/ZZ):} 4 Aktionen \\
\textbf{Mana-Kosten (MK):} 8 MP (Aktivierung des Zaubers), 4 MP pro Minute \\
\textbf{Reichweite (RW):} 4 Meter \\
\textbf{Wirkungsdauer (WD):} aufrechterhaltend \\
\textbf{Zielkategorie:} Kulturschaffende \\
\textbf{Spezialisierung:} Wahrsagerei \\
\textbf{Steigungsfaktor (SF):} B


\subsection{Silentium}
\label{chap:silentium}
\textbf{Probe (MTW):} KL/FF/KK \\
\textbf{Wirkung:} Der Zaubernde erschafft eine kugelförmige Zone, in der keine Geräusche mehr übertragen werden. Der Radius der Zone beträgt QS x 3 in Meter. Die Zone entsteht mit dem Zaubernden als Mittelpunkt. Er kann sich vor der Fertigkeitsprobe entscheiden, ob sie sich mit ihm mit bewegt oder an Ort und Stelle verbleiben soll. In letzterem Falle darf er sich maximal QS x 3 Meter von der Zone entfernen, ansonsten endet der Zauber. \\
\textbf{Zauberdauer (ZD/ZZ):} 8 Aktionen \\
\textbf{Mana-Kosten (MK):} 4 MP (Aktivierung des Zaubers) + 2 MP pro 5 Minuten \\
\textbf{Reichweite (RW):} selbst \\
\textbf{Wirkungsdauer (WD):} aufrechterhaltend \\
\textbf{Zielkategorie:} Zone \\
\textbf{Spezialisierung:} Psychokinetik \\
\textbf{Steigungsfaktor (SF):} B


\subsection{Solidirid}
\label{chap:solidirid}
\textbf{Probe (MTW):} MU/IN/KO \\
\textbf{Wirkung:} Von den Füßen des Zauberers ausgehend erscheint eine Brücke aus mattem Licht. Ihre maximale Länge beträgt QS x 5 Meter, die maximale Breite 1 Meter. Sie ist entweder waagerecht oder leicht gewölbt, muss aber an keiner Stelle den Boden berühren. Wird sie auf ihrer gesamten Fläche mit einem höheren Gewicht als 500 Kilogramm belastet, bricht sie zusammen und der Zauber endet vorzeitig. Das Gleiche geschieht, wenn sie mit Waffengewalt zerstört wird, wobei sie insgesamt 50 Strukturpunkte aufweist.\\
\textbf{Zauberdauer (ZD/ZZ):} 4 Aktionen \\
\textbf{Mana-Kosten (MK):} 8 MP \\
\textbf{Reichweite (RW):} selbst \\
\textbf{Wirkungsdauer (WD):} QS x 3 Minuten \\
\textbf{Zielkategorie:} Zone \\
\textbf{Spezialisierung:} Psychokinetik \\
\textbf{Steigungsfaktor (SF):} C


\subsection{Sumus Elixiere}
\label{chap:sumus_elixiere}
\textbf{Probe (MTW):} KL/IN/IN \\
\textbf{Wirkung:} Durch diesen Zauberspruch kann der Zauberer MP in Lebensmittel oder Zutaten für alchemistische Gebräue, wie Tränke, Öle und Bomben einfließen lassen. Zum einen erhöht er die Haltbarkeit von Lebensmitteln, Zutaten und/oder alchemistischen Gebräuen um die Wirkungsdauer des Zauberspruchs, zum anderen erhält der Alchemist eine Erleichterung von 1 bei der Herstellung eines alchemistischen Gebräus, pro verzauberte Zutat, die er dafür verwendet. \\
\textbf{Zauberdauer (ZD/ZZ):} 4 Aktionen \\
\textbf{Mana-Kosten (MK):} 8 MP \\
\textbf{Reichweite (RW):} Berührung \\
\textbf{Wirkungsdauer (WD):} QS in Jahren \\
\textbf{Zielkategorie:} Objekte \\
\textbf{Spezialisierung:} Objekt \\
\textbf{Steigungsfaktor (SF):} A


\subsection{Transversalis}
\label{chap:transversalis}
\textbf{Probe (MTW):} MU/CH/KO \\
\textbf{Wirkung:} Mittels dieses Zaubers kann ein Held sich und maximal QS x 5 Kilogramm Gegenstände (Kleidung, Waffen usw.) an einen anderen Ort teleportieren. Je näher Gegenstände am Körper liegen (meist die Kleidung), desto eher reisen sie mit dem Zaubernden mit. Sie reisen entweder am Stück mit oder bleiben zurück, werden jedoch nicht zerteilt. Die exakte Auswahl trifft der SL. Der Held muss schon einmal an dem Zielort des Zaubers gewesen sein oder ihn sehen können. Der Ort darf maximal QS x 3 Meilen vom Zaubernden entfernt sein. Wenn er ihn nur sieht, ist die Probe um 2 erschwert. \\
\textbf{Zauberdauer (ZD/ZZ):} 8 Aktionen \\
\textbf{Mana-Kosten (MK):} 8 MP + 1 MP pro Meile (nicht modifizierbar) \\
\textbf{Reichweite (RW):} selbst \\
\textbf{Wirkungsdauer (WD):} sofort \\
\textbf{Zielkategorie:} Objekt, Wesen \\
\textbf{Spezialisierung:} Sphären \\
\textbf{Steigungsfaktor (SF):} C


\subsection{Vogelzwitschern}
\label{chap:vogelzwitschern}
\textbf{Probe (MTW):} MU/IN/GE \\
\textbf{Wirkung:} Innerhalb von QS x 2 Meter um sich herum kann der Zauberer jedes beliebige Geräusch durch ein anderes ersetzen. Die Lautstärke bleibt dabei jedoch unverändert. Eine Stimme kann so etwa deutlich höher oder das Auftreten mit dem Fuß wie leiser Donner oder Fanfarenschall klingen. \\
\textbf{Zauberdauer (ZD/ZZ):} 2 Aktionen \\
\textbf{Mana-Kosten (MK):} 4 MP \\
\textbf{Reichweite (RW):} selbst \\
\textbf{Wirkungsdauer (WD):} QS x 3 Minuten \\
\textbf{Zielkategorie:} Zone \\
\textbf{Spezialisierung:} Illusion \\
\textbf{Steigungsfaktor (SF):} A

