{\let\clearpage\relax\chapter{Magische Fähigkeiten}}
Fähigkeiten können nur von Magiebegabten eingesetzt werden. Bestimmte Flüche können nur von Priestern/innen bestimmter Religionen ausgesprochen werden. Hexer haben sogenannte Hexer-Zeichen, die nur sie verwenden können.


\section{Einfache Fähigkeiten}
Einfache Fähigkeiten, die nicht für den Kampf, sondern für den Alltag gedacht sind. Z.B. Licht oder Feuer erzeugen. Wenn keine Reichweite (RW) angegeben ist, wird zum Wirken Körperkontakt zum Ziel benötigt.

\begin{longtable}{|p{4cm}|p{0.8cm}|p{2.2cm}|p{0.8cm}|p{0.8cm}|p{2.2cm}|p{0.8cm}|p{2.5cm}|}
\hline
\textbf{Einfache Fähigkeit} & \textbf{MK} & \textbf{MTW} & \textbf{SF} & \textbf{ZZ} & \textbf{MTP} & \textbf{RW} & \textbf{WD} \\

\hline
\textbf{Entzünden} & 1 & KL/IN/FF & A & - & - & 2m & - \\ \hline
\multicolumn{1}{r}{} & \multicolumn{7}{|p{13cm}|}{\textit{Entfacht eine Flamme um etwas (z.B. eine Fackel) zu entzünden.}} \\

\hline
\textbf{Wasserbindung} & 4 & IN/FF/FF & B & 2min & - & - & - \\ \hline
\multicolumn{1}{r}{} & \multicolumn{7}{|p{13cm}|}{\textit{Zieht aus der Umgebung Feuchtigkeit um Trinkwasser zu gewinnen. Der Gewinn hängt von der Umgebung ab. Kann auch verwendet werden um verunreinigtes Wasser zu reinigen.}} \\

\hline
\textbf{Materialisierung} & 4 & IN/FF/FF & A & 5min & - & 20m & 5min$\cdot$QS \\ \hline
\multicolumn{1}{r}{} & \multicolumn{7}{|p{13cm}|}{\textit{Beschwört ein materialisiertes Objekt für eine kurze Zeit. Die maximale Größe des Objektes hat von der QS ab.}} \\
\multicolumn{1}{r}{} & \multicolumn{7}{|p{13cm}|}{\textit{\textbf{QS I}: Tragbare Objekte wie z.B. Schmuck, Waffen, uws...}} \\
\multicolumn{1}{r}{} & \multicolumn{7}{|p{13cm}|}{\textit{\textbf{QS II}: Objekte die man nur zu mehreren tragen kann wie z.B. Truhen, Schränke, Wände, Türen usw...}} \\
\multicolumn{1}{r}{} & \multicolumn{7}{|p{13cm}|}{\textit{\textbf{QS III}: Objekte die man nicht tragen kann wie z.B. Brücken, Boote, usw...}} \\
\multicolumn{1}{r}{} & \multicolumn{7}{|p{13cm}|}{\textit{\textbf{Ausnahmen}: Lebensmittel, magische Gegenstände (z.B. Runen, Amulette)}} \\

\hline
\textbf{Illusion} & 2 & KL/KL/IN & A & 5min & - & 2m & $\infty$ \\ \hline
\multicolumn{1}{r}{} & \multicolumn{7}{|p{13cm}|}{\textit{Erzeugt eine statische Illusion, die sich nicht von der Physik beeinflussen lässt und auch keinen physikalischen Einfluss auf die Umgebung hat. (wie eine optische Täuschung)}} \\

\hline
\textbf{Druckwelle} & 4 & KL/MU/KK & B & 2AK & 3+QSW4 & 2+QS & - \\ \hline
\multicolumn{1}{r}{} & \multicolumn{7}{|p{13cm}|}{\textit{Wirft alle Gegner kegelförmig vom Anwender aus weg, schmeißt sie zu Boden und fügt ihnen Schaden zu. Jeder getroffene zweibeinige Gegner bekommt den Status "`Liegend"'.}} \\

\hline
\textbf{Eisscherbe} & 3 & KL/IN/FF & B & 1AK & QS$\cdot$3+1W6 & 25m & - \\ \hline
\multicolumn{1}{r}{} & \multicolumn{7}{|p{13cm}|}{\textit{Der Zauberwirkende beschwört eine Eisscherbe, die er auf ein Ziel abfeuert. Das Ziel hat lediglich einen Versuch auf Ausweichen, der um QS$\cdot$2 erschwert ist. Chance dem Ziel eine leichte Blutung zuzufügen: EW max.: 10\%, EW min.: 5\%}} \\

\hline
\textbf{Findibus} & 3 & IN/IN/FF & B & 2AK & - & - & 1h pro QS \\ \hline
\multicolumn{1}{r}{} & \multicolumn{7}{|p{13cm}|}{\textit{Der Zaubernde nimmt Bindung mit einem unbelebten Objekt auf. Solange die Bindung aktiv ist, weiß er exakt wo sich dieses Objekt befindet, solange es sich in der selben Welt befindet. Das Objekt darf nicht größer als 2x2x2 Meter groß sein.}} \\

\hline
\textbf{Magische Barriere} & 4 & IN/GE/KO & A & 2AK & - & - & 8AK \\ \hline
\multicolumn{1}{r}{} & \multicolumn{7}{|p{13cm}|}{\textit{Den Zaubernden umhüllt ein magischer Schutzschild, der den Rüstschutz um die QS erhöht.}} \\

\hline
\textbf{Aura Vitalis} & 2 & KL/IN/MU & A & 1AK & - & QS$\cdot$50m & 1AK \\ \hline
\multicolumn{1}{r}{} & \multicolumn{7}{|p{13cm}|}{\textit{Wirkt einen Spruch, der alles Leben im Geiste aufdeckt und nach Wärmebild markiert auf eine Entfernung von QS x 50 Meter. Erfasst Lebewesen durch Wände und andere Hindernisse hindurch. Der Wirkende erhält Informationen über Größe, Körperbau und Bewegung für 30 Sekunden (1AK).}} \\

\hline
\textbf{Magiesicht} & 2 & IN/IN/FF & A & 1AK & - & - & 6AK \\ \hline
\multicolumn{1}{r}{} & \multicolumn{7}{|p{13cm}|}{\textit{Lässt den Anwender jede Art von Magie erkennen.}} \\

\cline{2-8}
\caption{Einfache Fähigkeiten}
\label{tab:EinfacheSkills}
\end{longtable}


\section{Mächtige Fähigkeiten}

\begin{longtable}{|p{4cm}|p{0.8cm}|p{2.2cm}|p{0.6cm}|p{1cm}|p{2.2cm}|p{0.9cm}|p{2.2cm}|}
\hline
\textbf{Mächtige Fähigkeit} & \textbf{MK} & \textbf{MTW} & \textbf{SF} & \textbf{ZZ} & \textbf{MTP} & \textbf{RW} & \textbf{WD} \\

\hline
\textbf{Flammenball} & 10 & KL/KL/KO & B & 4AK & 10$\cdot$QS+2W6 & 50m & - \\ \hline
\multicolumn{1}{r}{} & \multicolumn{7}{|p{13cm}|}{\textit{Wirft eine Flammenkugel gerade aus auf ein Ziel. Das Ziel kann auch ein Objekt sein. In diesem Fall fängt das Objekt Feuer, wenn es aus einem brennbaren Material besteht.}} \\

\hline
\textbf{Wirbelsturm} & 10 & KL/MU/KO & B & 4AK & QSW12 & 100m & 20AK \\ \hline
\multicolumn{1}{r}{} & \multicolumn{7}{|p{13cm}|}{\textit{Erzeugt am Zielort einen Wirbelsturm mit einem Durchmesser von 10m, der alles und jeden in seinem Wirkungsbereich wegschleudert und Schaden austeilt. Getroffene Personen bekommen den Status "`Liegend"'. Kann nur im freien benutzt werden.}} \\

\hline
\textbf{Geisterbeschwörung} & 10 & KL/MU/IN & B & 6AK & s.u. & 1m & - \\ \hline
\multicolumn{1}{r}{} & \multicolumn{7}{|p{13cm}|}{\textit{Beschwört eine Erscheinung, die für die Gruppe kämpft.}} \\
\multicolumn{1}{r}{} & \multicolumn{7}{|p{13cm}|}{\textit{\textbf{LP}: 20+5$\cdot$QS, \textbf{TW}: 16, \textbf{TP}: 6+QSW6, 1 Angriff pro AK}} \\

\hline
\textbf{Portal} & 4 & KL/KL/IN & B & 6AK & - & 1m & 1min$\cdot$QS \\ \hline
\multicolumn{1}{r}{} & \multicolumn{7}{|p{13cm}|}{\textit{Errichtet ein Portal, das jeden der durchgeht zu dem festgelegten Ziel teleportiert. Das Ziel des Portals wird vor seiner Erschaffung festgelegt und muss gewisse Kriterien erfüllen: Das Ziel muss vom aktuellen Standort aus sichtbar sein oder der Anwender muss schon einmal am Ziel gewesen sein. Durch das Portal passen nur Personen, es werden aber auch Objekte teleportiert, solange sie durchpassen.}} \\
\multicolumn{1}{r}{} & \multicolumn{7}{|p{13cm}|}{\textit{\textbf{Maximale Entfernung} des Ziels: 150m $\cdot$ QS}} \\

\hline
\textbf{Angst} & 4 & KL/CH/KO & B & 5AK & - & 2m & 30min$\cdot$QS \\ \hline
\multicolumn{1}{r}{} & \multicolumn{7}{|p{13cm}|}{\textit{Der Verfluchte verspürt jedem und allem Gegenüber große Angst. Kann nicht auf Monster gewirkt werden.}} \\

\hline
\textbf{Runenzauber} & 6 & KL/CH/IN & C & 30min & - & 1m & $\infty$  \\ \hline
\multicolumn{1}{r}{} & \multicolumn{7}{|p{13cm}|}{\textit{Ermöglicht die Herstellung von Glyphen und das Einbetten/Entfernen von Glyphen in/aus Ausrüstungsgegenständen.}} \\

\hline
\textbf{Ewige Dienerschaft} & 25 & KL/CH/IN & C & 30min & - & 10km & 12h pro QS  \\ \hline
\multicolumn{1}{r}{} & \multicolumn{7}{|p{13cm}|}{\textit{Zwingt sein Ziel zur Dienerschaft für eine bestimmte Zeit. Das Ziel lässt sich durch Befehle lenken, kann aber nicht zu etwas gezwungen werden, das seinem Wesen widerspricht (Z.B. eine Priesterin der Melitele würde niemanden Töten). Der Verzauberte muss die Befehle hören und verstehen können. Benötigt einen persönlichen Gegenstand des Ziels. Magiebegabte Ziele können einen Widerstandswurf ablegen, um dem Fluch zu widerstehen.}} \\

\hline
\textbf{Dolores Communicare} & 5 & IN/MU/GE & B & 1AK & - & 50m & max. 10AK  \\ \hline
\multicolumn{1}{r}{} & \multicolumn{7}{|p{13cm}|}{\textit{Verbindet die Seele des Zaubernden mit einem humanoiden Ziel. Den ersten Schaden, den einer der beiden nimmt, wird in gleicher Höhe ohne Abzüge und Resistenzen auch dem anderen zugefügt. Wird der Schaden jedoch vom Zaubernden auf das Ziel geworfen, dann muss dieser einen Wurf auf Willenskraft ablegen, erschwert um die QS, um nicht eine Betäubungsstufe zu erhalten. Der Zauber hält bis zu 5min an, falls keiner der beiden in diesem Zeitraum Schaden nimmt.}} \\

\cline{2-8}
\caption{Mächtige Fähigkeiten}
\label{tab:MächtigeSkills}
\end{longtable}


\section{Auren}
Alle möglichen Auren (= AoE-Effekte). Können hauptsächlich von Priestern/innen der Melitele verwendet werden. Vor dem Wirken einer Aura muss der Spieler entscheiden, ob die Aura personen-, ort- oder objektgebunden ist. Wenn keine Reichweite (RW) angegeben ist, kann die Aura nur auf sich selber gewirkt werden.

\begin{table}[h]
\begin{center}
\begin{tabular}{|p{4cm}|p{0.8cm}|p{2.2cm}|p{0.8cm}|p{0.8cm}|p{2.2cm}|p{0.8cm}|p{2.5cm}|}
\hline
\textbf{Aura} & \textbf{MK} & \textbf{MTW} & \textbf{SF} & \textbf{ZZ} & \textbf{Radius} & \textbf{RW} & \textbf{WD} \\

\hline
\textbf{Friedensaura} & 7 & KL/CH/CH & B & 5min & 1m$\cdot$QS & - & 5min$\cdot$QS \\ \hline
\multicolumn{1}{r}{} & \multicolumn{7}{|p{13cm}|}{\textit{Alle Lebewesen innerhalb der Aura verspüren völlige Zufriedenheit. Mögliche Erschwernis, wenn sie auf jemanden gewirkt werden soll, der sehr wütend ist. Monster sind von der Aura nicht betroffen (Tiere schon).}} \\

\hline
\textbf{Warnaura} & 1/2h & KL/IN/IN & B & 10min & 10m$\cdot$QS & 3m & 2h pro Mana \\ \hline
\multicolumn{1}{r}{} & \multicolumn{7}{|p{13cm}|}{\textit{Der Erschaffer der Aura nimmt alle Lebewesen innerhalb der Aura wahr (auch im Schlaf).}} \\

\hline
\textbf{Aura der Stärke} & 7 & CH/CH/KK & C & 6min & 10m$\cdot$QS & - & 20min \\ \hline
\multicolumn{1}{r}{} & \multicolumn{7}{|p{13cm}|}{\textit{Alle dem Wirkenden wohlgesinnten Lebewesen erhalten +1KK $\cdot$ QS. Wirkt auch auf Tiergefährten und andere Lebewesen, die Zuneigung verspüren können.}} \\

\hline
\textbf{Aura der Wahrnehmung} & 3 & KL/IN/CH & A & 5min & 2m$\cdot$QS & - & 10min \\ \hline
\multicolumn{1}{r}{} & \multicolumn{7}{|p{13cm}|}{\textit{Verbessert Talente, die mit der Wahrnehmung zusammenhängen. Z.B. Sinnesschärfe, Fährtensuchen.}} \\

\cline{2-8}
\end{tabular}
\end{center}
\caption{Auren}
\label{tab:Auren}
\end{table}

\newpage
\section{Heilzauber}
Alle Fähigkeiten zum Heilen von Wunden. Manche können nur von Anhängern des \textit{Kults der Melitele} verwendet werden. Es gibt keine "`Instant"'-Heals. Das bedeutet, dass Heilung immer pro Runde stattfindet, beginnend in der Aktivierungsrunde $\rightarrow$ die erste Teilheilung findet sofort statt!

\begin{longtable}{|p{4cm}|p{1.1cm}|p{2.2cm}|p{0.8cm}|p{0.8cm}|p{2.2cm}|p{0.8cm}|p{2.2cm}|}
\hline
\textbf{Heilzauber} & \textbf{MK} & \textbf{MTW} & \textbf{SF} & \textbf{ZZ} & \textbf{Heilung} & \textbf{RW} & \textbf{WD} \\

\hline
\textbf{Heilzauber} & 1/2LP & KL/IN/FF & A & 1AK & M$\cdot$2+QSW4 & 5m & 1AK/2LP \\ \hline
\multicolumn{1}{r}{} & \multicolumn{7}{|p{13cm}|}{\textit{Stellt beim Ziel 2LP pro Runde her bis zu einem Maximum von 2$\cdot$M.}} \\
\multicolumn{1}{r}{} & \multicolumn{7}{|p{13cm}|}{\textit{\textbf{Beispiel}: Wenn der Heiler 4 Mana für diese Fähigkeit bezahlt und die QS II erreicht, heilt er sein Ziel um 8LP durch das ausgegebene Mana. Durch die QS II wirft er 2W4. Wenn er damit eine 1 und eine 3 würfelt, heilt er also zusätzlich 4LP $\rightarrow$ insgesamt also +12LP. Bei +2LP pro Runde bedeutet das, dass der Heileffekt 6 Runden lang auf dem Ziel aktiv ist, wobei die ersten 2LP noch in der selben Runde geheilt werden. (In den nächsten fünf Runden werden dann die restlichen LP regeneriert.)}} \\

\hline
\textbf{Wundenheilung} & 5 & KL/IN/FF & A & 1AK & - & - & 1-3AK \\ \hline
\multicolumn{1}{r}{} & \multicolumn{7}{|p{13cm}|}{\textit{Heilt Wunden in Abhängigkeit der erreichten QS. Kann z.B. Schmerz entfernen.}} \\
\multicolumn{1}{r}{} & \multicolumn{7}{|p{13cm}|}{\textit{\textbf{QS I}: leichte Wunden, WD: 1AK}} \\
\multicolumn{1}{r}{} & \multicolumn{7}{|p{13cm}|}{\textit{\textbf{QS II}: mittlere Wunden, WD: 2AK}} \\
\multicolumn{1}{r}{} & \multicolumn{7}{|p{13cm}|}{\textit{\textbf{QS III}: schwere Wunden, WD: 3AK}} \\

\hline
\textbf{Reanimation} & 20 & KL/CH/IN & A & 6AK & - & 1m & - \\ \hline
\multicolumn{1}{r}{} & \multicolumn{7}{|p{13cm}|}{\textit{Erweckt eine kürzlich verstorbene Personen (10min/20AK) zum Leben. D.h., der Zauber muss spätestens 14AK nach dem Todeszeitpunkt benutzt werden.}} \\

\cline{2-8}
\caption{Heilzauber}
\label{tab:Heilzauber}
\end{longtable}

%\section{Hexer-Zeichen}
%Zeichen, die nur von Hexern gewirkt werden können. Igni, Yrden, Quen, Axii, Aard.