{\let\clearpage\relax\chapter{Glyphen}}
Glyphen sind magischen Steine mit Runen, die in die Ausrüstung eingebettet werden können. Verzauberte Ausrüstungen können auch von nicht-magiebegabten Wesen benutzt werden. Es gibt offensive Glyphen für Waffen und defensive Glyphen für Rüstungen. Glyphen können nicht nur hergestellt werden, sonder auch bei bestimmten Händlern erworben oder gefunden werden. Diese haben jedoch eine festgelegte Qualitätsstufe.

\section{Herstellung}
Zur Herstellung von Glyphen werden "`magische Steine"' benötigt, welche dann mit einem Runenzauber verzaubert werden können ($\rightarrow$ Glyphen). Abhängig von der Stärke des Runenzaubers erhält die Glyphe eine Qualitätsstufe (QS). Die QS einer Glyphe kann nachträglich nicht geändert werden. Das gilt auch für Glyphen, die bei Händlern erworben oder in der Welt gefunden werden.

\section{Anwendung}
Glyphen können nur mit einer passenden Fähigkeit hergestellt und in Ausrüstung eingebettet/entfernt werden. In jeden Ausrüstungsgegenstand kann nur eine Glyphe eingebettet werden. Die Entfernung von Glyphen aus einem Gegenstand ist um die QS der Glyphe -1 erschwert. Beim erfolgreichen Entfernen einer Glyphe erhält man die Glyphe wieder. Ist das Entfernen nicht erfolgreich wird sie zerstört. Schilde können nicht verzaubert werden.

Glyphenzauber, z.B. \textit{Runenzauber}, sind mächtige Verzauberungen, die zur Ausführung einige Minuten benötigen. Anders als Ritualzauber, können sie jedoch überall gewirkt werden. Dennoch erhält der Anwender einen Bonus, wenn er den Glyphenzauber wie bei einem Ritual vorbereitet. 

Der Bonus wird dabei vom Spielleiter festgelegt. Üblicherweise liegen die Boni zwichen +2 und +6. Entgegen der Grundregel, dass sich Boni nicht auf die Qualitäts- bzw. Fertigkeitsstufe auswirken können, kann sich dieser Bonus posititv auf die Qualitätsstufe auswirken. Das bedeutet im Gegenzug auch, dass Mali auf Glyphenzauber sich negativ auf die QS auswirken können. Mali können dann vom Spielleiter erteilt werden, wenn der Anwender unter Druck steht oder körperlich bzw. psychisch behindert wird.

\section{Effekte}
Glyphen fügen der Ausrüstung einen (oder mehrere) Effekte hinzu. Die Effekt-Wahrscheinleichkeit (EW), wann der Effekt einer Glyphe ausgelöst wird, hängt vom zugefügten bzw. erlittenen Schaden ab (tatsächlicher, nicht theoretischer Schaden, d.h. Rüstungsschutz wird vorher abgezogen). Der zugefügte bzw. erlittene Schaden gibt die EW in \% an. Dieser Wert muss durch einen W100 (W00+W10) unterworfen werden. Die maximale Wahrscheinlichkeit kann durch die Glyphe festgelegt sein. 

\textit{Beispiel: Ich habe eine Waffe mit einer Crit-Glyphe, die die maximale EW auf 10\% und die minimale EW auf 5\% beschränkt. Wenn ich jetzt einen Gegner angreife und er (durch meinen Würfelwurf) 12 Schaden bekommt (Rüstungsschutz des Gegner wurde dabei berücksichtigt), wäre das eigentlich 12\% EW. Da die Glyphe aber maximal 10\% erlaubt, muss ich mit einen W100 die 10 unterwürfeln (0-9 $\rightarrow$ Erfolg; 10-99 $\rightarrow$ Fehlschlag). Wenn ich beim selben Beispiel nur 8 Schaden gemacht hätte, hätte ich auch nur 8\% EW, trotz der maximalen EW von 10\%. Wenn ich weniger Schaden als die minimale EW mache, ist die EW die minimale EW. D.h., wenn ich (im selben Beispiel) nur 3 Schaden mache, habe ich trotzdem 5\% EW.}

\begin{longtable}{|p{3cm}|p{12.5cm}|}
\hline
\textbf{Effekt} & \textbf{Beschreibung} \\

\hline
Crit & +QS TP bei jedem Angriff. \\
& \textbf{Effekt:} Erhöht den Schaden (TP) um 100\% (min. +2TP) \\
\hline
Silberschaden & +1+QS TP bei jedem Angriff gegen Monster (und Tiere/Bestien) \\
& \textbf{Effekt:} Erhöht den Schaden (TP) gegen Monster (und Tiere/Bestien) um 150\% (min. +2TP) \\
\hline
Durchschlagen & +2 $\cdot$ QS RD bei jedem Angriff \\
\hline
Stärke & +2 $\cdot$ QS TP bei jedem Angriff. \\
& \textbf{Effekt:} Fügt eine Wunde hinzu. Die Schwere der Wunde hängt von der QS ab. \\
& \textbf{QS I+II:} Leichte Wunde \\
& \textbf{QS III:} Mittlere Wunde \\
& \textbf{QS IV+:} Schwere Wunde \\
\hline
Agilität & Ziele von Nahkampfangriffen bekommt -QS BL bei jedem Angriff. \\
& \textbf{Effekt \#1:} +1BL für den Rest der Runde \\
& \textbf{Effekt \#2:} Ziel von Nahkampfangriffen bekommt -1BL für den Rest der Runde \\
\hline
Leichtgewichtig & +QS INI, +QS AW, -2BE (wirkt nicht auf Rüstungen mit +3BE oder mehr) \\
\hline
Steinhart & +2+QS RS. \\
\hline
Elastisch & +QS RS gegen Pfeile. \\
& \textbf{Effekt:} Pfeil prallt ab und verursacht keinen Schaden. \\
\hline

\caption{Glyphen-Effekte}
\label{tab:GlyphenEffekte}
\end{longtable}


\section{Offensive Runen}
\begin{longtable}{|l|l|l|l|l|}
\hline
\textbf{Rune} & \textbf{Effekt} & \textbf{EW min.} & \textbf{EW max.} & \textbf{Anmerkung} \\

\hline
Crit-Rune & Crit & 5*QS & 10*QS & - \\ 
\hline
Silber-Rune & Silberschaden & 5*QS & 10*QS & - \\ 
\hline
Durchschlag-Rune & Durchschlagen & 100 & 100 & - \\ 
\hline
Stärke-Rune & Stärke & 5*QS & 10*QS & - \\ 
\hline
AGI-Rune & Agilität & 15*QS & 100 & - \\ 
\hline

\caption{Offensive Runen}
\label{tab:OffensiveRunen}
\end{longtable}


\section{Defensive Runen}
\begin{longtable}{|l|l|l|l|l|}
\hline
\textbf{Rune} & \textbf{Effekt} & \textbf{EW min.} & \textbf{EW max.} & \textbf{Anmerkung} \\

\hline
Feder-Rune & Leichtgewichtig & 100 & 100 & verbessert die Stats \\ 
\hline
Stein-Rune & Steinhart & 100 & 100 & verbessert die Stats \\ 
\hline
Gummi-Rune & Elastisch & 10*QS & 10*QS+20 & - \\ 
\hline

\caption{Defensive Runen}
\label{tab:DefensiveRunen}
\end{longtable}