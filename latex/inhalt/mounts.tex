{\let\clearpage\relax\chapter{Reittiere}}
Reittiere können für den schnelleren Transport verwendet werden während Begleiter die unterschiedlichsten Funktionen haben können, z.B. die Unterstützung im Kampf.

Die normale Reichweite beträgt 12m. Das ist die (maximale) Geschwindigkeit, die im \textit{Schritt} erreicht wird. Im \textit{Trab} kann sich das Reittier 1,5-fach (18m pro Aktion) und im \textit{Galopp} doppelt so weit (24m) bewegen. Durch hohe Lasten, die das Reittier tragen oder ziehen muss, kann sich die Belastung (BE) des Pferdes erhöhen, was die (maximale) Reichweite des Pferdes, genauso wie bei Personen einschränkt. Jede BE verringert die maximale Reichweite pro Aktion um 1m ($\rightarrow$ -1m im \textit{Schritt}, -1,5m im \textit{Trab} und -2m im \textit{Galopp}).

\section{Allgemeines zum berittenen Kampf}
Pferde und Ponys dienen im Kampf bereitwillig als Reittiere. Reittiere, die kein Kampftraining haben, reagieren im Kampf verängstigt. Das kann dazu führen, dass sie während dem Kampf in Panik verfallen und versuchen den Reiter abzuschmeißen. Im Normalfall benötigt man mindestens eine Hand zum Reiten. 

Im berittenen Kampf kannst du die Lauf-Aktion deines Reittieres, zusätzlich zu deiner eigenen Aktion verwenden. Dadurch kannst du dich in einer Aktion fortbewegen und angreifen. Wie weit du dich in einer Aktion fortbewegen kannst, hängt von der Gangart des Pferdes ab. 

\subsection{Wenn dein Reittier im Kampf fällt} 
Wenn dein Reittier stürzt, muss dir eine Wurfprobe auf Reiten gelingen, um weich zu fallen und keinen Schaden zu erleiden. Misslingt der Wurf, erleidest du 1W6 Schadenspunkte.

\subsection{Wenn dein Reittier in Panik verfällt} Wenn dein Reittier in Panik verfällt, muss dir eine Wurfprobe auf Reiten gelingen, um das Pferd unter Kontrolle zu bringen. Wenn dein Reittier kein Kampftraining hat, wird die Probe um 2 erschwert. Wenn du einen Militärsattel verwendest, wird die Probe um 2 erleichtert. Misslingt der Wurf wirst du zu Boden geschmissen und erleidest 1W6 Schaden. Gelingt dir der Wurf behältst du das Pferd unter Kontrolle, kann sich in diesem Zug aber nicht mehr fortbewegen.

\subsection{Wann verfällt dein Reittier in Panik?} 
Wann ein Pferd in Panik verfällt, hängt vom SL ab. Dabei sollen bestimmte Faktoren berücksichtigt werden: Wie viele Personen/Monster befinden sich auf dem Schlachtfeld? Ist dein Reittier kampferfahren? (Wenn das Reittier kampferfahren ist, sollte es nur in seltenen Fällen während einem Kampf in Panik ausbrechen!) Trägt das Reittier Scheuklappen?

Im Normalfall verfällt ein Reittier ohne Kampftraining nach 3 bis 5 Aktionen oder wenn der Reiter angegriffen wird, in Panik.

\subsection{Wenn du bewusstlos wirst} 
Wenn du bewusstlos geschlagen wirst, hast du eine Chance von 50\% im Sattel zu bleiben (75\%, wenn du einen Militärsattel benutzt). Wenn du stürzt, erleidest du 1W6 Schadenspunkte. Ohne deine Führung meidet das Reittier den Kampf $\rightarrow$ er flieht.

Wie der berittene Fern-, Nah- und Zauberkampf im Detail funktioniert und was man dabei zu beachten hat, wird in den folgenden Unterkapiteln beschrieben.


\section{Beritterner Nahkampf}
Beim berittenen Nahkampf kann der Reiter nur mit einer Hand zuschlagen. D.h., dass Zweihandwaffen nicht benutzt werden können. (Bögen können benutzt werden. Siehe Kapitel \ref{chap:berittener_fernkampf}.) Grundsätzlich ist es möglich, dass der Spieler beim Reiten beide Hände benutzen kann. In diesem Fall muss der Spieler eine Probe auf \textit{Reiten} ablegen, um das Pferd erfolgreich mit den Knien zu lenken. Das ist nicht im \textit{Galopp} möglich. Wenn das Reittier kein Kampftraining hat, wird die Probe um 2 erschwert. Die Probe entfällt, falls der Spieler die Kampfsonderfertigkeit \textit{Berittener Schütze} hast.

\subsection{Geschwindigkeitsboni}
Bei Nahkampfangriffen vom Pferd aus, verursacht der Spieler zusätzlichen Schaden, abhängig von der Reitgeschwindigkeit. \textbf{Der Schaden wird um GS/2 erhöht.} Für den \textit{Galopp} gilt eine zusätzliche Regel: Ist das Ziel ein Humanoid, hat der Angriff eine 50\% Chance den Gegner sofort zu töten oder zumindest schwer zu verletzen, wenn es das Ziel nicht schafft auszuweichen oder zu blocken. Die konkrete Auswirkung bei einem Treffer aus dem \textit{Galopp} kann der SL situationsabhängig entscheiden. \textbf{Berittene Einheiten, die aus dem \textit{Galopp} heraus angreifen werden mit einer Erschwernis von -4 pariert/gekontert. Blocken und Ausweichen wird nicht erschwert.}

\subsection{Angriffe auf kleinere Ziele}
Wenn der Spieler eine Kreatur angreift, die kleiner als sein Reittier ist, erhält er einen \textbf{Bonus von +4TP auf seine Nahkampfangriffe}, weil er aus einer höheren Position angreift.


\section{Berittener Zauberkampf}
Solange sich das Reittier nicht im \textit{Galopp} befindet, kann wie gewohnt gezaubert werden. Auch beim berittenen Zauberkampf gilt, dass eine Hand frei sein oder einen Zauberstab halten muss, während die zweite Hand die Zügel hält. Im \textit{Galopp} ist zaubern nicht möglich.

Wenn das Pferd in Panik verfällt wird der aktuelle Zauber abgebrochen.


\section{Berittener Fernkampf}
\label{chap:berittener_fernkampf}
Wenn das Reittier steht, gibt es keine Mali. Wenn das Reittier geht (im \textit{Schritt}) wird der Angriff um 4 erschwert (-4TW). Im \textit{Galopp} wird der Angriff um 8 erschwert (-8TW). Im \textit{Trab} sind nur Glückstreffer möglich (eine 1 bei einem W20). Da für die Benutzung von Bögen zwei Hände nötig sind, muss der Spieler zusätzlich eine Probe ablegen. Das ist im Kapitel \ref{chap:freihaendig_reiten} genauer beschrieben. Weitere Boni/Mali werden genauso wie beim Fernkampf zu Fuß berechnet.

\subsection{Freihändig Reiten} 
\label{chap:freihaendig_reiten}
Wenn der Spieler freihändig reiten will, z.B. um seinen Bogen zu benutzen, muss er eine Probe auf \textit{Reiten} ablegen, um das Pferd erfolgreich mit den Knien zu lenken. Wenn sein Reittier kein Kampftraining hat, wird die Probe um 2 erschwert. Die Probe entfällt, falls der Spieler die Kampfsonderfertigkeit \textit{Berittener Schütze} hat.

\section{Weitere Sonderregeln}
Vom Reittier aus können keine Spezialmanöver, wie z.B. parieren ausgeführt werden.

Mit Schilden lassen sich nur Angriffe von vorne und der Seite des Schildarms blocken. Angriffe von der Seite des Waffenarms kann man nur mit der dort geführten Waffe blocken oder ihnen ausweichen.

Ausweichen auf Reittieren ist immer um 2 erschwert, außer man springt vom Pferd runter. 

Allgemein gilt, dass Rüstungen zu Pferde weniger behindern, da das Pferd einen Teil des Gewichtes trägt. Die Erschwernis auf Kampfproben durch \textit{Belastung} ist um 1 erleichtert (-1BE).

Das Wechseln von \textit{Schritt} zu \textit{Trab} zu \textit{Galopp} während einem Kampf muss immer mit einer Probe auf \textit{Reiten} bestätigt werden. Der Wechsel von \textit{Schritt} zu \textit{Galopp} ist nur möglich, wenn das Pferd den \textit{Carrière} beherrscht. Der \textit{Carrière} ist ein spezieller Geloppsprung, der einem Pferd antrainiert werden muss. Den Sprung beherrschen Pferde mit Kampftraining aber auch Renn- und Kutschpferde.

Wenn das Reittier verletzt wird, verfällt es sofort in Panik. 