% Definition von globalen Parametern, die derzeit auf der Titelseite verwendet werden. 
% Keine Sonderzeichen verwenden!
        % ä|Ä -> {\"a|A}, ö|Ö -> {\"o|O}, ü|Ü -> {\"u|U}, ß -> {\ss}

%----------- Allgemeine Angaben -------------------------------------------------------------
\newcommand{\praktikumTitel}{Grundregelwerk v2.1}
\newcommand{\projektTitel}{Das Pen \& Paper Rollenspiel im \textit{Witcher}-Universum}
\newcommand{\projektTitelKurz}{Das Pen \& Paper Rollenspiel im \textit{Witcher}-Universum}

\newcommand{\fakultaet}{<Name der Fakult{\"a}t>}
\newcommand{\studiengang}{<Studiengang>}
\newcommand{\schule}{The Witcher P\&P RPG}
\newcommand{\ort}{Schriesheim}

%----------- Angaben zur Person 1 -----------------------------------------------------------
\newcommand{\personName}{Marcel Ortega O{\ss}wald}
\newcommand{\personMatrnr}{1223456}

%----------- Angaben zur Person 2 (optional) ------------------------------------------------
\newcommand{\zweitepersonName}{Peter Mustermann}
\newcommand{\zweitepersonMatrnr}{1226543}

%----------- Angaben zu Betreuer 1 ----------------------------------------------------------
\newcommand{\betreuerName}{Prof. Dr. Peter M{\"u}ller}
\newcommand{\betreuerSchule}{\schule} % Bei Befarf "\schule" durch andere Hochschule/Universität ersetzen

%----------- Angaben zu Betreuer 2 (optional) -----------------------------------------------
\newcommand{\zweiterbetreuerName}{Herbert Gie{\ss}er}
\newcommand{\zweiterbetreuerSchule}{\schule} % Bei Befarf "\schule" durch andere Hochschule/Universität ersetzen

%-------------- Für Tabellen -----------------------------------------------------
\newcommand{\brcell}[2][l]{ % brcell = breakeable cell
  \begin{tabular}[t]{@{}l@{}}#2\end{tabular}}
 
%-------------- Tabs -------------------------------------------------------------
\newcommand\tab[1][1cm]{\hspace*{#1}}
